\section{Technologien}
\subsection{Anforderungen}\phantomsection\label{technologies:demands}
Abgelitten aus der Problemliste werden hier die Probleme umformuliert als Anforderungen dargestellt und in absteigender Reihenfolge nach Relevanz in Bezug auf die gegebene Aufgabenstellung aufgeführt.

Die Technologien dienen den Anforderungen, sollten sie:\ 
\begin{enumerate}
    \item kompilierbare Dokumente erzeugen
    \item alle Abschnitte in Dokumenten übersetzen
    \item kontextuell terminologisch richtige Übersetzungen wählen (die richtigen Lexeme/Wörter treffen)% Hier Lexeme, da z.B. 'Lexeme/Wörter' als eine Zeichenkette eingelesen wird 
    \item den Kontext selbstständig aus den wörtlichen und erreichbaren (lokalen) Informationen (Dateien) ablesen können
    \item den Kontext aus den mathematischen, graphischen, tabellarischen,\ldots Inhalten einer Datei ablesen können
    \item den Kontext aus externen Verweisen (Links) erfassen können (Lokal, als auch Web)
    \item \ldots
\end{enumerate}
\subsection{Denkbare Ansätze}% 
Alle Lösungswege und Workflows, die ich mir vorstellen kann und denken konnte. Definiert evtl.\ Rollen,
\subsection{Existierende Ansätze}% 
Alle Technologien, die diese Rolle (n) in den entsprechenden Ansätzen füllen könnten.
\subsection{Tests}% 
logischerweise:\ In den denkbaren Ansätzen schon gegenargumentieren, was unsinnig ist und warum. Reduziert die Menge an zu testenden Lösungen.
\subsection{Grenzen der Lösungen}% Mal schauen
\subsection{Takeaways}% Was geht wo noch besser?