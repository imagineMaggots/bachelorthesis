\subsection{Testverfahren}
Entlang der zuvor geschilderten Problemfälle ließen sich theoretisch gesehen unendlich viele explizite Fehlerbeispiele produzieren, welche ihrerseits inhaltlich einem \textit{Lorem Ipsum} gleichen könnten (\enquote{Lorem Ipsum} bezieht sich auf ein häufig genutztes Beispiel, wenn es darum geht Texte und deren graphische Aufbereitung zu visualisieren. Dieser Platzhalter-Text sieht aus wie Latein, ist inhaltlich aber sinnfrei (~\cite{https://www.loremipsum.de/ueber_lorem_ipsum.html, https://support.microsoft.com/de-de/topic/hinweise-zum-text-lorem-ipsum-dolor-sit-amet-in-der-hilfe-von-microsoft-word-bf3b0a9e-8f6b-c2ab-edd9-41c1f9aa2ea0})). Da man bei der Nutzung von \TeX{} aber immer die hauptsächliche Nutzung für wissenschaftliche/mathematische Kontexte im Vordergrund belassen sollte, werden insbesondere die auf ArXiV verfügbaren \TeX{}-Quellcodes von Interesse (wobei eine Kompilation von älteren Texten der Art durch~\cite{https://info.arxiv.org/help/bulk_data.html} zur Verfügung gestellt sind). Dies bedeutet, dass zunächst solche Quelltexte getestet werden, und inwiefern sie \enquote{richtige} Übersetzungen liefern. Dem hierigen Autor ist ein Bestätigen in dieser Hinsicht nur für das Sprachpaar EN-DE möglich, allerdings existieren Technologien (Methoden), wie (Sacre-) BLEU oder TeXBLEU, um algorithmische, maschinelle Bewertungsgrundlagen zu schaffen.

Diese Methodik bedient allerdings bisher nur eine Auswertung der rein sprachlichen Richtigkeit ggb.\ menschlichen Sprachen und der Maschinensprache \TeX{}. Auf diese Art und Weise werden einige spezifischere Fälle und \enquote{Ausnahmen} nicht näher betrachtet, wodurch gezielt auf diese geprobt werden muss. 

Neben den Bulk-Datensätzen von ArXiV entstanden daher auch eigene, kleinere Beispiele, welche dem Anhang~\ref{appendix:c} zu entnehmen sind.

\subsection{Tests}
\subsubsection{Erzielte Werte}% Halt die BLEU-Scores, sowie spezifische Beispiele, an welchen die Ansätze scheitern, und so, tabellarisch auflisten
\subsubsection{Auswertung}% Allgemeine Performance
\subsubsection{Übrige Probleme}% Könnte evtl. unterschiedlich für verschiedene spezifischere Technologien werden.