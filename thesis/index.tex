%	Größen in TeX:
%		mm, cm, ... uninteressant
%		1 em = Schriftgröße, ursprünglich
%		1 ex = Schriftgröße, skaliert auf die Box in welcher sich die Schrift befindet
%		1 pt = ein Pixel nur kann man nicht von echten Pixeln wie auf einem Bildschirm sprechen
%		\textwidth, \paperwidth, ...

\documentclass[11pt,toc,multi=tcblisting]{article}
\renewcommand{\familydefault}{\sfdefault}
\setcounter{secnumdepth}{6}

\usepackage[left=2cm,right=2cm]{geometry}
%\usepackage[document]{ragged2e} % Damit alles: Links aligned (der linkeste Punkt eines jeden Buchstaben ist auf einer Linie für alle zeilen) und so viel Platz nach rechts ausnutzt, wie möglich (nein, macht TeX nicht standardmäßig); muss man nicht für das gesamte Dokument machen (wie hier), geht auch per \raggedright oder \RaggedRight für/vor Paragraphen; gibt auch Umgebungen für rechtsbündigen Text https://de.overleaf.com/learn/latex/Text_alignment
\usepackage[dvipsnames]{xcolor}
\usepackage{booktabs,array}
\usepackage[author={Hendrik Theede}]{pdfcomment}
\usepackage{tabularx}


\usepackage[utf8]{inputenc}
\usepackage[T1]{fontenc}

\usepackage{inconsolata}
\usepackage[htt]{hyphenat}
\usepackage{listings}
% Note: lstset does not allow unnecessary linebreaks
\lstdefinestyle{texstyle}{
	% Colors
	backgroundcolor=\color{white},
    commentstyle=\color{SeaGreen},
    keywordstyle=\color{Magenta},
    numberstyle=\tiny\color{red},
    stringstyle=\color{Fuchsia},
    basicstyle=\ttfamily\footnotesize\color{black},
	% Linebreaks and Whitespace
    breaklines=true,% Sagt: Bei benötigtem Zeilenumbruch gehen wir in eine neue Zeile
	%breakatwhitespace=true,   
	showspaces=false,                
    showstringspaces=false,
    showtabs=false,
	frame=single,
	xleftmargin=13pt,
	aboveskip=13pt,
	%
	% Indentation
	tabsize=2,
	breakindent=13pt,
    framesep=2mm,
    framerule=0mm,
    abovecaptionskip=5mm,
    aboveskip=\baselineskip,
    belowskip=\baselineskip,
	lineskip=1ex,
	postbreak=\raisebox{0ex}[0ex][0ex]{\tiny\tiny$\hookrightarrow$}, % \rule[Offset (Y)]{Breite}{Höhe}, hiermit nochmal rumspielen. 
	% Captions
    captionpos=b,                    
    keepspaces=true,    
	% Linenumbers
    numbers=left,      % where to place numbers
    numbersep=0pt,     % distance to the (here) left   
    % Encodings
	inputencoding=utf8,
	extendedchars=true,
	literate=
		{ä}{{\"a}}1 {ö}{{\"o}}1 {ü}{{\"u}}1 {Ä}{{\"A}}1 {Ö}{{\"O}}1 {Ü}{{\"U}}1
}
\lstset{style=texstyle}



%%%%%%% Vorsicht, evtl Redundanzen im Code!!!
%%% Benötigt noch intensivere Dokumentation
%%% Siehe Package auf CTAN
\usepackage[many]{tcolorbox}
\tcbuselibrary{skins,breakable,listings}
%https://tex.stackexchange.com/questions/530859/add-option-to-point-to-code-file-in-tcblisting-environment 
\newcounter{texlst}
\newtcbinputlisting[use counter=texlst]{\commoncode}[3][]{%
        enhanced,
		noparskip,
		breakable,
		colback=gray,
		opacitybacktitle=.8,%
        fonttitle=\bfseries,
        title after break={\centering\footnotesize\itshape\strut\lstlistingname~\thelstlisting~--~continued},%
        listing only,
		listing options={style=texstyle, xleftmargin=-1mm},
		%%%
		%%% create: package that allows for easy usage of the modulo operation.
		%%%
		%%% why: in this case i want only every other example to receive a higher number, as the left (original) corresponds with a right (translation)
		%%%
		%%% needs to: subtract 1 every _other_ listing in which \thetcbcounter gets used
		%%%
		%%% how (without changing \thetcbcounter):
		% given: an increasing counter from 1 to n
		% want: a counter increasing only every other step (i.e. 1,1,2,2,3,3,...,...,n,n)
		% done by taking (given) as input and (want) as output: (want) = (given / 2) (rounded down to below int) + 1 // mathematical prove in work // i seem to be stupid
		%%%
		after upper={\centering\strut\TeX{} Code~\thetcbcounter:~#2},
        frame hidden,
		boxrule=10pt,
        listing file = {#3}, #1
}


\usepackage{minted}
\usepackage{graphicx}
\usepackage{pdfpages} 
\usepackage{caption}
\usepackage{subcaption}




\usepackage[english,ngerman]{babel}
\usepackage[english=british]{csquotes}

\usepackage{tikz}
\usetikzlibrary{calc,positioning}
\usepackage{pgfplots}
\usepackage[nottoc]{tocbibind}
\usepackage{amsmath}

% Danke an https://tex.stackexchange.com/questions/14342/verbatim-environment-that-can-break-long-lines
\usepackage{fancyvrb}
\usepackage{fvextra}
% !!! Funktioniert nur mit ASCII-Zeichen (nicht mit deutschen Umlauten ä,ö,ü(,ß))

\usepackage{natbib}
\bibliographystyle{agsm}

%%% Uni-HRO relevantes
%%% TeX-relevant (siehe: titlepage.tex)
\author{Hendrik Theede}
\date{02.12.2025}
\title{Automatische Sprachübersetzung von \LaTeX{}-Dokumenten}
\def\matrikelnummer{221201256}
\newcommand{\supervisor}{Prof.\ Dr.\ rer.\ nat.\ habil. Clemens H. Cap}					% Betreuer
\newcommand{\ief}{Fakultät für Elektrotechnik und Informatik}							% Fakultaet
\definecolor{colorscheme}{cmyk}{0.90, 0.30, 0.00, 0.00} 								% Farbschema der Fakultaet; Siehe Corp. Design
\newcommand{\iuk}{Lehrstuhl für Informations- und Kommunikationsdienste}				% Institut



%%% "Deutsche Sprache"-relevant
\renewcommand*\contentsname{\hypertarget{toc}{Inhaltsverzeichnis}} % cannot contain \par (and thus: newlines)
\renewcommand*\refname{Literaturverzeichnis} % can contain \par (and thus: newlines)
\renewcommand*\figurename{Abbildung}



% Muss ans Ende, weil sonst übernimmt ein Biber aus der Bibliothek
\usepackage{hyperref}					% TeX!	
\hypersetup{
		linktoc=all,
		allcolors=black,
		colorlinks=true, % Für offizielle Releases das % vorne wegnehmen. Ersetzt die Boxen um Links durch die eigentlichen Farben
		linkcolor=black,
		urlbordercolor={1 0 0},
		urlcolor=blue, 
		citecolor=magenta,
		breaklinks=true,
		pdftitle=Abschlussarbeit,
		pdfauthor=Hendrik Theede,
		pdfsubject=Matrikelnummer 221201256,
		pageanchor,
		backref,
}

\begin{document}

\makeatletter % um \@author und co zu nutzen
\thispagestyle{empty}

\begin{figure}[h!]
\begin{tikzpicture}[remember picture,overlay,shift=(current page.south west)] 
	\begin{scope}
		% Koordinaten für Titel...
		\coordinate (A) at (3.7cm,20cm);
		% ... und Angaben
		\coordinate (B) at (3.7cm,3cm);
		
		% Uni-Logo
		\node [right] at (2cm,25cm) {\includegraphics[width=12cm]{pictures/UNIHRO_LOGO_2025.pdf}};
		
		% Rahmen
		\draw[line width=2pt,colorscheme,rounded corners=2ex] (0cm,0cm) +(2cm,0cm) --(2cm,23.5cm) -- (19cm,23.5cm) -- (19cm,0cm);
		
		% Titel
		\node at (A) [below right] {\parbox{.85\textwidth}{\noindent\bfseries\sffamily{\Huge\raggedright \textcolor{colorscheme}{\@title}\par}}};
		
		% Angaben
		\node at (B) [above right] {\parbox{.85\textwidth}{
				\begin{tabular}{ll}
					Name: & \Large\@author				\\[2pt]
					Matrikelnummer: & \matrikelnummer					\\[1ex]
					Abgabedatum: & \@date				\\[5ex]
					Betreuer und Gutachter: & \supervisor  		\\[2pt]
					& Universität Rostock						\\[2pt]
					& \ief										\\
				\end{tabular}
			}
		};
		
		\fill[colorscheme] (0cm,0cm) +(2cm,0cm) rectangle (19cm,2cm);
		\path (3.7cm,1.5cm) [right,white] node{\Large\sffamily\textbf{\Large{Bachelorarbeit}} \small{am \iuk}};
		\path (3.7cm,.8cm) [right,white] node{\Large\sffamily\textbf{\ief}};
	\end{scope}
\end{tikzpicture}
\end{figure}
\makeatother



\pagenumbering{Roman}
\setcounter{page}{0}
\newpage
\newpage
\section*{Abstrakt}
placeholder
\newpage
\tableofcontents
\newpage
\pagenumbering{arabic}

\begingroup
\hypersetup{hidelinks,pdfborder={0 0 1},allbordercolors=magenta}% Hat länger gedauert, als mir lieb ist. sagt TeX: innerhalb dieser Gruppe sollen links nicht farbig sein. würde zuerst auch implizieren, dass keine Rahmen in der PDF angezeigt werden sollen. Wie löst man das? Indem man manuell sagt: Wir haben einen Rahmen um Hyperrefs mit Offset X = 0, Offset Y = 0 und Stärke in Pixeln: 1 (standard)

\section{Einleitung}
\subsection{Hintergrund}\phantomsection\label{einleitung:hintergrund}
% Was haben wir probiert und auf welches Problem sind wir gestoßen?
Beim Versuch herkömmliche Software zur Übersetzung von menschlicher Sprache auf \TeX{}-Quellcode anzuwenden, werden schnellig Dokumente erzeugt, welche entweder nicht vollständig übersetzt wurden oder sich nicht kompilieren lassen. Google Translate zeigt hierbei schnell die Gründe hierfür, bzw.\ wie sich diese äußern. Beispielsweise führt eine Übersetzung von \verb|hello wor\textit{ld}| nicht zu \verb|Hallo We\textit{lt}|, sondern zu \verb|hallo wor\textit{ld}|. Abgesehen von der Frage, wo die kursive Hervorhebung im eigentlichen String erfolgen soll, werden Leser eines kompilierten Dokumentes das Wort \enquote{Welt} erkennen können. Zuvor beschriebene Zeichenkette wird zu \enquote{hello wor\textit{ld}} aufgelöst, in welcher das Wort \enquote{world} für einen menschlichen Leser als das englische Wort für \enquote{Welt} erkenntlich bleibt.% warum doppelt
\\\noindent
% reviewed: 1



% Okay hier und da wird mal ein Wort ausgelassen, warum ist das in größeren Dokumenten problematisch? Man kann doch das Wort fix googlen? Oh weh wir weichen ab von TeX
Diese Tatsache scheint im ersten Augenblick nicht weiter verheerend, kann jedoch bei größeren Dokumenten einen Logikbruch hervorrufen. 
Betrachtet man \enquote{einen Übersetzer} zunächst als Konzept, mit welchem, ohne Vorwissen von den (sprachlichen) Inhalten eines Dokumentes, diese Inhalte in eine andere Sprache als die Originale übersetzt werden soll, stellt man ein Risiko des Kontextverlustes fest. Als einfaches sprachliches Beispiel dienen hier zum Beispiel Wörter mit zeitlichem/räumlichen Bezug. Einzelne Sätze schnell ihre sprachliche Bedeutung, wenn einzelne Wörter verloren gehen, welche für den Kontext oder die Semantik des Satzes unabdingbar sind. Der Satz \textit{Morgen wird es regnen.} könnte ohne das Wort \enquote{morgen} als Frage mit $($schwacher$)$ deutscher Rechtschreibung interpretiert werden können. $($\textit{Wird es regnen?}$)$. Nun wird keine Aussage mehr getroffen, sondern folgend eine Aussage erhofft.
% reviewed: 0

% Wieder zurück zu TeX! Ja gut, hier und da mal etwas TeX Syntax zu übersehen kann ja nicht schaden, solange das Dok. kompiliert... oder?
Genauso wie das Fehlen einzelner Wörter die sprachliche Bedeutung brechen kann, könnte auch das Übersetzen von \LaTeX{}-Makros zu einem semantischen Verlust (von einzelnen Worten) für die \TeX{}-Engine beitragen. Konträr zu dem zuvor geschilderten \enquote{Verpassen} von Texten, welche übersetzt werden sollen, steht man hier vor dem Problem, dass zu viele textliche Strings übersetzt werden. Bereits die Möglichkeit eine Dateiendung (in bestimmten Fällen) auszulassen, führt dazu, dass z.B.\ \verb|\include{clock}| fälschlich zu \verb|\include{Uhr}| übersetzt wird (Google Translaten am 06.10.2025), wohingegen \verb|\include{clock.tex}| erhalten bleiben würde. Ersteres hätte zur Folge, dass die eingebundene Datei nicht mehr referenziert und demnach während des Kompilier-Prozesses nicht mehr erfasst werden würde.\\\noindent 
% reviewed: 1

% Oh weh, was wenn der Kontext des Dokumentes verloren geht?
\LaTeX{} erlaubt viele solcher Syntax-Brüche und es könnten theoretisch gesehen auch (leicht) fehlerhafte \TeX{}-Quellcodes kompiliert werden. Während~\hyperref[einleitung:hintergrund]{einleitende Beispiele} eher wenig intuitiv folgende Beispiele präsentieren, zeigen jedoch die Folgen dieser weitreichende Probleme auf (innerhalb eines Dokumente). Als Beispiel für ein deutsches Wort, welches von einem übersetzenden Programm fälschlich aufgegriffen werden könnte, wäre \enquote{ungerade}. Insofern dieses Wort umgangssprachlich interpretiert werden würde, müsste es im z.B.\ Englischen in dem Wort \enquote{crooked} (im Sinne:\ \textit{schief}) enden, aber im Bezug auf eine Zahl als \enquote{odd}.\\\noindent
% reviewed: 0


%%%%%%%%%%%%%%%%%%%%%%%
%%%%%%%%%%%%%%%%%%%%%%%
% Ab hier: intensiver Review

Neben solchen rein technischen Details, darf eine menschliche Perspektive nicht missachtet bleiben und so dürfen keine Übersetzungsprozesse dazu führen, dass in einem Dokument versteckte (im Sinne:\ nicht lesbare) Inhalte entstehen. Solche sind zunächst aus verschiedenen Layouting-Problemen herleitbar und abhängig von einzelnen Sprachen unterschiedlich. 


% Daneben sollten allerdings auch keine unlesbaren Sektionen innerhalb der jeweiligen Dokumente entstehen, die aus von Layouting-Problemen resultieren, welche sich für die Übersetzung in einige Sprachen zeigen (jedoch in einigen Fällen unvermeidbar sind).\\ 
% \noindent
% Wünschenswert ist neben vorigen Aspekten auch Möglichkeiten für den Endnutzer zu erlauben, sollte dieser spezielle Übersetzungen oder Kontexte für einige Wörter wünschen, welche jedoch nicht aus dem Dokument selbst hervorgehen. % Schaffe ich es in dieser Arbeit das Wort "inhärent" NICHT zu verwenden?
% Außerdem sollte ein möglichst hoher Support für sowohl verschiedene menschliche Sprachen, aber auch verschiedene \LaTeX{}-Pakete gegeben sein, wobei Letzteres nur ein Bonus ist, sollten Systeme wie Ti\textit{k}Z, bzw.\ \texttt{pgfplots} oder Bib\TeX{} innerhalb \LaTeX{} (zusammen mit \TeX{}) nutzbar bleiben.% Da sich durch diese sämtliche Verhalten anderer Pakete reproduzieren ließen.
% Welche Fehler können entstehen?
\section{Problemfälle}
Mittels \TeX{} ist prinzipiell alles möglich (wie sich später zeigen wird), jedoch sollte man zunächst denkbare Schwierigkeiten für jegliche Sprachübersetzungen von \TeX{} und \LaTeX{} Dokumenten nicht nur dahingegen schildern, dass sie auftreten könnten, sondern auch dahingegen klassifizieren, inwiefern sie häufig zu erwarten sind, geschweige denn sinnvoll oder gar unsinnig sein könnten (da sie z.B.\ eine zukünftige Bearbeitung eines Dokumentes erschweren könnten). Beruft man sich zunächst nur auf die reine \LaTeX{}-Syntax, werden vorerst wahrscheinlich nur simplere Probleme erkennbar, jedoch führt eine Näherung an die \TeX{}-Engine (und insb.\ deren Primitiven) eine Vielzahl von komplexeren und speziellen Problemen mit sich. 

% Erwartung: Wer das nicht rafft ist doof.
\subsection{Simple Probleme}
% Befehle innerhalb der TeX-Engine (nativ) dürfen nicht übersetzt werden.
\paragraph*{Zeichenketten\label{par:zeichenketten}} sind eine übliche Art und Weise, mit welcher man Wörter einer Sprache darstellen kann. Jedoch gehen die meisten Übersetzungs-Tools nicht davon aus, dass solche Zeichenketten Zeichen beinhalten, welche das folgende Wort zu einem Befehl (für eine Programmiersprache) machen. Zwar würde z.B.\ Google Translate für die Zeichenkette \texttt{Hello} korrekterweise das Deutsche \texttt{Hallo} liefern, aber bereits die Präambel von \TeX{}-Dokumenten zeigt, wie \verb|\title|, \verb|\author| und \verb|\date| respektiv zu \verb|\Titel|, \verb|\Autorin| und \verb|\Datum| übersetzt werden würden (Stand: 01.10.2025). Benanntes Tool zeigt sich zudem inkonsistent. Beispielsweise wird \verb|\section{saw}| zu \verb|\Abschnitt{Säge}| übersetzt und \verb|\section{Introduction}| zu \verb|\section{Einführung}| übersetzt.

% Von Wörtern zu mehreren Wörtern
\paragraph*{Whitespace\label{par:zeichenketten}} sind eine herkömmliche Art verschiedene Wörter einer Sprache voneinander zu trennen (bspw.\ in den lateinischen oder kyrillischen Sprachen). Neben anfänglichen Schwierigkeiten, welche sich innerhalb von einzelnen Zeichenketten aufzeigen könnten, ist es genauso denkbar, dass einzelne Optionen in \TeX{} oder \LaTeX{} innerhalb von eckigen oder geschwungenen Klammern nicht übersetzt werden dürften, ohne die Syntax zu brechen oder übersetzt werden müssten, damit ein gesamtes Dokument übersetzt wird. Denkbar sind hier direkt für das Erstere Farbdefinitionen, wie zum Beispiel \verb|\definecolor{super light red}{rgb}{1,.5,.5}|. Sollte man versuchen~\ref{par:zeichenketten}/Zeichenketten dadurch zu lösen, dass man einfach die Präambel in der Übersetzung ausschließt (also alles vor \verb|\begin{document}|), so würde ein späteres Nutzen dieser Farbe \texttt{super light red} dafür sorgen, dass das Wort \textit{light} alleine steht, und somit für nicht weit durchdachte Ansätze als ein zu übersetzendes Wort gelten würde. 

\paragraph*{Einbinden von anderen Dateien\label{par:anderedateien}} ist eine denkbare Art größere \TeX{}-Dokumente in übersichtlichere kleinere Dateien zu strukturieren. Neben der Möglichkeit \TeX{}-Dokumente selbst via \texttt{include} und \texttt{input} in ein übergreifendes Dokument einzufügen, ist es jedoch auch möglich verschiedene andere, bildliche Formate (bspw.\ PNG, PDF, \ldots) im Dokument zu integrieren. Insbesondere bei PDF kann es sehr interessant werden, ob und inwieweit textliche Inhalte erfasst und übersetzt werden, jedoch sind PDF ihrerseits wieder eine von \LaTeX{} und \TeX{} abweichende Datei-/Dokumentenform und daher nicht weiter kritisch. Schade wäre es, wenn solche eingebundenen \TeX{}-Dateien und deren textlichen Inhalte übersehen werden würden, andererseits unerwartet jedoch hervorragend, sollten sogar textliche Inhalte von PDF erkannt (und übersetzt) werden. Einer Erkennung von textlichen Inhalten auf einem Bild wird nicht weiter nachgesehen, da hierbei davon ausszugehen sein sollte, dass die Übersetzung von textuellen Inhalten in Bildern eine andere, gesonderte Disziplin in der Bildverarbeitung ist. (#Evtl? Beispiele zeigen, dass hieran geforscht wird?)% # damit der Compiler meckert... (kompilieren kann es trotzdem)

% Denkt man nicht direkt dran, ist jedoch auch wichtig
\subsection{Komplexere Probleme}
Neben den zuvor geschilderten sehr einfachen Problemen, welche sich auch unabhängig von \TeX{} (und \LaTeX{}) zeigen könnten (denn z.B.\ ein Übersetzen von Hashtags im Social Media sollte keine neue Idee sein) und gelöst sein müssen (da ansonsten sehr einfache und rudimentäre Werkzeuge für eine Dokumentenerstellung verloren gehen, da man sich ohne diese simplem Formatierungsoptionen wieder auf einfache Textdateien berufen könnte).% Bitte die Klammer nochmal reviewen

\paragraph*{Makros} sind eine Möglichkeit mehrere \TeX{}-Befehle zusammenzufassen. Vor allem in \LaTeX{} sind eine Vielzahl dieser bereits vordefiniert, jedoch handelt es sich bei diesen meist um Wörter der englischen Sprache (\enquote{meist}: manche dieser englischen Wörter treten auch in anderen Sprachen auf, bspw.\ \textit{paragraph}$\leftrightarrow$\enquote{Paragraph}). Sollte es einem \TeX{}-User leichter fallen in der z.B.\ französischen Sprache zu arbeiten, so könnte dieser beispielsweise neue, französische Makros mit \\\verb|\newcommand{\anglais}{This is some \textit{formatted} \texttt{english} \TeX{}-t}|\\erzeugen. Das vorige Beispiel zeigt zudem auf, wie Texte innerhalb von \TeX{}-Makros \enquote{verschwinden} können und wirft die Frage auf, wann und wie solche Texte übersetzt werden sollten. Am sinnvollsten erscheint zunächst nur Zeichenketten zu übersetzen, welche sich mit der prominentesten Sprache des gesamten Dokumentes decken, welche allerdings nicht ohne weiteres bekannt ist. Selbst wenn in dem gesamten Dokument größtenteils englische Wörter vorliegen, ist eigentlich nur interessant, in welcher Sprache die reinen Strings (welche auf der PDF lesbar erscheinen) geschrieben sind. Selbst diese Information alleine ist theoretisch gesehen noch keine Grundlage für eine Aussage darüber, welche Sprache in solche einem Fall übersetzt werden müsste, da man hier Kenntnis des eigentlichen, entgültigen Dokumentes bräuchte, denn es könnte auch von Interesse sein, innerhalb eines größtenteils z.B.\ deutschsprachigen Dokumentes nur vereinzelte, englische Sätze zu übersetzen. Hierauf wird in Abschnitt~\ref{subsec:weitereschwierigkeiten} näher eingegangen, da sich dieses Problem zunächst recht einfach durch eine Auswahlmöglichkeit der Ausgangssprache (= die zu Übersetzende) lösen ließe.\\% und da man davon aussgeht, dass einzelne Dokumente üblicherweise, überwiegend in ein- und derselben Sprache verfasst werden und eher seltener Sprachwechsel vorkommen.
\noindent
Gleiches ist zu berücksichtigen, sollte das Kommando \verb|\renewcommand| verwendet werden, wobei dieses allerdings noch ein wenig mehr zulässt. Hiermit ist man auch dazu in der Lage existierende Befehle der \LaTeX{}-Syntax zu ändern, wodurch ein \verb|\Abschnitt{Einleitung}| ebenfalls valide \LaTeX{}-Syntax werden könnte, welche ein \TeX{}-Compiler als \verb|\section{Einleitung}| richtig interpretieren könnte, aber ein übersetzendes Programm könnte dieses womoglich in \verb|\section{example}| überführen. Dies scheint zunächst kein Problem zu sein, jedoch hätte zwischen einem \verb|\renewcommand{\section}{\Abschnitt}| genauso ein \verb|\newcommand{\section}{\frac{1+\sqrt{5}}{2}}| stattfinden können, wodurch \verb|\section{example}| nicht in einem Abschnitt mit Titel \enquote{example}, sondern in $\frac{1+\sqrt{5}}{2}${example} resultieren würde.

\paragraph*{Umgebungen} sind, wie der Name es vermuten lässt, einzelne Bereiche im Dokument, welche gesondert behandelt werden und für welche sich jegliche Einstellungen, wie z.B.\ Textfarbe, Textgröße, Schriftart, Font und vieles Weitere nur für eine solche Umgebung anpassen lassen. Einerseits kann man über geschwungene Klammern \verb|{}| eine Umgebung einmalig betreten oder verlassen, möchte jedoch auch die Möglichkeit erhalten diese erneut zu verwenden und ihr verschiedene Parameter zu übergeben. Eine Definition einer Umgebung in der Präambel lässt dies zu, % Es ist so schwer keine selbstzweifelnden Kommentare hier zu schreiben, wenn man sich selbst hasst.
wodurch sich neben den in~\ref{par:whitespace} aufgezeigten Problemen nicht nur für etwaige Farboptionen und -einstellungen Strings aufzeigen, welche nicht übersetzt werden dürfen, sondern auch eigens (vom \TeX{}-User) Ausgedachte (#Hier: Verweis auf Anhang passend).

% Eigentlich PGF, aber Frage mich gerade auch: könnten u.U. Links übersetzt werden? Und auch die Optionen von hyperref sollten mal ggb. einer Übersetzung geprüft werden!!!
\paragraph*{Pakete} bieten eine \TeX{}-Schnittstelle für die gesamte Welt! Zumindest rein theoretisch natürlich. Technisch gesehen bieten sie die Möglichkeit zuvor beschriebene Umgebungen und Makros in einer eigenen \texttt{.bst} zu bündeln, welche ihrerseits (vorrangig via) CTAN (jedoch auch auf jeglichem anderem Wege) zu anderen \TeX{}-Usern übertragen werden könnte. Verschiedene Pakete könnten hierbei eine Vielzahl individueller Probleme aufwerfen, zunächst ist jedoch mehr ein Fokus auf solche zu setzen, welche die Arbeit anderer Programme involvieren. Sie in Dokument mit einzubinden ist recht leicht und funktioniert nur auf eine begrenzte Anzahl an Methoden (\texttt{requirepackage} und \texttt{usepackage}) und sind ihrerseits, genauso wie~\ref{par:anderedateien} 

\subparagraph*{Ti\textit{k}Z} ist zum Einen eine Möglichkeit in \TeX{} zu malen, jedoch hauptsächlich dahingegen konszipiert in einem wissenschaftlichen Kontext verwendbare Diagramme mathematisch zu beschreiben oder auf Grundlage von Messwerten zu erzeugen. Die Syntax von Ti\textit{k}Z und \texttt{pgfplots} kann innerhalb eines Dokumentes auch freistehende englische Wörter beinhalten, wie zum Beispiel in\ldots
\begin{verbatim}
    \begin{tikzpicture}[h!]
        \centering
        \begin{axis}[
            domain=-8:8
        ]
        \addplot{x};
        \end{axis}
    \end{tikzpicture}
\end{verbatim}
\ldots bei welchem ein Übersetzen von \enquote{domain} Fehler produzieren würde, da Ti\textit{k}, bzw.\ \texttt{pgf} von einem englischen Wort ausgeht. 
\subparagraph*{Bib\TeX{}} wird genutzt um Zitationen/Referenzen/Literaturverweise innerhalb eines Einzelnen oder mehreren Dokumenten zu nutzen und zu verwalten. Die Bib\TeX{}-Notation selbst beläuft sich auf eine einfache JavaScript Object Notation und trägt mit einer Ausnahme nur nicht zu übersetzende Inhalte, wie den Autor, den Titel des Werkes (welcher in der Originalsprache oder der durch den Autoren genehmigten übersetzten Titel), das Datum, einer URL, einer DOI, einer Angabe darüber, ob das zitierte Werk aus einem Buch, einer laufenden Reihe an wissenschaftlichen Publikationen (bspw.\ \textit{nature}, \textit{science}, \textit{ACM Computating Surveys},\ldots) oder einer Konferenz (oder Ähnlichem) stammt. Neben diesen Angaben, welche allesamt nicht übersetzt werden brauchen, bleibt das Abstrakt eines zitierten Werkes interessant für einen Übersetzungsvorgang, sollte man davon ausgehen, dass man im Anschluss entstehende, übersetzte \texttt{.tex} Dateien an einen neuen Autoren übergeben möchte. %Welcher obviously die Sprache spricht, in welche übersetzt wurde.

%%% Könnte ne gute Überleitung zu den "Speziellen Problemen" ergeben
\subparagraph*{Mathematische Formeln} selbst sind kein eigenes Paket, jedoch einer der praktischsten Use-Cases von \TeX{}. Insbesondere für Menschen, welche sich eine handschriftliche Qualität und \enquote{Streichlust} (meint:\ das Durchstreichen auf dem Papier, sollte man sich verschrieben haben) mit der des Autoren (dieser Arbeit) teilen, sollte das digitale Medium \TeX{} einiges an Aufwand ersparen und jegliche Herleitungen deutlicher und übersichtlicher machen. % Theorem Package, jedoch auch insgesamt
Hierzu gibt es wiederum mehrere denkbare Pakete, welche diesen bereits in \TeX{} inhärent verankerten \enquote{\textit{math mode}} erweitern oder vereinfachen können. 

% Wo man eher weniger dran denkt
\subsection{Spezielle Probleme}
\paragraph*{Höhere Vernestungsgrade}% hier passt \relax mit rein, so als Beispiel, hab da ja ein zwei Tabellen rumliegen
\subparagraph{In Tabellen}
\subparagraph{In mathematischen Umgebungen}


% Woran man direkt denkt, wenn man LaTeX für eine wissenschaftliche Arbeit nutzen möchte

\paragraph*{Quelltexte}% Ja auch hieran
\paragraph*{Kommentare}% mit das Fieseste, was mir eingefallen ist

\paragraph*{Definitionen} 
\paragraph*{Catcode und Unicode}


\subsection{Weitere Schwierigkeiten}\label{subsec:weitereschwierigkeiten}
Beabsichtigt ist dieser Abschnitt nicht in der Reihe von Problemen aufgefasst, sondern als Schwierigkeit$($en$)$ formuliert, da man sich hier von den Problemen abwenden würde, welche in der \TeX{}-Syntax auftreten und bei sprachliche Hürden angelangt, welche sich für und zwischen verschiedenen Sprachen zeigen könnten.
% Siehe uni/thesisbsc/tests/prblems/list.md/##9
\paragraph*{Mehrdeutigkeiten} innerhalb einer Sprache führen unter Umständen zu missverständlichen Übersetzungen.  % Hier würde das von Warren, siehe andere Einleitung (introduction/intro.tex) evtl. eignen für weitere Beispiele, nicht nur welche von mir? k.P.
\paragraph*{Redewendungen} sind eine Art und Weise\ldots

\paragraph*{Wirrer Sprachwechsel} meint ein rapides Springen zwischen verschiedenen Menschensprachen innerhalb eines Dokumentes. Die Fragestellung hierbei ist, inwiefern ein sprachlicher Wechsel innerhalb eines Dokumentes erfasst wird, sollte eine automatische Spracherkennung der Ausgangssprache stattfinden. Dabei können verschiedenste (theoretisch: überabzählbar viele) Fälle auftreten, unter welchen z.B.\ Wechsel aus dem Deutschen in das Englische an beliebieger Stelle im Dokument, satzweisige Wechsel zwischen zwei und mehreren Sprachen, sowie ein nur kurzfristiger Wechsel in eine Sprache, innerhalb eines ansonstig monolingualen Dokumentes, welche allerdings Lexeme dieser beinhält (bspw.:\ ein norwegisches Dokument beinhält ein dänisches Zitat).
% Zusammenfassung der Inhalte und Ziele der jeweiligen Kapitel
\section{Problemfälle}% 
Schildert alle denkbaren Probleme, auf die ein Übersetzungsprogramm stoßen könnte, wenn es nicht weiß, dass ein \LaTeX{} Dokument vorliegt.
\subsection{Technische Semantik}% Die allgemeine Ausgangslage ist ein unbekanntes Dokument, von einem beliebigen Autoren und Schilderung erfolgt unabhängig von den eigentlichen Inhalten des Dokumentes (im Sinne: "leeres" Dokument, unbedeutend, aber LaTeX).
Listet die Probleme, welche Unverständnis für \LaTeX{} produzieren (hier:\ übergreifend, meint:\ die Organisation dahinter).
Hier noch kein Fokus auf menschensprachliche Inhalte, denn die sprachliche Semantik würde vollständig verloren gehen, sollte ein Dokument nicht produzierbar werden.

\subsubsection{Ausgangssituation (Was macht diese Art von Fehler anders als die danach?)}
\subsubsection{Exemplarische Beispiele}
\begin{itemize}
    \item Sonderzeichen 0d
    \item Leerzeichen 1d
    \item Zeilenbrüche 2d
    \item Dokumentenbrüche 3d
\end{itemize}

\subsection{Sprachliche Semantik}% Die allgemeine Ausgangslage ist ein Dokument, welches Inhalte trägt, aus welchen Rückschlüsse auf die Inhalte möglich sind (bspw. Autoren, Titel, Referenzen, ...). Dokument lässt Rückschlüsse auf die eigentlichen Inhalte zu, aber diese gehen aus dem Quellcode nicht hervor.
Listet die Probleme, welche kontextuell falsche Übersetzungen provozieren. An einigen Stellen eignet sich evtl.\ Google Translate aus technischen Gründen nicht mehr zur Demonstration.
Fokus auf Vorgaben, wie Quellverweise, welche den Kontext des Satzes abhängig von diesen Referenzen ändern und dadurch andere Wörter produzieren sollten.
Kontext-Quellen:
\subsubsection{Ausgangssituation (Was macht diese Art von Fehler anders als die danach?)}
\subsubsection{Exemplarische Beispiele}
\begin{itemize}
    \item Titel/Autoren,
    \item Dokumenten-interne Verweise/Referenzen
    \item Externe Referenzen
    \item Formeln/Tabellen
    \item Graphiken
    \item \ldots
\end{itemize}

\subsection{Spezifischer Technologien}
Listet die Probleme, welche spezielle Techniken berücksichtigen müssten.\\
Meint:\ Mehrfach-Kompilation erforderlich? Wurde bspw.\ ein Kapiteltitel (o.Ä.) übersetzt (table of contents o.Ä)?\\
Meint:\ Fokus auf entstehende Layouting-Probleme durch den unterschiedlichen Platzbedarf geschriebener Sprachen.\\
Meint auch:\ Übersetzen von Quellcodes anderer Programmiersprachen.\\
Meint auch:\ Dilemmas, die durch Makros entstehen könnten.\\
Meint hier endlich:\ Catcode.\\

\subsection{Weitere Schwierigkeiten}
Alltagsbeispiele. Abkürzungen. 
%\section{Ungelöst}
\subsection{Verfolgter Lösungsweg}%
Ausgehend von den zuvorigen Takeaways
\subsection{Gelöste Probleme}
\subsection{Lessons Learned}
\subsection{Fazit}
%\section{Fazit}
\subsection{Zusammenfassung}
\subsection{Ausblick}
\subsection{Weiterführend}

\endgroup

\newpage

\makeatletter
\section{Eigenständigkeitserklärung}
Ich versichere hiermit, dass ich die vorliegende Arbeit selbstständig angefertigt und ohne fremde Hilfe verfasst habe. %, es sei denn die Aufgabenstellung verlangte unausweichlich einer (maschinellen) Unterstützung
Dazu habe ich keine außer den von mir angegebenen Hilfsmitteln und Quellen verwendet und die den benutzten Werken inhaltlich und
wörtlich entnommenen Stellen habe ich als solche kenntlich gemacht. 
Ich versichere, dass die eingereichte elektronische Fassung mit den gedruckten Exemplaren übereinstimmt.
\vspace{2cm}
\begin{figure}[b]
	\raggedright{}
	Rostock, den \@date\\[8ex]% \date setzt das Datum (eine Variable), während \@ zuerst impliziert: Wir befassen uns mit einem Befehl, der eine Variable sein könnte und das @ bestätigt: es ist eine Variable und wir wollen auf den Wert zugreifen.
	$\overline{\text{\@author}}$% overline = linie über dem umklammerten, \text: hier steht Text in einer mathematischen Formel, 
	\vspace{2cm}
\end{figure}
\makeatother

\newpage


\bibliography{index}

\begingroup
\hypersetup{hidelinks,pdfborder={0 0 1},allbordercolors=magenta}
\newpage
\renewcommand\thesection{\Alph{section}}
\setcounter{section}{0}
\renewcommand*{\theHsection}{chX.\the\value{section}} % Danke an: https://tex.stackexchange.com/questions/71162/reset-section-numbering-between-unnumbered-chapters
% \thesection = der TeX interne Abschnittscounter
% \theHsection = der Hyperref interne Sektionencounter. Arbeite zwar nur in einem chapter, jedoch ist chX.0 ein neuer String ggb. (wahrscheinlich) ch0.0
\section{Anhänge}
\endgroup

\end{document}