\documentclass[11pt,toc]{article}
\renewcommand{\familydefault}{\sfdefault}
\setcounter{secnumdepth}{6}
\usepackage[left=2cm,right=2cm]{geometry}
%\usepackage[document]{ragged2e} % Damit alles: Links aligned (der linkeste Punkt eines jeden Buchstaben ist auf einer Linie für alle zeilen) und so viel Platz nach rechts ausnutzt, wie möglich (nein, macht TeX nicht standardmäßig); muss man nicht für das gesamte Dokument machen (wie hier), geht auch per \raggedright oder \RaggedRight für/vor Paragraphen; gibt auch Umgebungen für rechtsbündigen Text https://de.overleaf.com/learn/latex/Text_alignment
\usepackage[dvipsnames]{xcolor}
\usepackage{booktabs,array}
\usepackage[author={Hendrik Theede}]{pdfcomment}
\usepackage{tabularx}
\usepackage{listings}
\definecolor{codegreen}{rgb}{0,0.6,0}
\definecolor{codegray}{rgb}{0.5,0.5,0.5}
\definecolor{codepurple}{rgb}{0.58,0,0.82}
\definecolor{backcolour}{rgb}{0.95,0.95,0.92}

\lstdefinestyle{mystyle}{
    backgroundcolor=\color{backcolour},   
    commentstyle=\color{codegreen},
    keywordstyle=\color{magenta},
    numberstyle=\tiny\color{codegray},
    stringstyle=\color{codepurple},
    basicstyle=\ttfamily\footnotesize,
    breakatwhitespace=false,         
    breaklines=true,                 
    captionpos=b,                    
    keepspaces=true,                 
    numbers=left,                    
    numbersep=5pt,                  
    showspaces=false,                
    showstringspaces=false,
    showtabs=false,                  
    tabsize=2
}

\lstset{style=mystyle}



\usepackage{minted}
\usepackage{graphicx}
\usepackage{pdfpages} 
\usepackage{caption}
\usepackage{subcaption}




\usepackage[english,ngerman]{babel}
\usepackage[english=british]{csquotes}

\usepackage{tikz}
\usetikzlibrary{calc,positioning}
\usepackage{pgfplots}
\usepackage[nottoc]{tocbibind}
\usepackage{amsmath}

% Danke an https://tex.stackexchange.com/questions/14342/verbatim-environment-that-can-break-long-lines
\usepackage{fancyvrb}
\usepackage{fvextra}
% !!! Funktioniert nur mit ASCII-Zeichen (nicht mit deutschen Umlauten ä,ö,ü(,ß))

\usepackage{natbib}
\bibliographystyle{agsm}

%%% Uni-HRO relevantes
%%% TeX-relevant (siehe: titlepage.tex)
\author{Hendrik Theede}
\date{02.12.2025}
\title{Automatische Sprachübersetzung von \LaTeX{}-Dokumenten}
\def\matrikelnummer{221201256}
\newcommand{\supervisor}{Prof.\ Dr.\ rer.\ nat.\ habil. Clemens H. Cap}					% Betreuer
\newcommand{\ief}{Fakultät für Elektrotechnik und Informatik}							% Fakultaet
\definecolor{colorscheme}{cmyk}{0.90, 0.30, 0.00, 0.00} 								% Farbschema der Fakultaet; Siehe Corp. Design
\newcommand{\iuk}{Lehrstuhl für Informations- und Kommunikationsdienste}				% Institut



%%% "Deutsche Sprache"-relevant
\renewcommand*\contentsname{\hypertarget{toc}{Inhaltsverzeichnis}} % cannot contain \par (and thus: newlines)
\renewcommand*\refname{Literaturverzeichnis} % can contain \par (and thus: newlines)
\renewcommand*\figurename{Abbildung}



% Muss ans Ende, weil sonst übernimmt ein Biber aus der Bibliothek
\usepackage{hyperref}					% TeX!	
\hypersetup{
		linktoc=all,
		allcolors=black,
		colorlinks=true, % Für offizielle Releases das % vorne wegnehmen. Ersetzt die Boxen um Links durch die eigentlichen Farben
		linkcolor=black,
		urlbordercolor={1 0 0},
		urlcolor=blue, 
		citecolor=magenta,
		breaklinks=true,
		pdftitle=Abschlussarbeit,
		pdfauthor=Hendrik Theede,
		pdfsubject=Matrikelnummer 221201256,
		pageanchor,
		backref,
}

\begin{document}

\makeatletter % um \@author und co zu nutzen
\thispagestyle{empty}

\begin{figure}[h!]
\begin{tikzpicture}[remember picture,overlay,shift=(current page.south west)] 
	\begin{scope}
		% Koordinaten für Titel...
		\coordinate (A) at (3.7cm,20cm);
		% ... und Angaben
		\coordinate (B) at (3.7cm,3cm);
		
		% Uni-Logo
		\node [right] at (2cm,25cm) {\includegraphics[width=12cm]{pictures/UNIHRO_LOGO_2025.pdf}};
		
		% Rahmen
		\draw[line width=2pt,colorscheme,rounded corners=2ex] (0cm,0cm) +(2cm,0cm) --(2cm,23.5cm) -- (19cm,23.5cm) -- (19cm,0cm);
		
		% Titel
		\node at (A) [below right] {\parbox{.85\textwidth}{\noindent\bfseries\sffamily{\Huge\raggedright \textcolor{colorscheme}{\@title}\par}}};
		
		% Angaben
		\node at (B) [above right] {\parbox{.85\textwidth}{
				\begin{tabular}{ll}
					Name: & \Large\@author				\\[2pt]
					Matrikelnummer: & \matrikelnummer					\\[1ex]
					Abgabedatum: & \@date				\\[5ex]
					Betreuer und Gutachter: & \supervisor  		\\[2pt]
					& Universität Rostock						\\[2pt]
					& \ief										\\
				\end{tabular}
			}
		};
		
		\fill[colorscheme] (0cm,0cm) +(2cm,0cm) rectangle (19cm,2cm);
		\path (3.7cm,1.5cm) [right,white] node{\Large\sffamily\textbf{\Large{Bachelorarbeit}} \small{am \iuk}};
		\path (3.7cm,.8cm) [right,white] node{\Large\sffamily\textbf{\ief}};
	\end{scope}
\end{tikzpicture}
\end{figure}
\makeatother



\pagenumbering{Roman}
\setcounter{page}{0}
\newpage
\newpage
\section*{Abstrakt}
placeholder
\newpage
\tableofcontents
\newpage
\pagenumbering{arabic}

\begingroup
\hypersetup{hidelinks,pdfborder={0 0 1},allbordercolors=magenta}% Hat länger gedauert, als mir lieb ist. sagt TeX: innerhalb dieser Gruppe sollen links nicht farbig sein. würde zuerst auch implizieren, dass keine Rahmen in der PDF angezeigt werden sollen. Wie löst man das? Indem man manuell sagt: Wir haben einen Rahmen um Hyperrefs mit Offset X = 0, Offset Y = 0 und Stärke in Pixeln: 1 (standard)

\section{Einleitung}
\subsection{Hintergrund}\phantomsection\label{einleitung:hintergrund}
% Was haben wir probiert und auf welches Problem sind wir gestoßen?
Beim Versuch herkömmliche Software zur Übersetzung von menschlicher Sprache auf \TeX{}-Quellcode anzuwenden, werden schnellig Dokumente erzeugt, welche entweder nicht vollständig übersetzt wurden oder sich nicht kompilieren lassen. Google Translate zeigt hierbei schnell die Gründe hierfür, bzw.\ wie sich diese äußern. Beispielsweise führt eine Übersetzung von \verb|hello wor\textit{ld}| nicht zu \verb|Hallo We\textit{lt}|, sondern zu \verb|hallo wor\textit{ld}|. Abgesehen von der Frage, wo die kursive Hervorhebung im eigentlichen String erfolgen soll, werden Leser eines kompilierten Dokumentes das Wort \enquote{Welt} erkennen können. Zuvor beschriebene Zeichenkette wird zu \enquote{hello wor\textit{ld}} aufgelöst, in welcher das Wort \enquote{world} für einen menschlichen Leser als das englische Wort für \enquote{Welt} erkenntlich bleibt.% warum doppelt
\\\noindent
% reviewed: 1



% Okay hier und da wird mal ein Wort ausgelassen, warum ist das in größeren Dokumenten problematisch? Man kann doch das Wort fix googlen? Oh weh wir weichen ab von TeX
Diese Tatsache scheint im ersten Augenblick nicht weiter verheerend, kann jedoch bei größeren Dokumenten einen Logikbruch hervorrufen. 
Betrachtet man \enquote{einen Übersetzer} zunächst als Konzept, mit welchem, ohne Vorwissen von den (sprachlichen) Inhalten eines Dokumentes, diese Inhalte in eine andere Sprache als die Originale übersetzt werden soll, stellt man ein Risiko des Kontextverlustes fest. Als einfaches sprachliches Beispiel dienen hier zum Beispiel Wörter mit zeitlichem/räumlichen Bezug. Einzelne Sätze schnell ihre sprachliche Bedeutung, wenn einzelne Wörter verloren gehen, welche für den Kontext oder die Semantik des Satzes unabdingbar sind. Der Satz \textit{Morgen wird es regnen.} könnte ohne das Wort \enquote{morgen} als Frage mit $($schwacher$)$ deutscher Rechtschreibung interpretiert werden können. $($\textit{Wird es regnen?}$)$. Nun wird keine Aussage mehr getroffen, sondern folgend eine Aussage erhofft.
% reviewed: 0

% Wieder zurück zu TeX! Ja gut, hier und da mal etwas TeX Syntax zu übersehen kann ja nicht schaden, solange das Dok. kompiliert... oder?
Genauso wie das Fehlen einzelner Wörter die sprachliche Bedeutung brechen kann, könnte auch das Übersetzen von \LaTeX{}-Makros zu einem semantischen Verlust (von einzelnen Worten) für die \TeX{}-Engine beitragen. Konträr zu dem zuvor geschilderten \enquote{Verpassen} von Texten, welche übersetzt werden sollen, steht man hier vor dem Problem, dass zu viele textliche Strings übersetzt werden. Bereits die Möglichkeit eine Dateiendung (in bestimmten Fällen) auszulassen, führt dazu, dass z.B.\ \verb|\include{clock}| fälschlich zu \verb|\include{Uhr}| übersetzt wird (Google Translaten am 06.10.2025), wohingegen \verb|\include{clock.tex}| erhalten bleiben würde. Ersteres hätte zur Folge, dass die eingebundene Datei nicht mehr referenziert und demnach während des Kompilier-Prozesses nicht mehr erfasst werden würde.\\\noindent 
% reviewed: 1

% Oh weh, was wenn der Kontext des Dokumentes verloren geht?
\LaTeX{} erlaubt viele solcher Syntax-Brüche und es könnten theoretisch gesehen auch (leicht) fehlerhafte \TeX{}-Quellcodes kompiliert werden. Während~\hyperref[einleitung:hintergrund]{einleitende Beispiele} eher wenig intuitiv folgende Beispiele präsentieren, zeigen jedoch die Folgen dieser weitreichende Probleme auf (innerhalb eines Dokumente). Als Beispiel für ein deutsches Wort, welches von einem übersetzenden Programm fälschlich aufgegriffen werden könnte, wäre \enquote{ungerade}. Insofern dieses Wort umgangssprachlich interpretiert werden würde, müsste es im z.B.\ Englischen in dem Wort \enquote{crooked} (im Sinne:\ \textit{schief}) enden, aber im Bezug auf eine Zahl als \enquote{odd}.\\\noindent
% reviewed: 0


%%%%%%%%%%%%%%%%%%%%%%%
%%%%%%%%%%%%%%%%%%%%%%%
% Ab hier: intensiver Review

Neben solchen rein technischen Details, darf eine menschliche Perspektive nicht missachtet bleiben und so dürfen keine Übersetzungsprozesse dazu führen, dass in einem Dokument versteckte (im Sinne:\ nicht lesbare) Inhalte entstehen. Solche sind zunächst aus verschiedenen Layouting-Problemen herleitbar und abhängig von einzelnen Sprachen unterschiedlich. 


% Daneben sollten allerdings auch keine unlesbaren Sektionen innerhalb der jeweiligen Dokumente entstehen, die aus von Layouting-Problemen resultieren, welche sich für die Übersetzung in einige Sprachen zeigen (jedoch in einigen Fällen unvermeidbar sind).\\ 
% \noindent
% Wünschenswert ist neben vorigen Aspekten auch Möglichkeiten für den Endnutzer zu erlauben, sollte dieser spezielle Übersetzungen oder Kontexte für einige Wörter wünschen, welche jedoch nicht aus dem Dokument selbst hervorgehen. % Schaffe ich es in dieser Arbeit das Wort "inhärent" NICHT zu verwenden?
% Außerdem sollte ein möglichst hoher Support für sowohl verschiedene menschliche Sprachen, aber auch verschiedene \LaTeX{}-Pakete gegeben sein, wobei Letzteres nur ein Bonus ist, sollten Systeme wie Ti\textit{k}Z, bzw.\ \texttt{pgfplots} oder Bib\TeX{} innerhalb \LaTeX{} (zusammen mit \TeX{}) nutzbar bleiben.% Da sich durch diese sämtliche Verhalten anderer Pakete reproduzieren ließen.
%%%%%%%%%%%%% Versuche die deutsche wissenschaftliche Sprache verständlich zu halten:
%%%%%%%%%%%%% - Infinitiv so oft wie möglich (durch Substantivierung oder Erweiterung), damit Texte leichter zu verstehen sind.
%%%%%%%%%%%%% - Reale Konditionalsätze nutzen.
%%%%%%%%%%%%% - Auch ein wenig den Leser abzuholen und nicht nur stumpf und maschinell die beiden oberen Regeln einzuhalten :) 


\section{Problemfälle}
Sprachliche Uneindeutigkeiten können in vielen Sprachen auftreten und dadurch Missverständnisse produzieren (bspw.\ sarkastische Kommentare oder Mehrdeutigkeiten einzelner Worte). Besonders kritisch sind solche Uneindeutigkeiten jedoch für übersetzende Programme, welche Dokumente in eine andere, menschliche Sprache überführen zu suchen, da hier bereits einzelne missinterpretierte Wörter die \TeX{} Syntax brechen könnten, wodurch nur noch ein unzureichender Teil der Beschreibung des Dokumentes bestehen bleibt, aus welcher kein echtes Dokument entstehen kann\pdfcomment{Abschnitt: technische Semantik. Tue mich schwer mit der Namensgebung der Kapitel.}. Das Einhalten der \TeX{} Syntax alleine genügt jedoch nicht, um erwähnte Mehrdeutigkeiten von einzelnen Wörtern zu verhindern, da sie abhängig ihres Kontexts eine andere Übersetzung verlangen\pdfcomment{Abschnitt: Sprachliche Semantik}.% Diese Art von PDF-Kommentar ist sichtbar. Geht in Firefox. Dann gehe ich davon aus, dass es in einer kostenpflichtigen software (acrobat) ebenfalls umgesetzt ist. 
Weiterhin können innerhalb von verschiedenen, in Kombination mit \TeX{} genutzten Systeme einzelne \enquote{Fehler} entstehen. Ansätze zur Behebung dieser sind jedoch in manchen Fällen bereits konzeptionell unmöglich. Hieran anknüpfend existieren einige sehr spezifische sprachliche\pdfcomment{meint:\ sowohl \TeX{} und co., also auch menschliche Sprache}.\\% par
\noindent
Aus kurzer Nachverfolgung und Einschätzung dieser Probleme werden anschließend konkrete Anforderungen gestellt. Eine Erläuterung dieser Art macht es allerdings unabdingbar eine Software zu nutzen, welche auf das übersetzen spezialisiert ist und nicht von sich aus \LaTeX{} oder \TeX{}-konforme Dokumente erwartet. Daher werden Beispiele anhand von Google's Web-Service \enquote{Translate} aufgeführt.% parsec; review: 1




% Bleibt die Bedeutung für die TeX-Engine bestehen?
\subsection{Technische Semantik}\phantomsection\label{problems:technological}
% Hiermit sagen wir: Punkte/Elemente der nullten Dimension sind veränderbar.
\subsubsection{Sonderzeichen}\phantomsection\label{problems:dim0}
Manche menschliche Sprachen beinhalten gelegentlich Zeichen, welche keine direkte Bedeutung tragen und kein Teil auch nur eines Wortes der Sprache sind. Denkbare Beispiele hierfür sind Klammern, die (üblicherweise) für ein impliziert Erwähntes, allerdings nicht vorzeiglich verwendetes Wort genutzt werden.% Meint: Eigentlich sollte man das Wort dort nicht oder nicht unbegründet verwenden, jedoch passt es gerade im Redefluss, um den Kontext schneller rüberzubringen (und mit schneller ist immer "besser" verbunden)
Stilistische Mittel für Texte sind prinzipiell nicht an spezifische Zeichen gebunden und ein Programm kann immer davon ausgehen, dass ein \enquote{Tippfehler} entstehen kann. Beispiel~\ref{tab:problems:dim0} zeigt allerdings, wie vereinzelte Zeichenketten produzieren können, welche fälschlich (in diesem Kontext:\ unerwünscht) übersetzt werden.

\begin{table}[h!]
    \centering
    \begin{tabularx}{\textwidth}{X X}
        \toprule
            Original & Übersetzung\\
        \midrule
        % Siehe ~/tests/readme.md für namensgebung und "Wo ist die Datei?"
            Korrekt & \\[-13px] % relative Angabe ggb. echtem Zeilenumbruch (1em) // relativ in dem Sinne: wir gehen von der Position des erwarteten Zeilenumbruches aus und verschieben nach unten oder oben (+ bzw - in der Höhe)
            \commoncode{Test}{../examples/technical/0d/correct_original.tex} & \commoncode{Test}{../examples/technical/0d/correct.tex}\\[1em]
            Unerwünscht & \\[-13px]
            \commoncode{Test}{../examples/technical/0d/wrong_original.tex} & \commoncode{Test}{../examples/technical/0d/wrong.tex}\\[-1em]
        \bottomrule
    \end{tabularx}
    \caption{Der für einen \TeX{} Compiler relevante Befehl \texttt{label} bleibt unverändert, allerdings \texttt{section} wird f\"alschlicherweise als \texttt{Abschnitt} \"ubersetzt}\label{tab:problems:dim0}
\end{table}

\paragraph*{Vermutung}
Fraglich ist, warum \texttt{label} nicht erfasst werden sollte, obwohl die folgenden drei Wörter übersetzt werden. Ein String, welcher menschliche Sprache mit Sonderzeichen vermischt, kann dahingehend interpretiert werden, dass diese Sonderzeichen \textit{wie} Klammern verwendet werden. Seinerseits könnte \verb|:| also nicht als bekanntes \enquote{geteilt} aufgefasst werden, sondern als Klammern. Ersetzt man diese Klammern mit Leerzeichen resultiert aus dem ersten Beispiel \verb|\label problem encounter solve| und in zweitem Beispiel \verb|\section example|. Zu sehen ist hier also bereits, dass Google Translate bei einer, wenn man es so interpretieren möchte, \enquote{Vernestung} zweiten Grades scheitert, jedoch einfache Vernestungen noch erkennt\footnote{\enquote{Vernestung} meint die Verschachtelung von Klammern}.

\paragraph*{Takeaway}
Teile der \TeX{}-Syntax lassen sich anhand von \verb|\|, \verb|{|, \verb|}|, \verb|[|, \verb|]|, \verb|$|, \verb|$$| oder \verb|\%| erkennen und müssten daher ausgeschlossen werden. Anders als in mathematischen Formeln zeigen sich Sonderzeichen jedoch nicht paarweise auf, sodass sie nicht paarweise ignoriert werden können. Man kann sich diese Art von Fehlern wie 0-dimensionale Fehler vorstellen, wobei die nullte Dimension hierbei bei einem einzelnen Wort beginnt (welche als Punkte verstanden werden).% von mir zumindest



% Hiermit sagen wir: Kanten/Elemente der ersten Dimension sind veränderbar.
\subsubsection{Leerzeichen}\phantomsection\label{problems:dim1}% Meint: Wort wurde nicht als Syntaktisch relevant erkannt; Wort dürfte nicht übersetzt werden
Dem vorangegangenem Beispiel (zunächst) widersprechend, offenbart ein Auslassen von Zeichen (und dadurch ein Trennen von Worten) eine Vielzahl anderer möglicher Fehler, welche sich glücklicherweise schnell einheitlich beschreiben lassen.% oh fuck non onon on n on onn on. .... habs unironisch gedacht und geschrieben gerade. .-d d..d . .d .a. .d. . wtffff
Hierunter fallen meist freistehende Worter, welche als Parameter für verschiedene \LaTeX{} Umgebungen dienen. Die Übersetzung solcher Parameter kann schnell zu Fehlern in einem \TeX{}-Compiler führen.

\begin{table}[h!]
    \centering
    \begin{tabularx}{\textwidth}{X X}
        \toprule
            Original & Übersetzung\\
        \midrule
        % Siehe ~/tests/readme.md für namensgebung und "Wo ist die Datei?"
            Korrekt & \\[-13px]
            \commoncode{Test}{../examples/technical/1d/correct_original.tex} & \commoncode{Test}{../examples/technical/1d/correct.tex}\\[1em]
            Unerwünscht & \\[-13px]
            \commoncode{Test}{../examples/technical/1d/wrong_original.tex} & \commoncode{Test}{../examples/technical/1d/wrong.tex}\\[-1em]
        \bottomrule
    \end{tabularx}
    \caption{Fehler in einem einzeiligen Dokument}\phantomsection\label{tab:problems:dim1}
\end{table}
\paragraph*{Verdeutlichung}% Warum das ein Problem ist
Die Optionen innerhalb eckiger Klammern lassen Whitespace zu. Dies kann jedoch für die Nutzung einiger Funktionen in z.B.\ wichtigen Paketen wie \texttt{hyperref} dazu führen, dass falsche Wörter übersetzt werden, die ein Kompilieren des Dokumentes verhindern.

\paragraph*{Takeaway}% ich nutze "take-away" statt "lessons-learned", da "lessons-learned" bwl-talk ist.........!
Teile der \TeX-Syntax lassen sich nicht nur anhand der~\hyperref[problems:unexpectedCharacters]{zuvor} beschriebenen Zeichenketten erkennen, sondern lassen sich auch in Zeilen wiederfinden. Diese Art von Fehlern bahnt den Weg zu einer Dimension, wodurch nicht nur innerhalb eines Wortes (Punktes), sondern auch zwischen verschiedenen Punkten Fehler entstehen könnten (also innerhalb einer Zeile).


\newpage








% Hiermit sagen wir: Flächen/Elemente der zweiten Dimension sind veränderbar.
\subsubsection{Zeilenbrüche}\phantomsection\label{problems:dim2}
Abstrahiert man nun über einzelne Zeilen hinweg, so wird die folgende Art von Fehlerquelle direkt offensichtlich, sodass sie keiner detaillierten Schilderung mehr bedarf. Sollte man versuchen ein Dokument Zeile nach Zeile zu übersetzen und den Kontext der vorigen Zeile zu ignorieren, so werden schnellig Zeilen übersetzt (obwohl:\ diese Zeilen nicht übersetzt werden durften, da sie Befehle für \TeX{} beinhalten (siehe:~\ref{tab:problems:dim2})).

\begin{table}[h!]
    \centering
    \begin{tabularx}{\textwidth}{X X}
        \toprule
            Original & Übersetzung\\
        \midrule
        % Siehe ~/tests/readme.md für namensgebung und "Wo ist die Datei?"
            Korrekt & \\[-13px]%
            \commoncode{Test}{../examples/technical/2d/correct_original.tex} & \commoncode{Test}{../examples/technical/2d/correct.tex}\\[1em]%
            Unerwünscht & \\[-13px]%
            \commoncode{Test}{../examples/technical/2d/wrong_original.tex} & \commoncode{Test}{../examples/technical/2d/wrong.tex}\\[-1em]%
        \bottomrule
    \end{tabularx}
    \caption{Das Übersetzen der \texttt{hyperref} Optionen würde den Kompilier-Prozess scheitern lassen}\label{tab:problems:dim2}
\end{table}



\newpage




% Hiermit sagen wir: Körper/Elemente der dritten Dimension sind veränderbar.
\subsubsection{Dokumentenbrüche}\phantomsection\label{problems:dim3}
Noch abstrakter wird hier die dritte Dimension erreicht, indem Teile von \TeX{} nicht in einer einzelnen Datei vorliegen müssen, sondern auch in anderen Dateien vorliegen könnten. Diese Tasache wirft eine Vielzahl neuer (und teilweise system-abhängiger) Probleme auf. Hier wurd sich zunächst jedoch nur auf die Fähigkeiten der unveränderten \TeX{}-Engine konzentriert und deren vorgesehene Primitiven (für diesen Zweck).

\begin{table}[h!]
    \centering
    \begin{tabularx}{\textwidth}{X X}
        \toprule
            Original & Übersetzung\\
        \midrule

        % Siehe ~/tests/readme.md für namensgebung und "Wo ist die Datei?"
            Unerwünscht & \\[-13px]%
            \commoncode{Test}{../examples/technical/3d/correct_original.tex} & \commoncode{Test}{../examples/technical/3d/correct.tex}\\[1em]%

            Übersehen & \\[-13px]%
            \commoncode{Test}{../examples/technical/3d/wrong_original.tex} & \commoncode{Test}{../examples/technical/3d/wrong.tex}\\[-1em]%
        \bottomrule
    \end{tabularx}
    \caption{Übersetzung in einem \texttt{include} führt zum nicht-Übersetzen einer Datei. Ein zunächst nicht sonderlich \enquote{interessant} wirkendes Beispiel}\label{tab:problems:dim3}
\end{table}



%%%%%%%%%%%%%%%%%%%%%%%%%
%%%%% AB HIERRRRRRRRRRRRR GEHHHHHHTSS WEITER. BEDENKE ABSENDEFRIST BIS MIDNIGHT!



% Bleibt die kontextuelle Richtigkeit bestehen? Wird beispielsweise aus einer Referenz ein anderer Kontext übermittelbar, welcher eine andere Übersetzung provoziert?
\subsection{Sprachliche Semantik}\phantomsection\label{problems:linguistical}
\subsubsection{Ausgangssituation}
Probleme des vorherigen Teils~\ref{problems:technological} zu verhindern, stellt sowohl ein Kompilieren/Entstehen eines übersetzten Dokumentes sicher, als auch ein Erkennen alelr Inhalte von diesem. Dies ist jedoch als minimale~\hyperref[technologies:demands]{Anforderung} zu sehen sein. Die Fähigkeit alle Inhalte eines \LaTeX{} Dokumentes lesen zu können alleinig, zeigt sich jedoch als unausreichend, da das Übersetzen zwischen menschlichen Sprachen kontextuell in verschiedenen Lexemem (Wörtern) enden sollte.%\footnote{Das Übersetzen aus und in Gebärdensprachen ist vorerst nicht im Fokus. Denkbarer Ansatz wäre hier allerdings Dokumente mit Bildern zu erwarten, bei welchen man davon ausgeht, dass diese mit bekannten Methoden der Mustererkennung erkannt und in Wörter (wie sie in den lateinischen Sprachen verwendet werden) überführt werden können und dann maschinell ausgewertet werden können. Hierzu gibt es zahlreiche Technologien~\citep{expertSystemsWithApplications:rastgooRazie2021:signLanguageRecognitionADeepSurvey}, jedoch beschäftigt dies ein Übersetzen aus einer menschlichen Sprache nach \LaTeX{} und weicht damit von der gegebenen Aufgabenstellung ab. (Obwohl auch hierfür Ansätze existieren:~\cite{cornellComputerScienceComputerVisionPatternRecognition:korzhDmitrii2025:speechToLaTeXnewModelsAndDatasetsForConvertingSpokenEquationsAndSentences})}.
% Hier die eigentliche Annahme
Dieser Kontext kann einem Quellcode aus mehr als nur Wörtern entnommen werden. Das Bestimmen des Kontextes erfordert Kenntnis über \textit{bestimmte Elemente} eines Dokumentes, welche nicht vorhersehbar sind und charakterisiert werden könnten. Elemente dieser Art können sich verschieden äußern und sind nicht zwangsweise standardisiert, sodass das Einhalten von eventuell etablierten Paradigma (hinsichtlich einiger instituellen Kontexte) geprüft werden muss. 

Vorangegangenes zeigt dementsprechend erneut eine üppige Menge an erwartbaren Problemen. Damit Grenzen der Realität erhalten bleiben, % siehe Commit: 2025-10-13
müssen sich denkbare Anwendungsfälle von \LaTeX{} auf\ (z.B.) wissenschaftliche Arbeiten oder Veröffentlichungen konzentrieren und die dafür benötigten Darstellungsmöglichkeiten in den Vordergrund rücken (bspw.\ Zitationen, Tabellen, Formeln, Graphiken,\ldots).% Graphiken hier mit Beschreibung zu erwarten. Wird diese erfasst? Was wenn es sich um einfache Plots handelt, jedoch die Formel in der Caption nicht der Formel, welche eigentlich geplottet wird, gleicht? An welcher stelle wird der Kontext erfasst? Wie beim Lesen von oben nach unten oder _rein_ aus der Caption?
Exemplarisch werdem daher Beispiele aufgezeigt, deren Phrasen einen Kontext tragen, der nicht aus der sprachlichen Satzstellung allein hervorgeht. 
% Kontexte nicht aus der sprachlichen Satzstellung alleine hervorgehen.\\% Hier zu unterscheiden von: Wie könnten einzelne Wörter Fehler in TeX produzieren, schon behandelt.



% Examplarische Beispiele: Wenn unendlich viele Beispiele theoretisch gesehen möglich wären, aber "unendlich viel" nicht in die Abschlussarbeit passt. 
\subsubsection{Exemplarische Beispiele}
\paragraph*{Autoren/Titel}% Hier wird noch davon ausgegangen: Alle Info's liegen in dem Dokument vor, eher weniger Bezug auf etwas Technisches. 
%
% Kann der Ansatz eigenständig recherchieren ist die Frage?
%
% Fokus auf einen (1) Autoren. Vielleicht ist (1) ja bekannt. Hey wie cool.
%
Das Erwähnen des wortwörtlichen Namens oder Titels eines Dokumentes/Werkes wäre innerhalb dieses normalerweise nicht zu erwarten, aber aus diesem lassen sich unter Umständen mehr Informationen über den Kontext gewinnen (und:\ beide sind normalerweise in der Präambel eines Dokumentes vorhanden). Handelt es sich um einen bekannten Autoren könnte der Name einen Kontext liefern (welcher jedoch im Vorwege nur als eine Heuristik angesehen werden kann, da der Autor authentifiziert werden muss) oder der Titel trägt den Namen von einer bekannten Methode aus einem wissenschaftlichen Gebiet oder nennt ein solches Gebiet. Denkbare Beispiele umfassen:\ 
\enquote{$[\ldots]$\ of Communication}, \enquote{Bayesian $[\ldots]$}, \enquote{$[\ldots]$\ inference\ $[\ldots]$}
(entnommen aus einigen \textit{subject areas} von ACM:\ Transactions on Probabilistic Machine Learning).
Vergisst aber ein übersetzendes Programm diese Wörter des Titels an einer kritischen Stelle im Text, so würden fälschliche Übersetzungen entstehen.

%
% Kann der Ansatz eigenständig recherchieren ist die Frage?
%
% Fokus auf einen (1) Autoren. Vielleicht vermerkt er eine Referenz auf ein vorheriges Kapitel im Dokument? Kann er wörtlich mit Kapitel 2.1.x oder per \hyperref[chapter2:section1:paragraphx]
%
%
\paragraph*{Dokumenten-interne Verweise}% Bezug auf Paragraphen im Werk mit \ref und Hyperref. Wieder technischeres Problem. Inhalte alle da, 
Hyperreferenzen \enquote{erlauben} zunächst (scheinbar) recht wenig, implizieren jedoch eine Giganz an Möglichkeiten. Es wäre zu erwarten, dass jegliche Information, welche für die Übersetzung eines Satzes erforferlich ist, textlich vorliegt. \TeX{} erlaubt es allerdings auf solch einen Kontext zu referenzieren (via:\ \texttt{ref} oder \texttt{hyperref}) und ihn dadurch zu \textit{implizieren}, statt zu nennen.


\paragraph*{Externe Referenzen}% beinhält: wild reingeknallte links. nicht best practise, aber entstandener Software auf der ersten Seite einen Link zu einem git anzugeben (Fußnote) ist ganz nett für Spätere. Hier: Ohne BibTeX!)
%
% Kann der Ansatz eigenständig recherchieren ist die Frage? 
% 
% Fokus ist nicht auf dem einen Autoren, der ein Werk schreibt, sondern ein beliebiges Werk, welches sich auf die Arbeiten anderer bezieht
%
In \LaTeX{} kann auf verschiedene Arten und Weisen auf anderweitige Quellen verwiesen sein. 


%
% Kann der Ansatz das Dokument verstehen ist die Frage?
%
% 
%
\paragraph*{Formeln, Tabellen}



\paragraph*{Graphiken}% Eig nicht so recht lösbar, nur annäherbar. Hierauf wird näher im "spezifische Technologien" Part eingegangen. Bedarf evtl. Hilfe des Users (ein wenig kästchen umherschieben, bis es "so gut wie möglich" alles lesbar ist)







\begin{comment}

%%%%%%%%%% Für später relevant, für Makros und beliebige Nutzereingaben etc......:!:!:!!::!!::!
%Die Möglichkeiten andere Dateien in einem \LaTeX{} Dokument einzubeziehen, könnten nähern sich einer Unendlichkeit.% Warum? ist schwer zu beantworten, daher sollte man sich zunächst vor Augen führen, für welche Zwecke man auf andere Dokumente innerhalb eines Dokumentes verweisen möchte. // Warum eine Unendlichkeit? Es gibt unendlich viele Unendlichkeiten. Beweis bitte von Prof. Cap
%Ein Beschäftigen mit dieser führt zu nichts\footnote{invers},%, sodass zunächst die Frage gestellt werden müsste, wie diese entstehen können, indem alle denkbaren Arten, wie sich zwei oder mehrere Dateien gegenseitig referenzieren/benötigen könnten, aufgelistet wird. 
%weshalb ein Betrachten möglicher versteckter interner Abhängigkeiten zwischen verschiedenen \LaTeX{} Dokumenten unabdingbar ist.% [...], hence describing the possible internal dependencies of \LaTeX{} documents is necessary. // um von der Formulierung "weshalb mögliche interne Abhängigkeiten zwischen verschiedenen Dateien erläutert werden müssen" wegzukommen, sah ich mich gezwungen in die englische Sprache auszuweichen (da es mir dabei half zu erkennen, wie ich einen direkten Konjunktiv II vermeiden kann).
%%% Inwiefern wirkt sich das auf die gegebene Problematik aus
%Nur einen Teil eines Dokumentes zu übersehen, provoziert kontextuellen Verlust für Übersetzungen (Abschnitt~\ref{problems:dim3}). 
%\begin{enumerate}
%%    \item Permutationen (insgesamt 4, da wir nur zwischen $1$ und $n \textit{(beliebig vielen)}$ zu wechseln wünschen (2*2 Optionen / Urnen=1;Kugeln=2;Farben=2;zurücklegen)): 
%    \item Ein Dokument kann ein Anderes in sich tragen. 
%    \item Ein Dokument kann n Andere in sich tragen
%    \item n Dokumente können ein Anderes in sich tragen
%    \item n Dokumente können n Andere in sich tragen geht aus den beiden vorigen statements hervor und ist redundant (weil wir kennen bereits 1 dokument, welches n Andere tragen kann).
%\end{enumerate}
% Realität es gibt nicht unendlich viele Dokumente, da physikalisch nicht möglich (Archimedis). 
%Hier stoßen wir 


%%% Meint: Hier müssen wir die eigentlichen Literaturverweise kennen, um einen Kontext zu kennen.
\subsubsection{Unerreichbare Informationen}

\paragraph*{Beispiele}
\begin{table}[h!]
    \centering
    \begin{tabularx}{\textwidth}{X X}
        \toprule
            Original & Übersetzung\\
        \midrule
        % Siehe ~/tests/readme.md für namensgebung und "Wo ist die Datei?"; hoffentlich sieht sich der Herr Prof. Dr. rer. nat. habil. nicht den Quellcode an dieser Stelle an.
            Dokument & \\
            \lstinputlisting[language=TeX]{../examples/advanced/literature/example_original.tex} & \lstinputlisting[language=tex]{../examples/advanced/literature/example.tex}\\[2em]
            Bibliothek & \\
            \lstinputlisting[language=tex]{../examples/advanced/literature/example_original.bib} & \lstinputlisting[language=tex]{../examples/advanced/literature/example.bib}\\
            %%% Bemerkung: Übersetzung noch nicht erstellt.
        \bottomrule
    \end{tabularx}
    \caption{Beispiel für einen verpassten literarischen Kontext}\label{tab:problems:nonexisting}% mit reference 
\end{table}

\paragraph*{Beschreibungen}
Ein Dokument erwähnt ein Werk, in welchem es um die C-Programmierung geht. Rein aus den im System vorliegenden Dateien ist kein Kontext für das Wort \enquote{String} erkennbar, sodass ein Zugriff auf eine externe Ressource unabdingbar ist.

% Bemerkung: Nur die Suche (google.com) nach "salomon c programmierung" führt bspw. Gemini zu einer vermuteten Verwechslung mit dem Begriff "System" (Stand: 09.10.2025, 12:29).
% ISBN führt zur gleichen Minute direkt zum Institut (Angewandte Mikroelektronik und Datentechnik)... wobei Thalia denkt, dass ich "1984" online kaufen möchte...
\paragraph*{Abstrahierung}
Einfache Cloud-Architektur. Ein Client möchte auf ein beliebiges Wissen einer Webseite (bzw.\ dem Server und den beanspruchten Speicherplätzen in einem (beliebigen) Rechenzentrum\footnote{Hierbei ist nicht von Festspeicher zu reden. Aus Sicherheitsgründen sei davon auszugehen, dass sich die physischen Adressen des wissensrepräsentierenden Speichers regelmäßig und unvorhersehbar ändern} zugreifen).




%\subsubsection{Figuren und Tabellen}\phantomsection\label{problems:advanced:tables}
%\paragraph*{Beispiele}
%\paragraph*{Beschreibungen}
%\paragraph*{Abstrahierung}

%\subsubsection{Literaturverzeichnisse}\phantomsection\label{problems:advanced:bibtex}
%\paragraph*{Beispiele}
%\paragraph*{Beschreibungen}
%Bib\TeX{} erlaubt es an vielerlei Stelle eigene Strings in einer kompilierten \TeX{}-Datei zu verbergen.
%\paragraph*{Abstrahierung}


%\subsubsection{Category Codes}\phantomsection\label{problems:advanced:catcode}
%\paragraph*{Beispiele}
%\paragraph*{Beschreibungen}
%\paragraph*{Abstrahierung}


%%%%%%%%%%%%%%%%%%%%%%%
%%%%%%%%%%%%%%%%%%%%%%%% In Review. Richtige Inhalte, inadäquate Sortierung.
%%%%%%%%%%%%%%%%%%%%%%%%


\end{comment}

\subsection{Spezifischer Technologien}\phantomsection\label{problems:special}% Verändern die Übersetzung nicht direkt / Können diese nicht direkt verändern, jedoch Unleserlichkeiten (unschön anzusehen) oder Unlesbarkeiten (nicht zu sehen) im kompilierten Dokument erzeugen
% Mitunter NP-schwer (zur behebung dieser muss man meist heuristiken eingehen. was ist eine heuristik? an sich gar nichts, meint nur, dass wir fehler zulassen (bzw. eine fehlerwahrscheinlichkeit)).
Hier wenden wir uns von Problemen einer Übersetzung ab und widmen uns denen eines Lesers. Alle textlichen Inhalte eines Dokumentes zu übersetzen, als auch eine kontextuelle Fachsprache zu bewahren scheint aus abstrakterer Perspektive ausreichen, kann allerdings zu Situationen führen, in welchen Informationen verloren gehen, da diese vom Endnutzer nicht mehr gesehen werden können.% Meint: Wirklich gesehen, da in der PDF hinter etwas Anderem verborgen 
% Passiert wann? Fläche, welche die Texte benötigen werden nach der Übersetzung grßer
% Dazu: Wie viel Fläche braucht die Sprache mit geringster Flächer (für alle grammatikalisch richtigen Wortkombinationen?)
% Dazu: Wie viel Fläche braucht die Sprache mit größter Flächer (für alle grammatikalisch richtigen Wortkombinationen?)
%%% Wie zu bestimmen: folgt.



\subsubsection{Kommentare}\phantomsection\label{problems:advanced:comments}
\paragraph*{Beispiele}
\paragraph*{Beschreibungen}
%- zunächst als Unterklasse von~\ref{problems:unexpectedCharacters} zu erwarten
%- kann jedoch auch~\ref{problems:verticalSpacing} umfassen
Wohingegen sich~\ref{problems:advanced:comments} nicht mit anderen, in Kommentaren referenzierten, Dateien beschäftigt, soll sich hier auf solche Fälle konzentriert werde.
\paragraph*{Abstrahierung}
Hier treffen technische Fehler aus den ersten drei Kategorien (in~\ref{problems:dim0},~\ref{problems:dim1} und~\ref{problems:dim2} geschildert) aufeinander. In die dritte Dimension, also in andere Dateien, wird jedoch (\hyperref[problems:special:comments]{vorerst}) nicht traversiert, da auskommentierte Datei-Einbindungen nicht erfasst werden dürften. 
Ausgehend von~\ref{problems:advanced:comment} wird nun erwartet, dass eine Referenzierung von Dateien erwartet wird, welche sich in Kommentaren verbergen. Dies kann jedoch~\ref{problems:special:sourcecode} beinhalten.


\subsubsection{Dilemmatische Makros}\phantomsection\label{problems:special:macrodilemma}% dilemmatasitische Makros... Kombi aus dilemmatisch und fantastisch?
\paragraph*{Beispiele}
\paragraph*{Beschreibungen}
\paragraph*{Abstrahierung}

\subsubsection{TikZ und Layouting}\phantomsection\label{problems:advanced:layouting}
\paragraph*{Beispiele}
\paragraph*{Beschreibungen}
\paragraph*{Abstrahierung}

\subsubsection{Quellmehrsprachigkeit}\phantomsection\label{problems:special:sourcecode}
\paragraph*{Beispiele}
\paragraph*{Beschreibungen}
\paragraph*{Abstrahierung}
Quelltexte anderer Quellsprachen (Programmiersprachen) können ihrerseits auf andere Dateien verweisen, oder andere Syntaktik tragen. Das Erkennen dieser ist theoretisch gesehen leicht, jedoch praktisch gesehen schnellig zu übersehen. 









\subsection{Weitere Schwierigkeiten}\phantomsection\label{problems:additional}
\paragraph*{Kommentare}


\subsubsection{Glossare und Nomenklaturen}
\paragraph*{Beispiele}
\paragraph*{Beschreibungen}
\paragraph*{Abstrahierung}

\subsubsection{Weitere}
\paragraph*{Beispiele}
\paragraph*{Beschreibungen}
\paragraph*{Abstrahierung}

%\section{Technologischer Stand und denkbare Lösungswege} % Zu kürzen

% Welche zugrundeliegenden Übersetzungstools gibt es? Nicht: Wie funktionieren diese, sondern: Welche gibt es, die man nutzen könnte, für spätere Workflows!!!
\subsection{Generische, fundamentale Technologien} % impliziert "fundamental" hier einen Tech-Stack? also, dass alle folgenden gelisteten Technologien auf dem hierigen ChatGPT basieren? Wäre nämlich inhaltlich falsch, 
Google Translate wurde während der Erläuterung im vorherigen Kapitel herangezogen, da sich mit dieser Ressource recht einfach typische Fehler bei unbedachter Übersetzung von \LaTeX{} Dokumenten aufzeigen lassen. Solche Software erwartet als einen Input immer ein Lexem (Wort) einer menschlichen Sprache, welche (aktuell) keine \enquote{Sonderzeichen}, bzw.\ Symbole beinhalten (gemeint sind z.B.\ \verb|#,\,§,=,+,-,...|). Ein Mensch kann solche+ Zeichen im Lesefluss ignorieren, ein Programm (bzw.\ ein Computer) kann dies ohne (ein) Weiteres jedoch nicht.% Passt im Kontext Programm recht gut in dieser Formulierung. 

% Bitte nochmal reviewen ja? Ich sehe hier Use-Case... das Wort bitte nicht... i hate bwler 
Jedoch basieren heutige Technologien zur Sprachübersetzung nicht mehr auf Programmen, welche einen festen Input mit syntaktischen Vorgaben erwarten, sondern auf weitaus mächtigerer Software, welche je nach erforderlichem Use-Case eine andere Art an Input erwartet. Als Endnutzer würde man sich zunächst jegliche Optionen offen halten wollen, wodurch man recht schnell bei den Technologien größerer und bekannter Anbieter angelangt, welche Anwendungs- und Nutzerschnittstellen für ihre Sprachmodelle anbieten. Von den Dingen, welche aus marktwissenschaftlicher Argumentation heraus direkt weniger vielversprechend sein \textit{müssten} (da diese Anbieter andere Software zur Dokumentenerstellung produzieren, welche sie verständlicherweise gegenüber einen kostenfreien Technologie wie \TeX{} in den Vordergrund stellen möchten), wird zunächst noch nicht abgesehen, sondern jegliche denkbare Technologie hinsichtlich ihrer potentiellen Möglichkeiten betrachtet. 

\paragraph*{ChatGPT}\label{par:chatgpt}
Der von OpenAI präsentierte statistische Ansatz zur Entwicklung einer künstlichen Intelligenz ist ein im ersten Moment naheliegender Ansatz, da man sich hier erhoffen könnte, dass diese Technologie mit Hilfe eines passenden \enquote{Prompting} auf längere Zeit gesehen sowohl qualitativ hochwertige Übersetzungen erzielt, als auch geschickt jede \TeX{}-Syntax geschuldete Hürde umgeht. Allerdings scheitert dieser Ansatz% zum Glück
bereits konzeptionell, denn der \TeX{}-Compiler (jeder) selbst wird dazu in der Lage bleiben \textit{nur} reine Zeichenketten von Befehlen, Makros und Ähnlichem zu unterscheiden, da diese ansonsten nicht wie gewünscht in einem Dokument als solche Zeichenkette vorzufinden wären. Da dieses Programm selbst auch deterministisch arbeitet (und arbeiten muss), benötigt man an dieser Stelle noch keine Einbindung einer potentiellen Fehleranfälligkeit.

%%% Kurze Bemerkung zur Fragestellung jeweils:
%%% Wie sieht die API aus? Kann man überhaupt Kontext (z.B. TeX, in TeX geschrieben, ...) mitgeben? Falls nein: gar nicht weiter verfolgen.
%%% Gibt es solch umfangreiche Glossare wie in DeepL? Bzw. wie tuen sich die Anbieter mit menschensprachlichen Kontexten? (Angaben der Anbieter; sollten die einwandfrei mit TeX umgehen, dann gezielte sprachliche Fehlerquellen prüfen... des wird dann aber sehr aufwendig und erfordert unweigerlich dann eine BLEU)
%%% Kann man die Kontexte: "Das ist ein LaTeX"-Dokument und "Es geht um Thema XY" gesondert mitgeben, damit dieser Unterschied klar wird?
%%% Falls alles ja ergibt (was ich bezweifle): Gezieltes Testen auf die gelisteten Problemfälle! (Aber nur dann)
\paragraph*{DeepL}\label{par:DeepL}
\paragraph*{Microsoft Translate}\label{par:Microsoft Translate}
\paragraph*{Google Cloud Translate}\label{par:Google Cloud Translate}

% Wie könnte man das Problem lösen? Eingehen auf denkbare Workflows...
\subsection{Denkbare Workflows}
Setzt man sich vor die zunächst etwas einfacherere und bisher noch offene Problemstellung, wie man überhaupt alle textlichen Inhalte eines \LaTeX{} Dokumentes \texttt{computationally} erfasst,% gehört deutsch-syntaktisch dahin, english speakers would place an adverb after the verb itself
sind verschiedene Wege denkbar. Ausgehend von einem \TeX{} \enquote{Main}-Code könnte man diesen (a) übersetzen und dem zusehen, dass die \TeX{}-Syntax erhalten bleibt% schwer wegen CatCode %% VOR DEM KOMPILIEREN %%% Einzig logische bzgl der Aufgabenstellung
(b1) einen \TeX{}-Compiler dahingegen verändern, dass dieser zu übersetzende Strings zusätzlich in eine eigene Datei ausgibt, auf deren Grundlage übersetzt werden kann 
(b2) den \TeX{}-Kompilier-Prozess so zu ändern, dass während diesem jegliche textliche String-Token an ein übersetzendes Tool gesendet werden oder
(c) in ein anderes Format zur Beschreibung von Dokumenten überführen, wie bspw.\ (HTML, XML, PDF,~\ldots)%%% NACH DEM KOMPILIEREN
und ausgehend von diesem übersetzen. 

% Ändern eines Compilers...
\paragraph*{Workflows, von welchen abzusehen ist}% Argumentiert gegen b2
In erster Linie wäre es durchaus denkbar, dass man den Fakt ausnutzt, dass ein \TeX{}-Compiler jegliche rein/wirklich textlichen Strings (welche in kompilierter PDF als diese angezeigt werden) selbstständig und während der Kompilierung an eine API eine der zuvor gelisteten Übersetzungs-Tools sendet und danach wieder im Dokument integriert. Dies mag für Workflows funktionieren, in welchen \TeX{}-Dokumente nur selten kompiliert werden, jedoch bieten heutige Mittel die Möglichkeit eines Live-Rendering, bei welchem sehr schnell und öfters kompiliert wird (da ein Endnutzer schnellstmöglich dessen Änderungen im Dokument sehen möchte). Dies impliziert, dass bei jeder Kompilierung der \TeX{}-Datei auf eine entsprechende und kostenpflichtige API zugegriffen werden würde, wodurch schnellig und nicht direkt bemerkte Kosten anfallen könnten, wenn man nicht bedenkt, dass man auch nur die wirklich veränderten Teile (seit letzter Übersetzung) an ein Übersetzungstool senden müsste. Hierbei wird jedoch fraglich, inwiefern der gesamte Kontext des Dokumentes erhalten bleibt, bzw.\ wie man diesen live bestimmt. Was geschieht, wenn größere Textblöcke gelöscht werden und dadurch Kontext für spätere Textblöcke verloren geht? (Im Sinne:\ Wir sprechen von Zeile 1 bis x von Cybersecurity und löschen Zeilen 1 bis y, wobei y<x. Soll und muss dann der Kontext Cybersecurity erhalten bleiben? Woher kann und soll ein live-agierendes Programm dies wissen, wenn es immer nur Zugriff auf die in diesem Moment vorliegenden Texte hat, in welchen \textit{nun} kein Cybersecurity-Kontext mehr vorliegt\ldots und demnach nicht mehr in diesem Kontext richtige Übersetzungen gewählt werden können.)%!!!!!!!!!
Wann und wie kann man also, rein von der Menge des editierten/gelöschten Textes absehen, wann und wie sich der Kontext eines Dokumentes ändert? Die Logik innerhalb eines Dokumentes könnte sich bereit dass vollständig ändern, sollte man nur ein (sehr frühes) \enquote{nicht} löschen, wodurch die Logik fortan inversiert wäre.


%%% Nicht Teil der Aufgabenstellung, aber gut zu behalten für Ausblick (Was könnte man noch zstl. machen) Ich brauch mehr Zeit
\paragraph*{Nach TeX} entstehende PDF könnten theortisch gesehen auch als Grundlage innerhalb eines automatischen Workflows denkbar sein. Denkbar sind Technologien, welche z.B.\ eine PDF in HTML oder XML zerlegen und diese als Grundlage nutzen, um eine automatische Übersetzung, für welche bekannte Browser eine Unterstützung bieten, zu nutzen. Diese Übersetzten HTML danach wieder in PDF zu übertragen ist noch leichter, da dies die herkömmlichen Browser ohnehin können. Und wem das Kompilieren auf diesem Weg zu lange dauert, darf sich in der Zwischenzeit dem Dokument in HTML/XML bedienen, welches ein Browser anzeigen kann. Sollten hiernach noch kleinere Änderungen von Nöten sein, so müsste nur noch eine entstandene, übersetzte PDF nur noch in den erwähnten Überlappungen innerhalb Ti\textit{k}Z aufgrunde von unabdingbaren sprachlichen Verhältnissen angepasst werden, was dann einmalig in HTML erfolgen müsste. Hierbei könnten dann jedoch auch menschliche Spracheditoren den entstandenen Dokumenten einer Kontroll-/ Korrekturlesung unterziehen, sollte sich solche Menschen finden lassen. Dies erscheint in erster Betrachtung nachvollziehbar und logisch, sieht man jedoch genauer hin, so werden Schwierigkeiten bei dieser Herangehensweise offensichtlich, da man hier wohl kaum von einer gänzlich automatischen Übersetzung von \LaTeX{}-Dokumenten sprechen kann. Aus abstraktem Blickwinkel sind hier zwei Workflows denkbar. Zum einen könnte man \TeX{} und Ti\textit{k}Z gesondert voneinander betrachten, indem man sämtliche Ti\textit{k}Z-Graphiken einzeln in idealerweise skalierbare Vektorgraphiken kompiliert und innerhalb einer nach HTML exportierten \TeX{}-Datei als solche einbindet. Sämtliche textliche Strings sollten dann in dieser SVG vorzufinden sein, genauso wie es der Fall in XML ist (wodurch man sich hier auf bestimmte, und anhand eckiger Klammern identifizierbaren Tags, berufen kann, siehe: https://svgwg.org/svg2-draft/text.html#TextElement).% GOD FUCKING DAMNIT WIE SIEHT DENN SVG JETZT AUS????? IST DAS SPEZIFIZIERT? Einerseits: Ich hoffe ja: dann implementierung einfach, Andererseits: Ich hoffe nein, dann muss ich es nicht implementieren... JA ICH HABE BOCK AUF CODING UND WÜNSCHTE MIR EIGENTLICH NUR ENDLICH MAL DIE ZEIT FÜR EIN RICHTIGES, RICHTIGES, RICHTIGES CODING-Projekt... warum überhaupt compsci studieren alter. da macht man alles, außer zu programmieren, wobei programmieren das Einzige ist, was einem dabei hilft zu verstehen, wieso die Dinge im Studium beim Programmieren überhaupt jemals relevant geworden sind.
%... das wirkt auf den ersten Blick wie etwas, was bereits bei einem Kompilieren von TeX+TikZ zu HTML auftreten könnte und daher ein Feature eines Compilers sein könnte
% ich möcht auch mal was anderes machen als uni uni uni uni uni uni uni uni uni uni uni uni uni uni ABER WIE DENN BITTESCHÖN







% Hier beginnen die genauen! Fehler zu zeigen, jeweils an den Softwares
%% Test-Database: Arxiv. Siehe: einer der vielen coolen Links :) vllt lernst du es endlich mal Quellen ordentlich abzuspeichern... >:)
\subsection{Existierende Ansätze}

\paragraph*{Paper Ohri und Schmah}
\paragraph*{PlasTeX} und dann DeepL
\paragraph*{TransLaTex}
% outlier
\paragraph*{MathTranslate}


\paragraph*{Die \TeX{}-Compiler Familie} und rein innerhalb \TeX{}-basierte Ansätze sind durchaus denkbar, erfordern jedoch einen höheren menschlichen Aufwand. Auf diese Art und Weise wäre eine Trennung von \TeX{} und menschensprachlichen Texten direkt und ohne weiteres Nachdenken gegeben, sodass hierbei zwar Verluste hinsichtlich der Kompilationszeiten entstehen könnten, jedoch die gegebene Aufgabe technisch gesehen erfüllen könnten. Bemerkt sei hierbei, dass solch ein Ansatz natürlich bei Live-Editoren, wie beispielsweise Overleaf weniger geeignet wären, da diese für ihr Live-Rendering von PDFs bei jeder Veränderung des \TeX{}-Dokumentes 
\subparagraph*{Das translate package} und dessen Glossare anpassen
\subparagraph*{Einen Compiler} anpassen wäre möglich, sodass dieser alle Textstrings (hierbei: abhängig vom Token-Typ, welchen der \TeX{}-Parser liefert) in z.B.\ YAML überführt, welche ihrerseits als Input für eine KI genutzt werden könnte, sodass mit Recht wenig Kontext weiß, welche Token sie übersetzen soll und in welchem Kontext sie stehen. Jedoch braucht dies nicht unbedingt tiefersitzend in einem spezifischen Compiler verankert integriert werden, sondern ließe sich auch, wie vorangegangene Ansätze des Abschnittes es zeigten, auch gesondert lösen. Daher sollte zunächst die Frage geklärt werden, inwiefern es sinnvoll ist große, komplexe und dadurch nur mit hohem Zeitaufwand nachzuvollziehnde Technologien als nicht-offizieller Mitentwickler zu verändern, sodass bei jedem offiziellen, neuen Release dieser Software inoffizielle Änderungen \enquote{mitgeschliffen} werden müssten, oder ob es nicht vorzuziehen wäre, dass man ein gesondertes Problem getrennt löst, sollte dies, wie hier, möglich sein.% -> hier aber überleiten zu: dafür brauchen wir nicht ein Programm in einem Compiler, sondern können auch ein kleineres, unabhängiges und leichter zu verstehendes und damit leichter wartbares Programm erstellen



\endgroup

\newpage

\makeatletter
\section{Eigenständigkeitserklärung}
Ich versichere hiermit, dass ich die vorliegende Arbeit selbstständig angefertigt und ohne fremde Hilfe verfasst habe. %, es sei denn die Aufgabenstellung verlangte unausweichlich einer (maschinellen) Unterstützung
Dazu habe ich keine außer den von mir angegebenen Hilfsmitteln und Quellen verwendet und die den benutzten Werken inhaltlich und
wörtlich entnommenen Stellen habe ich als solche kenntlich gemacht. 
Ich versichere, dass die eingereichte elektronische Fassung mit den gedruckten Exemplaren übereinstimmt.
\vspace{2cm}
\begin{figure}[b]
	\raggedright{}
	Rostock, den \@date\\[8ex]% \date setzt das Datum (eine Variable), während \@ zuerst impliziert: Wir befassen uns mit einem Befehl, der eine Variable sein könnte und das @ bestätigt: es ist eine Variable und wir wollen auf den Wert zugreifen.
	$\overline{\text{\@author}}$% overline = linie über dem umklammerten, \text: hier steht Text in einer mathematischen Formel, 
	\vspace{2cm}
\end{figure}
\makeatother

\newpage


\bibliography{index}

\begingroup
\hypersetup{hidelinks,pdfborder={0 0 1},allbordercolors=magenta}
\newpage
\renewcommand\thesection{\Alph{section}}
\setcounter{section}{0}
\renewcommand*{\theHsection}{chX.\the\value{section}} % Danke an: https://tex.stackexchange.com/questions/71162/reset-section-numbering-between-unnumbered-chapters
% \thesection = der TeX interne Abschnittscounter
% \theHsection = der Hyperref interne Sektionencounter. Arbeite zwar nur in einem chapter, jedoch ist chX.0 ein neuer String ggb. (wahrscheinlich) ch0.0
\section{Anhänge}
\endgroup

\end{document}