\section{Beitrag}% Eigenwirkung.
\subsection{Existierenden Herangehensweisen}% Meint: Welchen Lösungsweg sind die anderen, gefundenen Ansätze auf abstrakterer Ebene gegangen. Abzugrenzen von der Auswertung der Tests. 
\subsubsection{Übersetzen des Quellcodes}
Diese Herangehensweise verfolgt den Weg den gesamten Quelltext, ohne größere Interpretation, als Eingabe für einen Übersetzer zu nutzen, um daraus einen neuen, übersetzten Quelltext zu erhalten, in welcher sämtliche \LaTeX{} relevanten Inhalte unverändert (ggb.\ dem Original) vorzufinden sind. Dieser neue Quelltext soll dann, genauso wie der Originale, mit Hilfe eines bekannten \TeX{}-Compilers (bspw.\ \texttt{pdflatex}, \texttt{xelatex}, \texttt{lualatex}, \ldots) in eine PDF überführt werden.

\subsubsection{Quelltext-filternde Ansätze}% Bildung des Syntax-Graphen, bzw. Filtern aus der Token-Menge
\subsubsection{Ausweichen in andere Formate}
% Meint eigene oder sowas wie XML... muss evtl nach "Andere"
\subsubsection{}
\subsection{Andere Lösungswege}
\subsubsection{Abstrakte Beschreibung}% Haben Workflow A, B, C, ... mit Rollen 1, 2, 3, .... Brauchen Technologien, die diese Rollen einnehmen können.
\subsubsection{Technische Umsetzungsmöglichkeiten}
\subsection{}