% Welche Arten von Fehlern können entstehen? (Was fällt einem auf, wenn man Dokumente schreibt und sich denkt: mmmmhh würde google translate das richtig übersetzen)
\section{Problemfälle}
Eine einzige, feste \TeX{}-Syntax existiert theoretisch gesehen nicht, wie ein~\hyperref[problems:advanced:catcode]{späterer Paragraph} aufzeigen wird.% So gehen in-document refs ganz gut.
Die Fähigkeit jegliche erdenkliche Zeichenkette (gegeben:\ diese ist auf einem Rechner darstellbar, siehe:~\cite{unicode}) sorgt zunächst für eine unendliche Menge an testbaren Problemen. Da es unmöglich ist eine unendliche Menge an Testfällen abzudecken, wird zunächst nur die vorgesehene \LaTeX{} (bzw.\ \TeX{}~Syntax nach~\cite{texbook}) betrachtet und die bereits rein innerhalb dieser schnellig auffallenden Fehler aufgezeigt, welche durch fälschlich übersetzte Zeichenketten entstehen könnten.\\\noindent
% reviewed: 1
% 
\subsection{Struktur}\phantomsection\label{problems:structure}% Warum Kategoriesieren?
\subsubsection{Klassifizierung}
Für spätere Testzwecke wurden die verschiedenen Problemfälle in einzelne Kategorien getrennt. Eine solche Kategorisierung ist technisch gesehen nicht zwingend, soll allerdings zur Verbesserung einer späteren Übersicht dienen. Die Einteilung konzentriert sich vorrangig auf die Komplexität des Problemes und in einer Reihenfolge, in welcher sie auch bei der Nutzung von \TeX{} für ein beliebiges Dokument auftreten könnten. 
%
% Definition: Direkte Probleme
Hierbei seien \textit{direkte Probleme} zunächst eher simple und technisch leicht zu behebende Probleme, welche sich durch ein einheitliches Vorgehen beheben lassen könnten.% Könnten muss an dieser Stelle so folgen
Zudem beschränkt man sich hier nur auf den benötigten Zugriff auf eine einzelne Datei.
%
% Definition: Indirekte Probleme und Zusammenfassung zu "simplen Problemen"
\textit{Indirekte Probleme} formen potentielle Schwierigkeiten, da sie einen Zugriff auf weitere Dateien benötigen, da sie von einer Ausgangsdatei referenziert werden. Technisch gesehen sind sie jedoch auch durch einheitliche Paradigma zu bewältigen. Daher werden sie im Kapitel~\ref{problems:simple} zusammengefasst und fortan als \enquote{simple Probleme} bezeichnet.
%
% Definition: Fortgeschrittene Probleme
\enquote{Fortgeschrittene Probleme} beziehen sich auf Probleme, welche sich paradigmatisch lösen lassen, % man kann einen Quellcode schreiben; paradigmatisch = Mustern folgend
aber zusätzliches Vorwissen verlangen. Dieses Vorwissen lässt sich bei diesen Problemfällen jedoch ermitteln, da der Entstehungsweg dieses Wissens nachvollziehbar ist.% Meint: Wir gucken uns an, wo und wann zu übersetzende Wörter hinter Makros, in Paketen, anderen >>Programmen<< versteckt wird.
\enquote{Speziellere Probleme} umfassen Probleme, welche sich nicht ohne Vorkenntnis des tatsächlichen Dokumentes beheben lassen. Nach diesen wird zudem noch auf ein paar rein sprachliche Schwierigkeiten eingegangen, welche unter Anderem zu solchen speziellen Problemen führen könnten.

\subsubsection{Beschreibung}
Jeder Problemfall wird zunächst durch ein Beispiel demonstriert\footnote{welche aus einem Test mit Hilfe von Google Translate in Firefox durchgeführt wurden (Oktober 2025)}, danach wörtlich in einer Beschreibung erläutert und wird abschließend abstrahiert und versucht auf konkrete \TeX{} Primitiven zurückgeführt zu werden. Die fortgeschritteneren Probleme bedürfen teilweise mehreren Beispielen zur Beschreibung, lassen sich jedoch auf ähnliche Äußerungen zurückführen. Einführende Beispiele werden in der Form \texttt{ausführbarer Code} wird zu \texttt{ausführbarer Code} übersetzt, aber \texttt{ausführbarer Code} zu \texttt{fehlerhafter Code}\footnote{\enquote{Code} bezieht sich in beiden Fällen auf \TeX{}-Quelltexte} (1-dimensional). Einige Beispiele benötigen mitunter eine 2-dimensionale Darstellung, da sie mehrere Zeilen Quellcode umfassen und werden tabellarisch nebeneinander gestellt~\pdfcomment[color=red, icon=Paragraph]{Evtl.\ werden alle Beispiele in tabellarische Form überführt aus. Finde ich persönlich sehr ansehlich}. Einige Probleme auf höherer Abstraktionsebene lassen sich allerdings nur schwierig veranschaulichen, wodurch Beispiele sich teils wieder auf Schilderungen von Situationen berufen, um diese Übersicht übersichtlich zu halten.% valider Grund

% Namensgebung aus ~/tests/readme.md wahrscheinlich geeigneter
\subsection{Simple Probleme}\phantomsection\label{problems:simple}
% 0d zu 1d
\subsubsection{Unerwartete Zeichen}\phantomsection\label{problems:dim0}
\paragraph*{Beispiel}
\verb|\label{problem:encounter:solve}| wird zu \verb|\label{Problem:Begegnung:Lösung}| übersetzt, aber \verb|\section{example}| zu \verb|\Abschnitt{Beispiel}|.
\paragraph*{Beschreibung}
\paragraph*{Abstrahierung}
Teile der \TeX{}-Syntax lassen sich anhand von \verb|\|, \verb|{|, \verb|}|, \verb|[|, \verb|]|, \verb|$|, \verb|$$| oder \verb|\%| erkennen und müssten daher ausgeschlossen werden. Man kann sich diese Art von Fehlern wie 0-dimensionale Fehler vorstellen, wobei die nullte Dimension hierbei bei einem einzelnen Wort beginnt (welche als Punkte zu verstehen sind).

% 0d zur ersten Dimension
\subsubsection{Fehlinterpretierte Wörter}\phantomsection\label{problems:dim1}% Meint: Wort wurde nicht als Syntaktisch relevant erkannt
\paragraph*{Beispiel}
\verb|\begin{myenvironment}[fontsize=red]| wird zu \verb|\begin{myenvironment}[fontsize=red]| übersetzt, aber \verb|\begin{myenvironment}[ fontsize = red ]| zu \verb|\begin{myenvironment}[ fontsize = rot ]|.
\paragraph*{Beschreibung}
\paragraph*{Abstrahierung}
Teile der \TeX-Syntax lassen sich nicht nur anhand der~\hyperref[problems:unexpectedCharacters]{zuvor} beschriebenen Zeichenketten erkennen, sondern lassen sich auch in Zeilen wiederfinden. Diese Art von Fehlern bahnt den Weg zu einer Dimension, wodurch nicht nur innerhalb eines Wortes (Punktes), sondern auch zwischen verschiedenen Punkten Fehler entstehen könnten (also innerhalb einer Zeile).


% Vertikales Spacing / Zweite Dimension
\subsubsection{Übersichtliche Quelltexte}\phantomsection\label{problems:dim2}
\paragraph*{Beispiel}
\begin{table}[h!]
    \centering
    \begin{tabularx}{\textwidth}{X X}
        \toprule
            Original & Übersetzung\\
        \midrule
        % Siehe ~/tests/readme.md für namensgebung und "Wo ist die Datei?"
            Beispiel 1\lstinputlisting[language=tex]{../tests/simple/2d/correct_original.tex} & \lstinputlisting[language=tex]{../tests/simple/2d/correct.tex}\\[2em]
            Beispiel 2\lstinputlisting[language=tex]{../tests/simple/2d/wrong_original.tex} & \lstinputlisting[language=tex]{../tests/simple/2d/wrong.tex}\\
        \bottomrule
    \end{tabularx}
    \caption{Beispiel für eine Zeile, welche übersetzt werden darf}\label{tab:problems:dim2}
\end{table}
\paragraph*{Beschreibung}
% Zwar ist es praktisch, wenn Quelltexte durch zusätzlichen Whitespace nicht nur auf horizontaler, sondern auch auf vertikaler Ebene übersichtlicher werden, allerdings führt dies bei einer zeilenweisen Übersetzung dazu, dass der Kontext aus voriger Zeile evtl. verloren geht.
\paragraph*{Abstrahierung}
Teile der \TeX-Syntax lassen sich nicht nur anhand von einzelnen Zeilen oder Zeichenketten erkennen, sondern könnten sich auch auch in verschiedenen Zeilen wiederfinden lassen. Diese Art von Fehlern kann 2-dimensional betrachtet werden, wodurch Fehler auch zwischen Zeilen entstehen können.

% Dreidimensionales Spacing
\subsubsection{Mehrere Dateien}\phantomsection\label{problems:dim3}
\paragraph*{Beispiel}
\begin{table}[h!]
    \centering
    \begin{tabularx}{\textwidth}{X X}
        \toprule
            Original & Übersetzung\\
        \midrule
        % Siehe ~/tests/readme.md für namensgebung und "Wo ist die Datei?"; hoffentlich sieht sich der Herr Prof. Dr. rer. nat. habil. nicht den Quellcode an dieser Stelle an.
            Datei $x$ \lstinputlisting[language=TeX]{../tests/simple/3d/x_original.tex} & \lstinputlisting[language=tex]{../tests/simple/3d/x.tex}\\[2em]
            Datei $y$ \lstinputlisting[language=TeX]{../tests/simple/3d/y_original.tex} & \lstinputlisting[language=tex]{../tests/simple/3d/y_original.tex}\\
        \bottomrule
    \end{tabularx}
    \caption{Beispiel für eine übersehende Datei}\label{tab:problems:dim3}
\end{table}
\paragraph*{Beschreibung}
Die Übersetzung von Datei $x$, welche Datei $y$ via \texttt{input} oder \texttt{include} % Unterschied: input darf vernested werden, include nicht. wir gehen von einem größeren Werk aus, welches aus x Kapiteln besteht, welche integriert werden sollen. die einzelnen Kapitel werden danach nur in Abschnitte zerlegt, welche durch inputs eingebunden werden sollen
führt zwar dazu, dass Datei $x$ übersetzt wird, aber Datei $y$ nicht.
\paragraph*{Abstrahierung}% Hier war abstrahieren leichter als beschreiben
Teile der \TeX-Syntax müssen nicht zwingend in einer Datei vorliegen, sondern könnten auch in verschiedenen Dateien integriert sein. Die Klassifizierung simpler Probleme gelangt in \TeX{} hier bereits in der dritten Dimension an, weswegen sich fortan bereits mit \enquote{fortgeschrittenen} Problemen beschäftigt werden muss (welche sich Teils über mehrere \enquote{Dimensionen} erstrecken).

\subsection{Fortgeschrittenere Probleme}\phantomsection\label{problems:advanced}
\subsubsection{Kommentare}
\paragraph*{Beispiele}
\paragraph*{Beschreibungen}
%- zunächst als Unterklasse von~\ref{problems:unexpectedCharacters} zu erwarten
%- kann jedoch auch~\ref{problems:verticalSpacing} umfassen
\paragraph*{Abstrahierung}
Hier treffen simple Fehler aus den ersten drei Kategorien (in~\ref{problems:dim0},~\ref{problems:dim1} und~\ref{problems:dim2} geschildert) aufeinander. In die dritte Dimension, also in andere Dateien, wird jedoch nicht traversiert, da auskommentierte Datei-Einbindungen nicht erfasst werden sollten. 


\subsubsection{Tabellen}\phantomsection\label{problems:tables}
\paragraph*{Beispiele}
\paragraph*{Beschreibungen}
\paragraph*{Abstrahierung}

\subsubsection{Literaturverzeichnis}\phantomsection\label{problems:bibtex}
\paragraph*{Beispiele}
\paragraph*{Beschreibungen}
\paragraph*{Abstrahierung}


\subsubsection{TikZ und Layouting}\phantomsection\label{problems:layouting}
\paragraph*{Beispiele}
\paragraph*{Beschreibungen}
\paragraph*{Abstrahierung}

\subsubsection{Category Codes}\phantomsection\label{problems:catcode}
\paragraph*{Beispiele}
\paragraph*{Beschreibungen}
\paragraph*{Abstrahierung}


\subsection{Spezielle Probleme}\phantomsection\label{problems:special}
\subsubsection{Glossare und Nomenklatur}\phantomsection\label{problems:glossaries}
\paragraph*{Beispiele}
\paragraph*{Beschreibungen}
\paragraph*{Abstrahierung}

\subsubsection{Dilemmatische Makros}\phantomsection\label{problems:macrodilemma}
\paragraph*{Beispiele}
\paragraph*{Beschreibungen}
\paragraph*{Abstrahierung}

\subsection{Weitere Schwierigkeiten}\phantomsection\label{problems:additional}
\paragraph*{Beispiele}
\paragraph*{Beschreibungen}
\paragraph*{Abstrahierung}
