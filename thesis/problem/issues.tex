%%%%%%%%%%%%% Versuche die deutsche wissenschaftliche Sprache verständlich zu halten:
%%%%%%%%%%%%% - Infinitiv so oft wie möglich (durch Substantivierung oder Erweiterung), damit Texte leichter zu verstehen sind.
%%%%%%%%%%%%% - Reale Konditionalsätze nutzen.

\section{Problemfälle}
Uneindeutigkeiten\label{probleme:uneindeutigkeiten} in der Sprache sind für einen Leser oft schwer nachzuvollziehen. In \TeX{} muss allerdings zu mindestens einem Zeitpunkt die Information über das Aussehen des entgültigen Dokumentes in einer modellartigen Form vorliegen. Diese Information kann in \LaTeX{} verborgen sein oder aber bei einem Übersetzen verloren gehen.
Die Probleme unterteilen sich in verschiedene Fälle und werden hinsichtlich des Kontextes dieses Informationsverlustes in translative (beim Übersetzen), technische (\LaTeX{}), spezifische technische (in Kombination mit \TeX{} nutzbare Programme) und sprachliche Probleme. Sprachliche Probleme verursachen teilweise dillematische Probleme,%~\hyperref[probleme:uneindeutigkeiten]{beschriebene Gründe} nicht immer lösbar und sind daher als \enquote{Schwierigkeit} formuliert.
welche als \enquote{Schwierigkeiten} und nicht als zu lösende Probleme dargestellt werden. Einzelne aufgeführte Beispiele zur Veranschaulichung beschriebener Probleme sind mit Hilfe von nicht spezifischer Software erzeugt (Google Translate), um zu zeigen, dass jeweilige Situation in \textit{einer} Software zur Übersetzung von menschensprachlichen Inhalten entstehen \textit{könnten}.



\subsection{Translativ}\phantomsection\label{problems:simple}

% Hiermit sagen wir: Punkte/Elemente der nullten Dimension sind veränderbar.
\subsubsection{Sonderzeichen}\phantomsection\label{problems:dim0}
% Klasse Problem, da Unterklasse mit Funktionen, die genutzt werden können, wenn man ein Problem hat. Die einzig benötigte ist problem::encounter::solve(problem::stack::pop()) // Wir lösen das oben auf dem Stapel an Problemen liegende Problem zuerst
\begin{table}[h!]
    \centering
    \begin{tabularx}{\textwidth}{X X}
        \toprule
            Original & Übersetzung\\
        \midrule
        % Siehe ~/tests/readme.md für namensgebung und "Wo ist die Datei?"
            Korrekt & \\[-13px] % relative Angabe ggb. echtem Zeilenumbruch (1em) // relativ in dem Sinne: wir gehen von der Position des erwarteten Zeilenumbruches aus und verschieben nach unten oder oben (+ bzw - in der Höhe)
            \commoncode{Test}{../examples/simple/0d/correct_original.tex} & \commoncode{Test}{../examples/simple/0d/correct.tex}\\[1em]
            Unerwünscht & \\[-13px]
            \commoncode{Test}{../examples/simple/0d/wrong_original.tex} & \commoncode{Test}{../examples/simple/0d/wrong.tex}\\[-1em]
        \bottomrule
    \end{tabularx}
    \caption{Fehler in einem Token}\phantomsection\label{tab:problems:dim0}
\end{table}


\paragraph*{Beschreibung und Begründung}
Der für einen \TeX{} Compiler relevante Befehl \verb|\label| bleibt unverändert, allerdings \texttt{section} wird fälschlicherweise als \texttt{Abschnitt} übersetzt.  
Warum \texttt{label} nicht erfasst werden sollte, wenn die folgenden drei Wörter übersetzt wurden, wirft Fragen auf. Zu sehen ist ein String, welcher menschliche Sprache mit Sonderzeichen vermischt. Da insbesondere Klammern in (vielen) sprachlichen Kontexten hilfreich sind, werden deren Inhalte selbst zunächst nach zusammenhängenden Worten durchsucht, welche ihrerseits durch \verb|:| getrennt sind, oder als Klammern betrachtet werden können. Ein Entfernen der Klammern lässt lässt in erstem Beispiel \verb|\label problem encounter solve| und in zweitem Beispiel \verb|\section example| stehen. Zu sehen ist hier also bereits, dass Google Translate bei einer, wenn man es so interpretieren möchte, Vernestung zweiten Grades scheitert, jedoch einfache Vernestungen noch erkennt\footnote{\textit{Vernestung}: Klammern in Klammern, wie in der Mathematik}.

\paragraph*{Takeaway}
Teile der \TeX{}-Syntax lassen sich anhand von \verb|\|, \verb|{|, \verb|}|, \verb|[|, \verb|]|, \verb|$|, \verb|$$| oder \verb|\%| erkennen und müssten daher ausgeschlossen werden. Anders als in mathematischen Formeln zeigen sich Sonderzeichen jedoch nicht paarweise auf, sodass sie nicht paarweise ignoriert werden können. Man kann sich diese Art von Fehlern wie 0-dimensionale Fehler vorstellen, wobei die nullte Dimension hierbei bei einem einzelnen Wort beginnt (welche als Punkte zu verstehen sind).




% Hiermit sagen wir: Kanten/Elemente der ersten Dimension sind veränderbar.
\subsubsection{Leerzeichen}\phantomsection\label{problems:dim1}% Meint: Wort wurde nicht als Syntaktisch relevant erkannt; Wort dürfte nicht übersetzt werden
\begin{table}[h!]
    \centering
    \begin{tabularx}{\textwidth}{X X}
        \toprule
            Original & Übersetzung\\
        \midrule
        % Siehe ~/tests/readme.md für namensgebung und "Wo ist die Datei?"
            Korrekt & \\[-13px]
            \commoncode{Test}{../examples/simple/1d/correct_original.tex} & \commoncode{Test}{../examples/simple/1d/correct.tex}\\[1em]
            Unerwünscht & \\[-13px]
            \commoncode{Test}{../examples/simple/1d/wrong_original.tex} & \commoncode{Test}{../examples/simple/1d/wrong.tex}\\[-1em]
        \bottomrule
    \end{tabularx}
    \caption{Fehler in einem einzeiligen Dokument}\phantomsection\label{tab:problems:dim1}
\end{table}
\paragraph*{Beschreibung und Begründung}% Warum das ein Problem ist
Die Optionen innerhalb eckiger Klammern lassen auch Whitespace zu. Dies kann jedoch für die Nutzung einiger Funktionen in z.B.\ wichtigen Paketen wie \texttt{hyperref} dazu führen, dass falsche Wörter übersetzt werden, die ein Kompilieren des Dokumentes verhindern.

\paragraph*{Abstrahierung}
Teile der \TeX-Syntax lassen sich nicht nur anhand der~\hyperref[problems:unexpectedCharacters]{zuvor} beschriebenen Zeichenketten erkennen, sondern lassen sich auch in Zeilen wiederfinden. Diese Art von Fehlern bahnt den Weg zu einer Dimension, wodurch nicht nur innerhalb eines Wortes (Punktes), sondern auch zwischen verschiedenen Punkten Fehler entstehen könnten (also innerhalb einer Zeile).










\newpage















% Hiermit sagen wir: Flächen/Elemente der zweiten Dimension sind veränderbar.
\subsubsection{Zeilenbrüche}\phantomsection\label{problems:dim2}
\begin{table}[h!]
    \centering
    \begin{tabularx}{\textwidth}{X X}
        \toprule
            Original & Übersetzung\\
        \midrule
        % Siehe ~/tests/readme.md für namensgebung und "Wo ist die Datei?"
            Korrekt & \\[-13px]
            \commoncode{Test}{../examples/simple/2d/correct_original.tex} & \commoncode{Test}{../examples/simple/2d/correct.tex}\\[1em]
            Unerwünscht & \\[-13px]
            \commoncode{Test}{../examples/simple/2d/wrong_original.tex} & \commoncode{Test}{../examples/simple/2d/wrong.tex}\\[-1em]
        \bottomrule
    \end{tabularx}
    \caption{Fehler in einem einzeiligen Dokument}\phantomsection\label{tab:problems:dim2}
\end{table}


\paragraph*{Beschreibung}
% Zwar ist es praktisch, wenn Quelltexte durch zusätzlichen Whitespace nicht nur auf horizontaler, sondern auch auf vertikaler Ebene übersichtlicher werden, allerdings führt dies bei einer zeilenweisen Übersetzung dazu, dass der Kontext aus voriger Zeile evtl. verloren geht.
\paragraph*{Abstrahierung}
Teile der \TeX-Syntax lassen sich nicht nur anhand von einzelnen Zeilen oder Zeichenketten erkennen, sondern könnten sich auch auch in verschiedenen Zeilen wiederfinden lassen. Diese Art von Fehlern kann 2-dimensional betrachtet werden, wodurch Fehler auch zwischen Zeilen entstehen können.

\newpage




% Hiermit sagen wir: Körper/Elemente der dritten Dimension sind veränderbar.
\subsubsection{Dokumentenbrüche}\phantomsection\label{problems:dim3}
\paragraph*{Beispiel}
\begin{table}[h!]
    \centering
    \begin{tabularx}{\textwidth}{X X}
        \toprule
            Original & Übersetzung\\
        \midrule
        % Siehe ~/tests/readme.md für namensgebung und "Wo ist die Datei?"
            Korrekt & \\[-13px]
            \commoncode{Test}{../examples/simple/3d/correct_original.tex} & \commoncode{Test}{../examples/simple/3d/correct.tex}\\[1em]
            Unerwünscht & \\[-13px]
            \commoncode{Test}{../examples/simple/3d/wrong_original.tex} & \commoncode{Test}{../examples/simple/3d/wrong.tex}\\[-1em]
        \bottomrule
    \end{tabularx}
    \caption{Fehler in einem einzeiligen Dokument}\phantomsection\label{tab:problems:dim2}
\end{table}
\paragraph*{Beschreibung}
Die Übersetzung von Datei $x$, welche Datei $y$ via \texttt{input} oder \texttt{include} % Unterschied: input darf vernested werden, include nicht. wir gehen von einem größeren Werk aus, welches aus x Kapiteln besteht, welche integriert werden sollen. die einzelnen Kapitel werden danach nur in Abschnitte zerlegt, welche durch inputs eingebunden werden sollen
führt zwar dazu, dass Datei $x$ übersetzt wird, aber Datei $y$ nicht.
\paragraph*{Abstrahierung}% Hier war abstrahieren leichter als beschreiben
Teile der \TeX-Syntax müssen nicht zwingend in einer Datei vorliegen, sondern könnten auch in verschiedenen Dateien integriert sein. Die Klassifizierung simpler Probleme gelangt in \TeX{} hier bereits in der dritten Dimension an, weswegen sich fortan bereits mit \enquote{fortgeschrittenen} Problemen beschäftigt werden muss (welche sich Teils über mehrere \enquote{Dimensionen} erstrecken). Eine vierte Dimension existiert physikalisch nicht, ist jedoch mathematisch formulierbar\footnote{physikalisch: die vierte Dimension ist die Zeit, wenn man eine nicht-euklidische 3-dimensionale Bewegung verlangt (Teleportation)} und äußert sich in diesem Kontext auf eine Erhöhung von Laufzeitkomplexitäten.





% Hier kommt die Zeit dazu.
\subsection{Technisch}\phantomsection\label{problems:advanced}
%%% Hier ein wenig weg von der eigentlichen Struktur, jedoch für die konkreten Beispiele wieder gleich
\subsubsection{Besonderheiten dieses Abschnittes}
Vorige Kapitel bahnte sich nach und nach von zunächst Wörtern (Punkten) zu mehereren Wörtern (Kanten) zu Sätzen (Graphen) und final zu größeren Gebilden (Körper, um das Wort \enquote{Hypergraph} zu vermeiden). Hier hören räumliche Verständnisse der Welt auf, wodurch folgende Probleme durch zeitliche Abhängigkeiten verschiedener Ereignisse aufgezeigt werden. Auf dieser Ebene ist es sehr praktisch, dass nur wenige Situationen eintreten können. Entweder erfolgt zu Beginn eines Dokumentes eine Änderung, welche Einfluss auf einen späteren Teil hat (vorgreifend); an einem Punkt im Dokument eine Änderung, welche Einfluss auf diese Stelle im Dokument hat (akut); oder eine Änderung im Dokument, welche Einfluss auf einen bereits existierenden Teil des Dokumentes hat (rückwirkend).% Dieser Abschnitt muss überarbeitet werden.

\subsubsection*{Arten von Änderungen innerhalb eines Dokumentes und Auswirkungen dieser auf das Dokument}% Titel muss überarbeitet werden
Der verwirrenden Formulierung vorigen Abschnittes etwas Entzwirrung zu leisten, eine kurze Schilderung des Gemeinten. Das Hauptproblem besteht darin, dass Befehle, welche Strukturen in einem Dokument verändern, die ihrerseits vor dem Befehl im Quelltext erscheinen, dazu führen, dass diese Strukturen vergrößert werden. \TeX{} selbst löst dies über verschiedenste Hilfsdateien, als auch erneute Kompilierprozesse. Geht man allerdings davon aus, dass übersetzende Programme nur einfach (im Sinne: ein Mal) ablaufen sollen, so würden etwaige Anpassungen in Hilfsdateien übersehen werden (die mitunter Strings enthalten, welche kontextuelle oder sprachliche Inhalte tragen). 

\paragraph*{Rückwirkende Änderungen}
Bei jedem Hinzufügen eines z.B.\ neuen Kapitels, einer neuen Tabelle, einer neuen Abbildung (und dergleichen) kann und sollten jeweilige Verzeichnisse bearbeitet worden sein. Jedoch ist aus logischer Perspektive eine Änderung eines bereits existierenden Teils eines Dokumentes nicht mehr möglich, wenn dieser Teil des Dokumentes bereits geschrieben wurde. Deshalb muss das Dokument erneut und von vorne bearbeitet werden, damit diese Änderung vorgenommen werden kann. Die Frage die sich stellt: Soll gewartet werden, bis alle rückwirkenden Änderungen erfasst wurden oder soll jede rückwirkende Änderung ihrerseits sofort vorgenommen werden? Eine Antwort findet sich darin, ob eine rückwirkende Änderung ihrerseits solche produzieren kann. Falls sie dazu in der Lage wäre, würde dies unendliche Schleifen bilden können, insofern entstehende rückwirkende Änderungen eine Auswirkung auf Teile des Dokumentes vor dieser rückwirkenden Änderung haben könnten.


\paragraph*{Akute Änderungen}
\subparagraph*{Beschreibung}
Können bedenklich werden.


\paragraph*{Vorgreifende Änderungen}
\subparagraph*{Beschreibung}
Sollte die Situation entstehen, dass ein Befehl in einem Dokument eine Veränderung in einem späteren Teil des Dokumentes verursacht, ist dies zunächst unbedenklich.

%%%%%%%%%%%%%%%%%%%%%%%%%
%%%%%%%%%%%%%%%%%%%%%%%% In Review. Richtige Inhalte, inadäquate Sortierung.
%%%%%%%%%%%%%%%%%%%%%%


\subsubsection{Hilfsdateien}% multi-file handling

\paragraph{Struktur dieses Abschnittes}
Eine Datei nutzt ein Literaturverzeichnis (Bib\TeX{}), pdf\TeX{} (\verb|\pdfcomment{}|), \verb|\footnote|, ein Inhaltsverzeichnis, ein Glossar, 

\paragraph{Beispielliste}
\subparagraph*{Inhaltsverzeichnisse, Abbildungsverzeichnisse, Tabellenlisten}
\paragraph*{Beispiel}
\paragraph*{Erläuterung}
Ändern wir den Titel eines Paragraphen oder Abschnittes, dann erfasst dies der \TeX{} Compiler beim ersten Durchlauf. Jedoch der String im Inhaltsverzeichnis kann nur verändert werden, sobald diese Information zu Beginn des nächsten Kompilierungsprozesses in der entsprechenden Hilfsdatei vorliegt (\texttt{.toc}).

\subparagraph{Backrefs}
\paragraph*{Beispiel}
Bedarf evtl.\ einer bildlichen Veranschaulichung. Meint: Inhaltsverzeichnis, Tabelle,\ldots vor einer \textit{backwards reference} verschiebt die echte Position der (Phantom-) Sektion. Daher muss zunächst bestimmt werden, wo im Dokument eine Referenz auf ein existierendes Label stattfindet, welches vorherig bereits vergeben wurde\pdfcomment{Ein Erneuern eines Labels via renewcommand ergibt keinen Sinn, da dieses forward references verfälschen würde.}.
\paragraph*{Erläuterung}
Ein Verweis auf einen vorherigen Paragraphen kann nur klickbar verlinkt werden, wenn die Information, an welcher Stelle er sich im Dokument befindet, bereits klar ist. Da nach der Referenz weitere Abschnitte folgen können, welche vorherige Elemente mit variabler Größe verändern könnten, muss zunächst die Größe dieser bestimmt sein und gegenüber dieser kann dann der 

\subparagraph*{Literaturverzeichnisse}
\paragraph*{Beispiel}
\paragraph*{Erläuterung}

\subparagraph*{PDF Funktionen}
\paragraph*{Beispiel}
\paragraph*{Erläuterung}

\subparagraph*{\enquote{footnotes}}
\paragraph*{Beispiel}
\paragraph*{Erläuterung}






\paragraph{Abstrahiertes Problem}
Falls auf Hilfsdateien zugegriffen wird, ist das mehrfache Kompilieren eines \LaTeX{} Dokumentes unvermeidbar. Gegeben $n_{h}$ womoglich existierenden Hilfsdateien, welche alle ihrerseits zu übersetzende Inhalte beinhalten können, folgt eine minimale Laufzeit $\mathcal{O}(n_h)$. 
%Ein einmaliges Kompilieren eines \LaTeX{} Dokumentes ist unmöglich, falls ein Zugriff auf Hilfsdateien nötig ist.


%%%%%%%%%% Für später relevant, für Makros und beliebige Nutzereingaben etc......:!:!:!!::!!::!
%Die Möglichkeiten andere Dateien in einem \LaTeX{} Dokument einzubeziehen, könnten nähern sich einer Unendlichkeit.% Warum? ist schwer zu beantworten, daher sollte man sich zunächst vor Augen führen, für welche Zwecke man auf andere Dokumente innerhalb eines Dokumentes verweisen möchte. // Warum eine Unendlichkeit? Es gibt unendlich viele Unendlichkeiten. Beweis bitte von Prof. Cap
%Ein Beschäftigen mit dieser führt zu nichts\footnote{invers},%, sodass zunächst die Frage gestellt werden müsste, wie diese entstehen können, indem alle denkbaren Arten, wie sich zwei oder mehrere Dateien gegenseitig referenzieren/benötigen könnten, aufgelistet wird. 
%weshalb ein Betrachten möglicher versteckter interner Abhängigkeiten zwischen verschiedenen \LaTeX{} Dokumenten unabdingbar ist.% [...], hence describing the possible internal dependencies of \LaTeX{} documents is necessary. // um von der Formulierung "weshalb mögliche interne Abhängigkeiten zwischen verschiedenen Dateien erläutert werden müssen" wegzukommen, sah ich mich gezwungen in die englische Sprache auszuweichen (da es mir dabei half zu erkennen, wie ich einen direkten Konjunktiv II vermeiden kann).
%%% Inwiefern wirkt sich das auf die gegebene Problematik aus
%Nur einen Teil eines Dokumentes zu übersehen, provoziert kontextuellen Verlust für Übersetzungen (Abschnitt~\ref{problems:dim3}). 
%\begin{enumerate}
%%    \item Permutationen (insgesamt 4, da wir nur zwischen $1$ und $n \textit{(beliebig vielen)}$ zu wechseln wünschen (2*2 Optionen / Urnen=1;Kugeln=2;Farben=2;zurücklegen)): 
%    \item Ein Dokument kann ein Anderes in sich tragen. 
%    \item Ein Dokument kann n Andere in sich tragen
%    \item n Dokumente können ein Anderes in sich tragen
%    \item n Dokumente können n Andere in sich tragen geht aus den beiden vorigen statements hervor und ist redundant (weil wir kennen bereits 1 dokument, welches n Andere tragen kann).
%\end{enumerate}
% Realität es gibt nicht unendlich viele Dokumente, da physikalisch nicht möglich (Archimedis). 
%Hier stoßen wir 


%%% Meint: Hier müssen wir die eigentlichen Literaturverweise kennen, um einen Kontext zu kennen.
\subsubsection{Unerreichbare Informationen}

\paragraph*{Beispiele}
\begin{table}[h!]
    \centering
    \begin{tabularx}{\textwidth}{X X}
        \toprule
            Original & Übersetzung\\
        \midrule
        % Siehe ~/tests/readme.md für namensgebung und "Wo ist die Datei?"; hoffentlich sieht sich der Herr Prof. Dr. rer. nat. habil. nicht den Quellcode an dieser Stelle an.
            Dokument & \\
            \lstinputlisting[language=TeX]{../examples/advanced/literature/example_original.tex} & \lstinputlisting[language=tex]{../examples/advanced/literature/example.tex}\\[2em]
            Bibliothek & \\
            \lstinputlisting[language=tex]{../examples/advanced/literature/example_original.bib} & \lstinputlisting[language=tex]{../examples/advanced/literature/example.bib}\\
            %%% Bemerkung: Übersetzung noch nicht erstellt.
        \bottomrule
    \end{tabularx}
    \caption{Beispiel für einen verpassten literarischen Kontext}\label{tab:problems:nonexisting}% mit reference 
\end{table}

\paragraph*{Beschreibungen}
Ein Dokument erwähnt ein Werk, in welchem es um die C-Programmierung geht. Rein aus den im System vorliegenden Dateien ist kein Kontext für das Wort \enquote{String} erkennbar, sodass ein Zugriff auf eine externe Ressource unabdingbar ist.

% Bemerkung: Nur die Suche (google.com) nach "salomon c programmierung" führt bspw. Gemini zu einer vermuteten Verwechslung mit dem Begriff "System" (Stand: 09.10.2025, 12:29).
% ISBN führt zur gleichen Minute direkt zum Institut (Angewandte Mikroelektronik und Datentechnik)... wobei Thalia denkt, dass ich "1984" online kaufen möchte...
\paragraph*{Abstrahierung}
Einfache Cloud-Architektur. Ein Client möchte auf ein beliebiges Wissen einer Webseite (bzw.\ dem Server und den beanspruchten Speicherplätzen in einem (beliebigen) Rechenzentrum\footnote{Hierbei ist nicht von Festspeicher zu reden. Aus Sicherheitsgründen sei davon auszugehen, dass sich die physischen Adressen des wissensrepräsentierenden Speichers regelmäßig und unvorhersehbar ändern} zugreifen).




\subsubsection{Figuren und Tabellen}\phantomsection\label{problems:advanced:tables}
\paragraph*{Beispiele}
\paragraph*{Beschreibungen}
\paragraph*{Abstrahierung}

\subsubsection{Literaturverzeichnisse}\phantomsection\label{problems:advanced:bibtex}
\paragraph*{Beispiele}
\paragraph*{Beschreibungen}
Bib\TeX{} erlaubt es an vielerlei Stelle eigene Strings in einer kompilierten \TeX{}-Datei zu verbergen.
\paragraph*{Abstrahierung}


\subsubsection{Category Codes}\phantomsection\label{problems:advanced:catcode}
\paragraph*{Beispiele}
\paragraph*{Beschreibungen}
\paragraph*{Abstrahierung}


%%%%%%%%%%%%%%%%%%%%%%%%%
%%%%%%%%%%%%%%%%%%%%%%%% In Review. Richtige Inhalte, inadäquate Sortierung.
%%%%%%%%%%%%%%%%%%%%%%




\subsection{Spezifischer Technologien}\phantomsection\label{problems:special}
Mitunter NP-schwer.
\subsubsection{Kommentare}\phantomsection\label{problems:advanced:comments}
\paragraph*{Beispiele}
\paragraph*{Beschreibungen}
%- zunächst als Unterklasse von~\ref{problems:unexpectedCharacters} zu erwarten
%- kann jedoch auch~\ref{problems:verticalSpacing} umfassen
Wohingegen sich~\ref{problems:advanced:comments} nicht mit anderen, in Kommentaren referenzierten, Dateien beschäftigt, soll sich hier auf solche Fälle konzentriert werde.
\paragraph*{Abstrahierung}
Hier treffen simple Fehler aus den ersten drei Kategorien (in~\ref{problems:dim0},~\ref{problems:dim1} und~\ref{problems:dim2} geschildert) aufeinander. In die dritte Dimension, also in andere Dateien, wird jedoch (\hyperref[problems:special:comments]{vorerst}) nicht traversiert, da auskommentierte Datei-Einbindungen nicht erfasst werden dürften. 
Ausgehend von~\ref{problems:advanced:comment} wird nun erwartet, dass eine Referenzierung von Dateien erwartet wird, welche sich in Kommentaren verbergen. Dies kann jedoch~\ref{problems:special:sourcecode} beinhalten.


\subsubsection{Dilemmatische Makros}\phantomsection\label{problems:special:macrodilemma}% dilemmatasitische Makros... Kombi aus dilemmatisch und fantastisch?
\paragraph*{Beispiele}
\paragraph*{Beschreibungen}
\paragraph*{Abstrahierung}

\subsubsection{TikZ und Layouting}\phantomsection\label{problems:advanced:layouting}
\paragraph*{Beispiele}
\paragraph*{Beschreibungen}
\paragraph*{Abstrahierung}

\subsubsection{Quellmehrsprachigkeit}\phantomsection\label{problems:special:sourcecode}
\paragraph*{Beispiele}
\paragraph*{Beschreibungen}
\paragraph*{Abstrahierung}
Quelltexte anderer Quellsprachen (Programmiersprachen) können ihrerseits auf andere Dateien verweisen, oder andere Syntaktik tragen. Das Erkennen dieser ist theoretisch gesehen leicht, jedoch praktisch gesehen schnellig zu übersehen. 









\subsection{Sprachliche Schwierigkeiten}\phantomsection\label{problems:additional}
\subsubsection{Glossare und Nomenklaturen}
\paragraph*{Beispiele}
\paragraph*{Beschreibungen}
\paragraph*{Abstrahierung}

\subsubsection{Weitere}
\paragraph*{Beispiele}
\paragraph*{Beschreibungen}
\paragraph*{Abstrahierung}
