% Welche Fehler können entstehen?
\section{Problemfälle}
Mittels \TeX{} ist prinzipiell alles möglich (wie sich später zeigen wird), jedoch sollte man zunächst denkbare Schwierigkeiten für jegliche Sprachübersetzungen von \TeX{} und \LaTeX{} Dokumenten nicht nur dahingegen schildern, dass sie auftreten könnten, sondern auch dahingegen klassifizieren, inwiefern sie häufig zu erwarten sind, geschweige denn sinnvoll oder gar unsinnig sein könnten (da sie z.B.\ eine zukünftige Bearbeitung eines Dokumentes erschweren könnten). Beruft man sich zunächst nur auf die reine \LaTeX{}-Syntax, werden vorerst wahrscheinlich nur simplere Probleme erkennbar, jedoch führt eine Näherung an die \TeX{}-Primitiven eine Vielzahl von komplexeren und speziellen Problemen mit sich. 

% Erwartung: Wer das nicht rafft ist doof.
\subsection{Simple Probleme}
% Befehle innerhalb der TeX-Engine (nativ) dürfen nicht übersetzt werden.
\paragraph{Zeichenketten}\label{para:zeichenketten} sind eine übliche Art und Weise, mit welcher man Wörter einer Sprache darstellen kann. Jedoch gehen die meisten Übersetzungs-Tools nicht davon aus, dass solche Zeichenketten Zeichen beinhalten, welche das folgende Wort zu einem Befehl (für eine Programmiersprache) machen. Zwar würde z.B.\ Google Translate für die Zeichenkette \texttt{Hello} korrekterweise das Deutsche \texttt{Hallo} liefern, aber bereits die Präambel von \TeX{}-Dokumenten zeigt, wie \verb|\title|, \verb|\author| und \verb|\date| respektiv zu \verb|\Titel|, \verb|\Autorin| und \verb|\Datum| übersetzt werden würden (Stand: 01.10.2025). Benanntes Tool zeigt sich zudem inkonsistent. Beispielsweise wird \verb|\section{saw}| zu \verb|\Abschnitt{Säge}| übersetzt und \verb|\section{Introduction}| zu \verb|\section{Einführung}| übersetzt.

% Von Wörtern zu mehreren Wörtern
\paragraph{Whitespace} sind eine herkömmliche Art verschiedene Wörter einer Sprache voneinander zu trennen (bspw.\ in den lateinischen oder kyrillischen Sprachen). Neben anfänglichen Schwierigkeiten, welche sich innerhalb von einzelnen Zeichenketten aufzeigen könnten, ist es genauso denkbar, dass einzelne Optionen in \TeX{} oder \LaTeX{} innerhalb von eckigen oder geschwungenen Klammern nicht übersetzt werden dürften, ohne die Syntax zu brechen oder übersetzt werden müssten, damit ein gesamtes Dokument übersetzt wird. Denkbar sind hier direkt für das Erstere Farbdefinitionen, wie zum Beispiel \verb|\definecolor{super light red}{rgb}{1,.5,.5}|. Sollte man versuchen~\ref{para:zeichenketten} dadurch zu lösen, dass man einfach die Präambel in der Übersetzung ausschließt (also alles vor \verb|\begin{document}|), so würde ein späteres Nutzen dieser Farbe \texttt{super light red} dafür sorgen, dass das Wort \textit{light} alleine steht, und somit für nicht weit durchdachte Ansätze als ein zu übersetzendes Wort gelten würde. 

\paragraph{Mehrere Dateien} sind eine denkbare Art größere \TeX{}-Dokumente in übersichtlichere kleinere Dateien zu strukturieren. Neben der Möglichkeit \TeX{}-Dokumente selbst via \texttt{include} und \texttt{input} in ein übergreifendes Dokument einzufügen, ist es jedoch auch möglich verschiedene andere, bildliche Formate (bspw.\ PNG, PDF, \ldots) im Dokument zu integrieren. Insbesondere bei PDF kann es sehr interessant werden, ob und inwieweit textliche Inhalte erfasst und übersetzt werden, jedoch sind PDF ihrerseits wieder eine von \LaTeX{} und \TeX{} abweichende Datei-/Dokumentenform und daher nicht weiter kritisch.

\subsection{Komplexere Probleme}
\paragraph{Definitionen}
\paragraph{Makros}
\paragraph{Umgebungen}
\paragraph{Pakete}
\paragraph{Quelltexte}

\subsection{Spezielle Probleme}
\paragraph{Höhere Vernestungsgrade}
\paragraph{Catcode}
\paragraph{Abstrakte im Literaturverzeichnis}

\subsection{Weitere Schwierigkeiten}
% Siehe uni/thesisbsc/tests/prblems/list.md/##9