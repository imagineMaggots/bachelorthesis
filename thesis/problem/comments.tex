%%%%% Kurzer Reminder für die wichtigsten Tücken (aber auch 'benefits' (EN für: Vorteil)) der deutschen Sprache für nativ dieser Sprache Mächtigen, welche nicht in wissenschaftliche Arbeiten fließen sollten. 
%%%%% 
%%%%% Meist recht offensichtlich.
%%%%%
%%%%% Konjunktiv 2: Impliziert meist eine Form von "Subjekt hätte etwas tun müssen" oder "Subjekt wird etwas tun" (oder schlimmer noch passiv: "mit Subject wird etwas geschehen", hier braucht es sogar das Partizip 2). Intuitiv wird klar, dass etwas nicht stattfand. Eine Quelle von möglicher Redundanz. Modalverben stehen oft in Verbindung mit solchen Implikationen ("dürfte ich etwas", "müsste ich etwas", "könnte ich etwas"... meist: ja, theoretisch gesehen "kann" man alles. Daher nicht schreiben: "man müsste/könnte/sollte etwas [so und so] machen", sondern: "A existiert, wodurch sich B ergibt). Siehe: Konditionalsätze
%%%%% Konzessivsätze: Leicht am Konjunktiv 2 zu erkennen. "Entgegen Behauptung X, ist die Realität Y". Die Realität gehört an den Anfang und aus dieser sollte genug Information hervorgehen, dass Behauptung X nicht ausformuliert werden muss.
%%%%% 
%%%%% Implikationen: Erklärung bedürfte eigentlich einem (mir noch unbekanntem) wissenschaftlichen Teilgebiet der "Aussagentheorie". Entspringt der Modularität deutscher Substantive ("Rindfleischetikettierungsüberwachungsaufgabenübertragungsgesetz" als Beispiel. Wir beschäftigen uns mit der Rindfleischetikettierung, welche überwacht werden soll. Das "s" nach "Rindfleischetikettierung" entstammt einem implizierten Genitiv (Überwachung der Rindfleischetikettierung = Rindfleischetikettierungsüberwachung), genauso wie das "s" nach "[...]überwachung[...]" (Aufgabenübertragung (bei) der Überwachung). Letztendliches "s" vor "[...]gesetz" ist am einfachsten, da sich Gesetze immer auf etwas beziehen. "Ein Gesetz einer Sache" kann zu "Gesetz der Sache" vereinfacht werden, aus welcher der Genitiv wieder hervorgeht).  

%%%%% Konditionalsätze: 
%%%%% "Aus These X, folgt Hypothese". "Daher folgt die nächste Hypothese". (Hypothese = These, welche auf einer anderen These aufbaut. Grundlage der Mathematik)
%%%%% Formal entweder eine Bedingung in einem Nebensatz (if-else statement) oder Schlussfolgerung in einem Hauptsatz (Folgesatz, vergleichbar: Zuweisungen). 
%%%%% - Arten: Real und Irreal = Erfüllbare und nicht erfüllbare Bedingungen. 
%%%%% - meint: Solange ich atme, bin ich. (Wenn ich für immer aufhören würde zu atmen, wäre erste Bedingung nicht mehr erfüllt. Ich bin ein Mensch, daher stimmt die Aussage.)
%%%%% - meint: Gäbe es keine Luft, wäre ich nicht. (Annahme: Die Luft bleibt uns vorerst noch einatembar erhalten)
%%%%% - meint: Hätte es nie Luft gegeben, hätte es mich nie geben können. (Wir schließen den Kreis zu Konjunktiv 2 und bemerken das Plusquamperfekt).

%%%%% Plusquamperfekt: Als ich (etwas) tat, war (etwas anderes) bereits passiert. Oder Konditionalsatz mit Konjunktiv 2: Wäre (etwas) nie geschehen, hätte (Resultat) nie erscheinen können. Passiert umgangssprachlich oft "hätte ich (das und das [besser]) gemacht, hätte ich (das und das) bereits erreicht" bspw.: "Hätte ich keinen zweiten Versuch für eine Abschlussarbeit benötigt, hätte ich bereits den Studiengang beenden können. (Hier bemerkt man direkt: "beenden können" kann zu "beendet" gekürzt werden. Partizip II ist meist umgangssprachlich, da textlich nicht von Nöten).


%%%%%%%%%%%%% Versuche die deutsche wissenschaftliche Sprache verständlich zu halten:
%%%%%%%%%%%%% - Infinitiv so oft wie möglich (durch Substantivierung oder Erweiterung), damit Texte leichter zu verstehen sind. // Formen von "ist" (abgelitten aus: "esse"), also eine Angabe von Existenz ist immer zu erwarten.
%%%%%%%%%%%%% - Reale Konditionalsätze nutzen. Wir präsentieren Fakten.

% Welche Arten von Fehlern können entstehen? (Was fällt einem auf, wenn man Dokumente schreibt und sich denkt: mmmmhh würde google translate das richtig übersetzen)


% Inhalt: Einige in issues.tex störende Kommentare

\section{Problemfälle}
%Eine einzige, feste \TeX{}-Syntax existiert theoretisch gesehen nicht, wie ein~\hyperref[problems:advanced:catcode]{späterer Paragraph} aufzeigen wird.% So gehen in-document refs ganz gut.
%Die Fähigkeit jegliche erdenkliche Zeichenkette (gegeben:\ diese ist auf einem Rechner darstellbar, siehe:~\cite{unicode}) sorgt zunächst für eine unendliche Menge an testbaren Problemen. Da es unmöglich ist eine unendliche Menge an Testfällen abzudecken, wird zunächst nur die vorgesehene \LaTeX{} (bzw.\ \TeX{}~Syntax nach~\cite{texbook}) betrachtet und die bereits rein innerhalb dieser schnellig auffallenden Fehler aufgezeigt, welche durch fälschlich übersetzte Zeichenketten entstehen könnten.\\\noindent
% reviewed: 1
% 
%\subsection{Struktur}\phantomsection\label{problems:structure}% Warum Kategoriesieren?
%\subsubsection{Klassifizierung}
%Für spätere Testzwecke wurden die verschiedenen Problemfälle in einzelne Kategorien getrennt. Eine solche Kategorisierung ist technisch gesehen nicht zwingend, soll allerdings zur Verbesserung einer späteren Übersicht dienen. Die Einteilung konzentriert sich vorrangig auf die Komplexität des Problemes und in einer Reihenfolge, in welcher sie auch bei der Nutzung von \TeX{} für ein beliebiges Dokument auftreten könnten. 
%
% Definition: Direkte Probleme
%Hierbei seien \textit{direkte Probleme} zunächst eher simple und technisch leicht zu behebende Probleme, welche sich durch ein einheitliches Vorgehen beheben lassen könnten.% Könnten muss an dieser Stelle so folgen
%Zudem beschränkt man sich hier nur auf den benötigten Zugriff auf eine einzelne Datei.
%
% Definition: Indirekte Probleme und Zusammenfassung zu "simplen Problemen"
%\textit{Indirekte Probleme} formen potentielle Schwierigkeiten, da sie einen Zugriff auf weitere Dateien benötigen, da sie von einer Ausgangsdatei referenziert werden. Technisch gesehen sind sie jedoch auch durch einheitliche Paradigma zu bewältigen. Daher werden sie im Kapitel~\ref{problems:simple} zusammengefasst und fortan als \enquote{simple Probleme} bezeichnet.
%
% Definition: Fortgeschrittene Probleme
%\enquote{Fortgeschrittene Probleme} beziehen sich auf Probleme, welche sich paradigmatisch lösen lassen, % man kann einen Quellcode schreiben; paradigmatisch = Mustern folgend
% aber zusätzliches Vorwissen verlangen. Dieses Vorwissen lässt sich bei diesen Problemfällen jedoch ermitteln, da der Entstehungsweg dieses Wissens nachvollziehbar ist.% Meint: Wir gucken uns an, wo und wann zu übersetzende Wörter hinter Makros, in Paketen, anderen >>Programmen<< versteckt wird.
%\enquote{Speziellere Probleme} umfassen Probleme, welche sich nicht ohne Vorkenntnis des tatsächlichen Dokumentes beheben lassen. Nach diesen wird zudem noch auf ein paar rein sprachliche Schwierigkeiten eingegangen, welche unter Anderem zu solchen speziellen Problemen führen könnten.

%\subsubsection{Beschreibung}
%Jeder Problemfall wird zunächst durch ein Beispiel demonstriert\footnote{welche aus einem Test mit Hilfe von Google Translate in Firefox durchgeführt wurden (Oktober 2025)}, danach wörtlich in einer Beschreibung erläutert und wird abschließend abstrahiert und versucht auf konkrete \TeX{} Primitiven zurückgeführt zu werden. Die fortgeschritteneren Probleme bedürfen teilweise mehreren Beispielen zur Beschreibung, lassen sich jedoch auf ähnliche Äußerungen zurückführen. Einführende Beispiele werden in der Form \texttt{ausführbarer Code} wird zu \texttt{ausführbarer Code} übersetzt, aber \texttt{ausführbarer Code} zu \texttt{fehlerhafter Code}\footnote{\enquote{Code} bezieht sich in beiden Fällen auf \TeX{}-Quelltexte} (1-dimensional). Einige Beispiele benötigen mitunter eine 2-dimensionale Darstellung, da sie mehrere Zeilen Quellcode umfassen und werden tabellarisch nebeneinander gestellt~\pdfcomment[color=red, icon=Paragraph]{Evtl.\ werden alle Beispiele in tabellarische Form überführt, welche ich persönlich anschaulicher finde, jedoch mehr Platz rauben würde}. Einige Probleme auf höherer Abstraktionsebene lassen sich allerdings nur schwierig veranschaulichen, wodurch Beispiele sich teils wieder auf Schilderungen von Situationen berufen, um diese Übersicht übersichtlich zu halten.% valider Grund