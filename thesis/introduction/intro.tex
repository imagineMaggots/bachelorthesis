% Titel -> Problemstellung
\section{Einleitung}
% Warum sollte man Sprachübersetzung automatisieren?
\subsection{Maschinelle Sprachübersetzung}
Die Mehrsprachigkeit auf unserem Planeten zeigt insbesondere für die Menschheit einige Hürden auf, welche man bereits seit den 1950er Jahren maschinell zu überwinden versucht (\cite{rockefeller:warrenWeaver:translation1952}, welcher zudem bemerkt, dass sich einige Gemeinsamkeiten zwischen menschlichen Sprachen aufzeigen, wenn auch nicht präzise formulierbar, jedoch von statistischem Mehrwert).\cite{rockefeller:warrenWeaver:translation1952} stellt außerdem ein sprachenunabhängiges Verständnis von einigen in der Natur auftretenden Dingen heraus, unabhängig von der (damaligen, als auch heutigen) Herkunft der Menschen und vermerkt zudem Korrelationen (hier: Ähnlichkeiten, nicht: Korrelation in der Statistik) in der wörtlichen graphischen Beschreibung von Szenarien und Handlungen, welche sich eigentlich nicht in ein Wort zusammenfassen lassen (als Beispiel führt er hier die ähnlichen Bedeutungen von \enquote{den chinesischen Wörtern für \textit{to shoot} und \textit{to dismiss}} an, wobei in beide Wörter eine ähnliche Bedeutung, wie \textit{to fire} interpretiert werden kann. Die deutsche Sprache findet jedoch ein direktes Wort für diese Bedeutung: \textit{(jemanden) zu kündigen}, wobei auch \textit{(jemanden) zu feuern} dasselbe aussagen würde). Ihm kam bereits 1947 der Gedanke, dass sich der Computer für die Übersetzung von menschlicher Sprache eignen könnte, sollte man die semantischen Zusammenhänge und daraus entstehenden Probleme der jeweiligen Sprachen mathematisch darstellen und lösen können.\cite{associationForComputingMachinery:adamLopez:statisticalMachineTranslation2008} betont, wie sich das Interesse an eine maschinellen Sprachübersetzung durch zuvor erwähnten Warren Weaver zurückführen lässt und demnach eine Disziplin ist, welche beinahe so alt ist, wie der elektronische Computer selbst.

% Gibt es im Alltag bekannte Technologien?
\subsection{Heutige Übersetzungswerkzeuge}
Auch heutzutage ist ein Drang nach schnellstmöglicher Übersetzung aus dem Alltag kaum wegzudenken. Nicht weit braucht man sich bewegen und selbst die Augen können (fairerweise: fast) geschlossen bleiben und schon sieht man auf vielerlei Webseiten, welche man besucht entweder von der Website selbst vorgeschlagene Übersetzungen in verschiedenste Sprachen (wie z.B. bei Wikipedia, Amazon, \ldots) oder seit gewisser Zeit die mittlerweile in Browsern (z.B. FireFox und Chrome) integrierte Option Seiteninhalte maschinell, durch kunstliche Intelligenz, übersetzen zu lassen.

% Welche digitalen Formen von Dokumenten gibt es?
\subsection{Digitale Dokumente}
Nun sind \enquote{schicke} Webseiten nicht der einzige Kontext, in welchem textliche Inhalte auf einem Rechner auftreten und bei welchem ein Interesse bestehen könnte, jene zu übersetzen. Betrachten wir zunächst ein wenig abstrakter jegliche Formen, in welcher unser Bildschirm einen Text darstellen kann, als Dokumente. Dokumente umfassen in Praxis dann verschiedene Dateitypen, beginnend bei einfachen Textdateien, binäre änderliche Formen (denkbar:~.docx,~.pptx,\ldots), binäre unveränderliche Formen (ISO32000$-$2), jedoch auch Applikationen, welche benötigt werden diese anzuzeigen (bspw.\ Adobe Acrobat, PDF24, FireFox,\ldots). Diese Dokumente zu erstellen kann insbesondere für eine dem \texttt{portable document format} folgende PDF schwierig sein, sollte man sich nur auf die Spezifikation alleine berufen. Einfacher wird es jedoch, sollte man Auszeichnungssprachen, wie z.B.\ HTML nutzen um Dokumente in der Form zu erzeugen, was für Webseiten geeignet ist, welche keiner festen Form unterliegen müssen, aber bei Dokumenten, die einer vorgegebenen Formatierung obliegen (wie wissenschaftliche Arbeiten), kann der Versuch für jede einzelne Seite diese Vorgaben einzuhalten schnell ermüdigend werden (so stelle man sich vor, man müsste innerhalb eines 400 Seiten langem Dokumentes zu Beginn eine neue Seite einheften und danach alle folgenden Seitenzahlen manuell ändern). Mit diesen Problemen beschäftigte sich doch bereits Donald E.\ Knuth, wodurch er ein Typsetting-System mit dem (ursprünglichen Namen) \TeX{} entwickelte\citep{texbook}, mit welchem heutzutage ein wesentlicher Teil wissenschaftlicher Arbeiten erstellt werden (#Referenz).

% Warum ist die Übersetzung innerhalb von Browsern einfacher? Impliziert: Im Vergleich zu Was? leitet über zu: Problemen in TeX
\paragraph{Automatische Sprachübersetzung in Browsern ist vergleichsweise einfach}, denn heutzutage kann man vor allem durch die Mächtigkeit von HTML5 und CSS alleinig seinen Webseiten immensen Charme verleihen, und das mit nur wenig Aufwand und Kopfzerbrechen. Mit Ausnahme von nutzerbezogenen Daten ist im Web also davon auszugehen, dass es keinen triftigen Grund gibt anzuzeigende Texte in seinem Back-End verschwinden zu lassen, denn ob diese nun in HTML oder in JavaScript (oder Ähnlichem) verankert sind, ist zunächst irrelevant, da diese Text-Strings, um angezeigt zu werden, ultimativ im Browser zu einem gesamten Hypertext-Dokument (welches der Spezifikation von\cite{whatwg:htmlstandard} folgen muss). Anders ist dies bei \TeX{} (bzw.\ \LaTeX{}), da sich hier anzuzeigende textliche Inhalte auf nahezu unverhersagbare Art und Weise verstecken lassen. (\enquote{nahezu}: vollständig versteckt werden können diese Inhalte selbstverständlich nicht, da ansonsten auch kein \TeX{}-Compiler diese finden würde).
