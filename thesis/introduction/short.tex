\section{Einleitung}
Wohingegen sich die Sprachübersetzung im Web schnell auf gängige Technologien wie DeepL oder Google's Gemini zurückführen lässt, zeigt sich eine ähnliche Übersetzung von \TeX{} und \LaTeX{} Dokumenten nur in ernüchternder Weise verfolgt. Lösungsansätze zu diesem Problem existieren bereits, allerdings gehen diese oftmals Umwege und trennen die Fähigkeiten der \TeX{}-Engine nicht in jedem Fall von den Technologien, welche verwendet werden sollen, um textliche Inhalte einer menschlichen Sprache in eine Andere zu übersetzen.\\
\noindent
Wo eine naive Nutzung solcher Software bereits im Alltag schnell Schwierigkeiten aufzeigt, ist insbesondere in einem wissenschaftlichen und mathematischem Kontext eine gezielte Verwendung der dieser Technologien erstrebenswert, sodass nicht jegliche Texte unabhängig voneinander und kontextlos übersetzt werden. Andernfalls wäre es denkbar, dass das deutsche Wort \enquote{ungerade} seine Bedeutung gegenüber einer mathematischen Operation verliert (nach welcher eine Zahl modulo 2 in 1 resultiert) und als umgangssprachliches \enquote{schief} interpretiert wird und im Englischen respektiv als \enquote{odd}, bzw.\ \enquote{crooked} übersetzt werden würde. Neben einer solchen Erhaltung von Kontexten ist auch eine selbstständige Erkennung der zu übersetzenden Sprache (Originalsprache eines Dokumentes) interessant, jedoch nicht zwingend erforderlich.\\
\noindent
Weiterhin dürfen Übersetzungsprozesse selbstverständlich nicht darin enden, dass eine entstehende (bspw.) PDF entweder vollständig unlesbar wird. % TeX Syntax kaputt
Daneben sollten allerdings auch keine unlesbaren Sektionen innerhalb der jeweiligen Dokumente entstehen, die aus von Layouting-Problemen resultieren, welche sich für die Übersetzung in einige Sprachen zeigen (jedoch in einigen Fällen unvermeidbar sind).\\ 
\noindent
Wünschenswert ist neben vorigen Aspekten auch Möglichkeiten für den Endnutzer zu erlauben, sollte dieser spezielle Übersetzungen oder Kontexte für einige Wörter wünschen, welche jedoch nicht aus dem Dokument selbst hervorgehen. % Schaffe ich es in dieser Arbeit das Wort "inhärent" NICHT zu verwenden?
Außerdem sollte ein möglichst hoher Support für sowohl verschiedene menschliche Sprachen, aber auch verschiedene \LaTeX{}-Pakete gegeben sein, wobei Letzteres nur ein Bonus ist, sollten Systeme wie Ti\textit{k}Z, bzw.\ \texttt{pgfplots} oder Bib\TeX{} innerhalb \LaTeX{} (zusammen mit \TeX{}) nutzbar bleiben.% Da sich durch diese sämtliche Verhalten anderer Pakete reproduzieren ließen.