\section{Einleitung}
Die schnellstmögliche und einfache Erstellung von Dokumenten beliebiger Natur (formlos) wird heutzutage oftmals über Produkte bekannter Anbieter abgewickelt (bspw.\ Microsoft's Word, PowerPoint, etc., die vergleichbaren \enquote{LibreOffice}, sowie Apple-Produkte).
Unterliegen Dokumente allerdings strengeren stilistischen Vorgaben (bspw.\ bei wissenschaftlichen Abhandlungen) entsteht der Vorteil, dass sich diese Vorgaben wie ein Regelsatz behandeln lassen, aus welchem bestimmte, feste Dokumenten-Strukturen hervorgehen. Hierzu existiert bereits ein geläufiges System namens \LaTeX{} (mit dem \textit{La} nach einem der ursprünglichen Entwickler~\cite{latexCompanion:leslieLamport}), welches selbst auf dem von~\cite{texbook} entwickelten Zeichensetzungs-System und der verbundenen Programmiersprache \TeX{} basiert.
Die \TeX{}-Syntax selbst basiert auf englischen Begriffen, allerdings ist nicht davon auszugehen, dass nur englischsprachige Menschen \LaTeX{} und \TeX{} nutzen werden. Quelltexte und Dokumentenbeschreibungen werden also nicht immer in einer rein englischsprachigen Form vorliegen (z.B.\ \verb|\chapter{Erstes Kapitel: Einleitung}| oder der Quelltext dieses Werkes). 
Ein Zurückführen solcher Dokumente in die englische Sprache ist einfach, insofern ein Verständnis der deutschen Sprache besteht (\verb|\chapter{First Chapter: Introduction}|). 
Die andere Richtung wirft allerdings eine Mehrzahl an Problemen auf, wenn ohne Vorkenntnisse von \TeX{} (bzw.\ dessen syntaktische Elemente) übersetzt wird. In genanntem Beispiel würde dann das Wort \texttt{chapter} aufgegriffen werden und die Zeichenkette \verb|\Kapitel{Erstes Kapitel: Einleitung}| entstehen (ohne weitere, jedoch mögliche Anpassungen entsteht hier keine Kapitelüberschrift mehr. Das erstere \enquote{Kapitel} würde ignoriert werden und die innerhalb der Klammern stehende Zeichenkette als einfacher Fließtext gedruckt werden). 

~\cite{bellSystemTechnicalJournal:claudeShannon1948:mathematicalTheoryOfCommunication} beschäftigte sich bereits 1948 mit wesentlichen Grundlagen der heutigen Darstellung und Übertragung von Informationen, insbesondere der menschlichen Sprache und Kommunikation. 
Heutige maschinelle Systeme zu diesem Zweck (bspw.\ ChatGPT, DeepL, Gemini und co.) wirken zunächst wie \enquote{magische} Blackboxen, arbeiten jedoch auf Grundlage von statistischen Modellen. Das zugrundeliegende Konzept kann jedoch sehr schnell auf den Punkt gebracht werden:\ 
% Zu überarbeiten:::!!!
Eine künstliche Intelligenz (KI) erhält einen Input, für welchen ein bestimmter Output erwartet wird (Beispiel: \enquote{Einfügen} als nächstes Wort eines zu übersetzenden Satzes und \enquote{insertion} im Kontext innerhalb des Satzes als Erwartung (substantiviertes Verb)), und produziert einen Output, welcher mit dem Erwarteten abgeglichen wird. Sollte das Resultat von der Erwartung abweichen (z.B.\ \enquote{insert} entstehen), so kann dieser Fehler erkannt werden und im Modell dazu beitragen, dass (gegeben einer bestimmten, sequentiellen Folge von Wörtern (in der Satzstruktur)) dieser \enquote{Fehler} von nun an seltener passiert. Allerdings wird klar, dass eine KI unabdingbar Fehler machen muss, denn nur so kann diese \enquote{lernen}. Diese theoretische Grundlage führt dazu, dass immer alle möglichen Permutationen einer Übersetzung in Betracht gezogen werden müssen, so unwahrscheinlich sie auch seien. Angemerkt sei zu dem Vorherigen, dass bekannte und bereits rein in den menschlichen Sprachen auftretende Problem (bzw.\ mögliche Missverständnisse bereits im Verstehen einer Sprache, ausgelöst durch Mehrdeutigkeiten von Wörtern) in dieser Arbeit nicht näher verfolgt werden.

% Wie sich diese Fehler äußern könnten wird im anschließenden Kapitel behandelt. Wie kommt man jetzt wieder von KI zu "einem Übersetzer"?
Bekannte Technologien, wie z.B.\ Google Translate, DeepL und co.\ scheinen rein wort-interne Probleme zu lösen. Sprachliche Missverständnisse könnten aber immer dann entstehen, wenn sich mehrere Sprachen miteinander vermischen, welche sich ein Lexikon teilen (bspw.\ hier:\ \TeX{},\LaTeX{}, \ldots und die, zunächst, englische Sprache). Die Frage ob ein Wort übersetzt werden darf oder nicht, unterscheidet einzelne Problemarten (Fälle). Die Hoffnung besteht, dass diese Probleme bereits gelöst sind, allerdings wird \textit{ein Übersetzer} in der folgenden Schilderung einzelner Problemfälle Fehler machen können \textit{müssen}.

\newpage