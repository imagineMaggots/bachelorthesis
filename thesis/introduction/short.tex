%%%%% Comment-structure:
%%%%%   % Simple comment, explaining the below sentence/paragraph
%%%%%   %%% Introductory remarks for subsections / questions asked / goals to be achieved
%%%%%   %% Intercepting/Concluding remarks for subsection and takeaway for the next subsection.

%%%%% Kommentar-Struktur: 
%%%%%   % Einfacher Kommentar, der den Satz/Paragraphen unter ihm übersetzt
%%%%%   %%% Einleitende Bemerkungen für Unterabschnitte / Gestellte Fragen / zu erreichende Ziele
%%%%%   %% Zwischengrätschende/Abschließende Bemerkung für Unterabschnitt und Takeaway für den Nächsten

%%%%% Unter den Abschnitten meist noch ein PDF-Kommentar evtl. für Prof./Betreuer. Kann helfen

\section{Einleitung}
\subsection{Hintergrund}\phantomsection\label{einleitung:hintergrund}
\pdfcomment{Der erste Satz}
%%% Ziel: Lesergruppe einschränken.
%%% Fragen: Worum geht es (abstrakt)? Dokumentenerstellung -> Dokumente fester Form -> TeX und LaTeX -> Übersetzen von TeX und LaTeX (und wie Probleme entstehen)
% Leser informieren: Es geht um die Dokumentenerstellung. Bisher noch allgemein interessant.
Die schnellstmögliche und einfache Erstellung von Dokumenten beliebiger Natur (formlos) wird heutzutage oftmals über Produkte bekannter Anbieter abgewickelt (bspw.\ Microsoft's Word, PowerPoint, \ldots und die vergleichbaren \enquote{LibreOffice}-Software, sowie die korrespondierenden Apple-Produkte).
% Hier lesergruppe eingrenzen. Es geht nicht um beliebige Dokumente, sondern "wissenschaftlich formatierte" Texte.
Unterliegen Dokumente allerdings strengeren stilistischen Vorgaben (bspw.\ bei wissenschaftlichen Veröffentlichungen) entsteht der Vorteil, dass sich diese Vorgaben wie ein Regelsatz behandeln lässt, aus welchem sich bestimmte, vorprogrammierte Dokumentenstrukturen definieren lassen. Hierzu existiert bereits ein geläufiges System welches den Namen \LaTeX{} trägt (mit dem \textit{La} nach einem der ursprünglichen Entwickler~\cite{latexCompanion:leslieLamport}), welches selbst auf dem von~\cite{texbook} entwickelten Zeichensetzungs-System und der verbundenen Programmiersprache \TeX{} basiert.

% Leser kennen TeX und LaTeX (oder wollen es kennenlernen). Diese Systeme sind aber bereits sehr gigantisch. 
% Eingrenzen/Einführen in die Thematik: Sprachübersetzung, hierzu: Es gibt mehrere Sprachen. 
Die \TeX{}-Syntax selbst basiert auf englischen Begriffen, allerdings ist nicht davon auszugehen, dass nur englischsprachige Menschen \LaTeX{} und \TeX{} nutzen werden. Quelltexte und Dokumentenbeschreibungen werden also nicht immer in einer rein englischsprachigen Form vorliegen (z.B.\ \verb|\chapter{Erstes Kapitel: Einleitung}| oder der Quelltext dieses Werkes). 
% Es gibt mehrere Sprachen innerhalb eines TeX-Dokumentes. Übersetzen aus einer beliebigen Sprache nach EN ist erstmal unproblematisch
Ein Zurückführen solcher Dokumente in die englische Sprache ist einfach, insofern ein Verständnis der deutschen Sprache besteht (\verb|\chapter{First Chapter: Introduction}|). 
% Aber aus EN heraus kann Probleme aufwerfen
Die Rückrichtung zeigt sich allerdings genau dann problematisch, sollte ohne Vorkenntnisse von \TeX{} (bzw.\ dessen syntaktische Elemente) bestehen. In genanntem Beispiel würde dann das Wort \texttt{chapter} aufgegriffen werden und die Zeichenkette \verb|\Kapitel{Erstes Kapitel: Einleitung}| entstehen (ohne weitere, mögliche Anpassungen entsteht hier keine Kapitelüberschrift mehr. Das erstere \enquote{Kapitel} würde ignoriert werden und die innerhalb der Klammern stehende Zeichenkette als einfacher Fließtext gedruckt werden). 

%% Wir wissen: Es geht um die Sprachübersetzung von LaTeX-Dokumenten. 
%% Mancher weiß: Heute nutzt man künstliche Intelligenzen, sind die evtl schlau genug, dass von alleine zu raffen.

\subsection{Thematische Einordnung}\phantomsection\label{einleitung:motivation}
\pdfcomment{TeX-Kommentare meinerseits lauten: Ziel: Verdeutlichen, dass KI nichts Magisches ist, jedoch Varianz trägt. Ziel: Abgrenzen von bekannten Problemen der maschinellen Übersetzung generell}
%%% Ziel: Abgrenzen von bekannten Problemen der maschinellen Übersetzung generell
%%% Ziel: Verdeutlichen, dass KI nichts Magisches ist, jedoch Varianz trägt.
% KI der Einfachheit her zunächst wie "Blackboxen"
Bereits~\cite{bellSystemTechnicalJournal:claudeShannon1948:mathematicalTheoryOfCommunication} beschäftigte sich mit theoretischen Grundlagen der Darstellung und Übertragung von Informationen, insbesondere der menschlicher Sprache und Kommunikation. Heutige maschinelle Systeme zu diesem Zweck (bspw.\ ChatGPT, DeepL, Gemini und co.) wirken zunächst wie \enquote{magische} Blackboxen, arbeiten jedoch auf Grundlage von statistischen Modellen. Spricht man hier von Magie, dann ist jeder Mitarbeiter eines Wetterdienstes oder der Klimaforschung \enquote{bezaubernd}. Das zugrundeliegende Konzept kann jedoch sehr schnell auf den Punkt gebracht werden:\ 
% Warum KI immer Fehler machen können müssten!
Eine KI erhält einen Input, für welchen ein bestimmter Output erwartet wird (Beispiel: \enquote{Einfügen} als nächstes Wort eines zu übersetzenden Satzes und \enquote{insertion} im Kontext innerhalb des Sazes als Erwartung (substantiviertes Verb)), und produziert einen Output, welcher mit dem Erwarteten abgeglichen wird. Sollte das Resultat von der Erwartung abweichen (z.B.\ \enquote{insert} entstehen), so kann dieser Fehler erkannt werden und im Modell dazu beitragen, dass (gegeben einer bestimmten, sequentiellen Folge von Wörtern (Satzstruktur)) dieser \enquote{Fehler} von nun an seltener passiert. Allerdings wird klar, dass eine KI unabdingbar Fehler machen muss, denn nur so kann diese lernen. Dies gilt es stets zu berücksichtigen, wodurch theotische gesehen immer mehrere Permutationen betrachtet werden müssen, sollten Probleme entstehen, in welchen ein Input mehrere Ausgaben erzeugen könnte. Angemerkt sei zu dem Vorherigen, dass bekannte und bereits rein in den menschlichen Sprachen auftretende Problem (bzw.\ mögliche Missverständnisse bereits im Verstehen einer Sprache, ausgelöst durch Mehrdeutigkeiten von Wörtern) in dieser Arbeit nicht näher verfolgt werden.\pdfcomment{Ggf., sollte sich eine geeignete Stelle finden, passend eingebracht und erläutert werden}


%% Zu dem pdf-kommentar

% Okay hier und da wird mal ein Wort ausgelassen, warum ist das in größeren Dokumenten problematisch? Man kann doch das Wort fix googlen? Oh weh wir weichen ab von TeX
% Eine Betrachtung eines \enquote{Übersetzers} als Konzept veranschaulicht die Problematik auf abstrakterer Ebene. Sollte der Kontext des Dokumentes unbekannt sein, werden sich unausweichlich semantische Fehler einschleichen. Bereits~\hyperref[einleitung:hintergrund]{das gezeigte Beispiel} könnte für z.B.\ eine Folie einer Lesung den restlichen Kontext der Seite entfernen und dadurch die Möglichkeit bieten umgangssprachliche Bedeutungen in Wörter zu interpretieren, anstatt einer Mathematischen (bspw.\ \enquote{ungerade} könnte im Englischen \enquote{crooked}, statt \enquote{odd} produzieren). Noch weitere sprachliche Beispiele finden sich schnellig durch Wörter mit zeitlichem/räumlichen Bezug. Der Satz \texttt{Morgen wird es regnen.} könnte ohne das Wort \enquote{morgen} als Frage mit unzureichend eingehaltener deutscher Grammatik interpretiert werden. $($\textit{Wird es regnen?}$)$. Hierbei verliert man eine getroffene Aussage über das Wetter, welches bekanntlicherweise nur schwer vorhergesagt werden kann\pdfcomment{Paragraph wird evtl.\ gelöscht oder komplett verschwinden}.% Siehe ansonsten: Internetseiten des deutschen Wetterdienstes. Evtl braucht das 'nen Anhang???
% reviewed: 2

\newpage

% Daher muss nach einer Lösung gesucht werden, welche diese technischen und sprachlichen Hürden überwinden kann. Neben diesen gänzlich technischen Details, darf die Perspektive des Lesers (wörtlich) nicht missachtet bleiben und kein Übersetzungsprozess darf zu einem \enquote{verstecktem} Inhalten führen. Diese Verbergungen von evtl.\ relevanten Informationen resultieren aus verschiedenensten Layouting-Problemen, ähnlich wie bei der Skalierung von Boxen auf Webseiten (Anhang~\ref{anhang:webFontScaling}) und ist von den jeweiligen Paaren an Sprachen abhängig (zwischen welchen übersetzt wird).\footnote{Zwei adjazente Textfelder müssen sich zwangsläufig überlagern, wenn eine dieser Flächen größer werden muss, da z.B.\ die Textgröße nicht verkleinert werden kann und das Wachstum dieses Textfeldes nur in das Gebiet einer Anderen stattfinden kann. Dies wäre z.B.\ mit Hilfe von Inhaltsangaben auf Lebensmitteln vorstellbar. Sollten die Wörter der ersten Spalte übersetzt werden und dadurch mehr Textfläche nach rechts benötigen, entstehen beschriebene Überlappungen.}% Meint: TikZ und irgendwelche hübschen Bilder; verschiedene Sprachen sind unterschiedlich groß (benötigen unterschiedlich viel Fläche)
% Wünschenswert ist neben vorigen Aspekten auch Möglichkeiten für einen Endnutzer zu finden, welche die Möglichkeit bewahren manuelle Anpassungen vorzunehmen, insofern dies gewünscht ist. Außerdem sollte ein möglichst hoher Support für verschiedene menschliche Sprachen, aber auch verschiedene \LaTeX{}-Pakete gegeben sein, wobei Letzteres nur ein Bonus ist, sollten Systeme wie Ti\textit{k}Z, bzw.\ \texttt{pgfplots} oder Bib\TeX{} innerhalb des \LaTeX{} Dokumentes nutzbar bleiben\pdfcomment{Letzter Teilsatz geht noch nicht eindeutig hervor}.% Da sich durch diese sämtliche Verhalten anderer Pakete reproduzieren ließen.
% reviewed: 1
