\section{Einleitung}
\subsection{Hintergrund}\phantomsection\label{einleitung:hintergrund}
\pdfcomment{Der erste Satz}% Lesergruppe einschränken.
% Alle Leser: Welches Vorwissen brauche ich und kann ich auf Grundlage des Paragraphen erschließen, ob ich weiterlesen sollte?
Nach Jahren des Gedankenmachens darüber, wie man möglichst schnell und einfach Dokumente erstellt, gelangt man heutzutage dabei an, dass zu viele Möglichkeiten für beide Fälle existieren. Bekannte Technologien, wie zum Beispiel Microsofts seit Jahrzehnten bekannte Reihe an Software \enquote{Office} und vergleichbare Software bieten einfache Wege jegliche Wörter und Zeichen, welche benötigt werden, so zu platzieren, dass jegliche Information, welche man darstellen möchte, passend codiert steht.

% Zu schnelle/unvorsichtige Leser: 
Neben solch einfachen Wegen beliebige Dokumente zu erstellen, sollte es nach einer Weile des Erstellens von \enquote{beliebigen} Dokumenten von Interesse werden, dass alle Dokumente, welche man selbst erstellt, einer festen Form folgen. Hierzu müsste man zunächst beschreiben, wie \enquote{ein} Dokument auszusehen hat. Das Kopfzerbrechen hierüber haben jedoch bereits andere Forscher überwunden~\cite{texbook}~\cite{latexCompanion:leslieLamport}.

% Hier gefiltert auf Leute, die TeX und LaTeX kennen und dann Problemschilderung
Eine Herausforderung besteht darin beliebige \TeX{} und \LaTeX{}-Dokumente in andere menschliche Sprachen zu übersetzen. 

% Bitte ab hier nochmal reviewen, danke:
% Warum nicht einfach irgendwelche, bekannte Software zur Übersetzung von irgendwelchen Sprachen nutzen?
Software zur Übersetzung von menschlicher Sprache auf \TeX{}-Quellcode anzuwenden, kann Dokumente erzeugen, welche die \TeX{}-Syntax brechen oder nur zu Teilen und damit unvollständig übersetzt wurden. 
% Warum geht das nicht? Exemplarisch zeigen. Braucht keine Technologien, nur Konzepte (kein google translate, sondern "ein Übersetzer" reicht hier noch)
Mit Hilfe von Google Translate lassen sich wesentliche Gründe ermitteln und welche weitere Folgen aus einem Übersetzungsprozess entstehen können, der nicht von einem \LaTeX{}-Dokument ausgeht (mit:\ \TeX{}, der Programmiersprache für \textit{typesetting} (Zeichensetzung) von~\cite{texbook} sowie der Erweiterung des funktionellen Umfangs durch~\cite{latexCompanion:leslieLamport}).
% Irgendwie hier erst Beispiele. Sehe gerade nur die Highlights in vscode, weil augen überfordert, durch zu viel screen-time
Bereits einfachste Funktionen dieses Systems können Übersetzungen einfacher Zeichenketten verhindern. Ein Übersetzen (via Google Translate) von \verb|hello wor\textit{ld}| liefert nicht \verb|Hallo We\textit{lt}|, sondern \verb|hallo wor\textit{ld}|. Abgesehen von der Frage, wo die kursive Hervorhebung im eigentlichen String erfolgen sollte, würden Leser eines kompilierten Dokumentes das Wort \enquote{Welt} erkennen.% par
Die beschriebene Zeichenkette wird von \TeX{} als \enquote{hello wor\textit{ld}} dargestellt, in welcher das Wort \enquote{world} für einen menschlichen Leser als das englische Wort für \enquote{Welt} eindeutig erkennbar ist. Fehlt die Kenntnis über eine der Sprachen (DE,EN) oder ein natürliches Sprachverständnis verliert die Wortkette einen Teil ihrer Bedeutung.% par
Besonders fatal wird dies, wenn das Auslassen von auch nur einem Wort keine Rückschlüsse mehr auf einen größeren Kontext mehr zulässt.\ \verb|$\mathbb{P}$robability density function| wäre ein denkbarer stilistischer Weg bereits in z.B.\ einem Folientitel bereits eine Notation für eine Wahrscheinlichkeitsdichtefunktion einzuführen. Hierbei würde der Verlust des Wortes \enquote{probability} den stochastischen Kontext aufheben. Der Verlust des Wortes \enquote{density} würde einen Kontext innerhalb der Stochastik verändern und ohne das Wort \enquote{function} ist fraglich, wovon die Rede ist. Vor allem in größeren Dokumenten könnten hierdurch Logikbrüche entstehen.% Wahrscheinlichkeitsdichte? Wie dicht Wahrscheinlichkeiten aneinander sind ergibt wenig Sinn. Integration über einen bestimmten Bereich der Wahrscheinlichkeitsdichtefunktion resultiert in einer Wahrscheinlichkeit. 
\\\noindent
% reviewed: 3

%% Warum ist es so schwer _keine_ Konzessivsätze zu schreiben??? warum möchte man sich selbst ständig selber widersprechen?
% Okay hier und da wird mal ein Wort ausgelassen, warum ist das in größeren Dokumenten problematisch? Man kann doch das Wort fix googlen? Oh weh wir weichen ab von TeX
Eine Betrachtung eines \enquote{Übersetzers} als Konzept veranschaulicht die Problematik auf abstrakterer Ebene. Sollte der Kontext des Dokumentes unbekannt sein, werden sich unausweichlich semantische Fehler einschleichen. Bereits~\hyperref[einleitung:hintergrund]{das gezeigte Beispiel} könnte für z.B.\ eine Folie einer Lesung den restlichen Kontext der Seite entfernen und dadurch die Möglichkeit bieten umgangssprachliche Bedeutungen in Wörter zu interpretieren, anstatt einer Mathematischen (bspw.\ \enquote{ungerade} könnte im Englischen \enquote{crooked}, statt \enquote{odd} produzieren). Noch weitere sprachliche Beispiele finden sich schnellig durch Wörter mit zeitlichem/räumlichen Bezug. Der Satz \texttt{Morgen wird es regnen.} könnte ohne das Wort \enquote{morgen} als Frage mit unzureichend eingehaltener deutscher Grammatik interpretiert werden. $($\textit{Wird es regnen?}$)$. Hierbei verliert man eine getroffene Aussage über das Wetter, welches bekanntlicherweise nur schwer vorhergesagt werden kann\pdfcomment{Paragraph wird evtl.\ gelöscht oder komplett verschwinden}.% Siehe ansonsten: Internetseiten des deutschen Wetterdienstes. Evtl braucht das 'nen Anhang???
% reviewed: 2

\newpage
\subsection{Anforderungen}\phantomsection\label{einleitung:tex}
% Wieder zurück zu TeX! Ja gut, hier und da mal etwas TeX Syntax zu übersehen kann ja nicht schaden, solange das Dok. kompiliert... oder?
Genauso wie das Fehlen einzelner Wörter die sprachliche Bedeutung für einen Menschen brechen kann, treten ähnliche Probleme auch in \TeX{} auf. Einzelne \LaTeX{} Makros \textit zu übersetzen kann einen semantischen Verlust für einen \TeX{} Compiler mit sich führen. Als einfaches Beispiel zeigt sich hier die Möglichkeit in bestimmten Fällen eine Dateiendung auszulassen, sollte man ein \LaTeX{} Dokument in mehrere \TeX{} Dateien trennen wollen.\ \verb|\include{clock}| zu \verb|\include{Uhr}| zu übersetzen (wie bspw.\ Google Translate am 06.10.2025) würde nun nicht mehr als \verb|\include{clock.tex}| interpretiert werden, sondern als \verb|\include{Uhr.tex}| (dessen Existenz nicht garantierbar ist).\\\noindent 
% reviewed: 3

Daher muss nach einer Lösung gesucht werden, welche diese technischen und sprachlichen Hürden überwinden kann. Neben diesen gänzlich technischen Details, darf die Perspektive des Lesers (wörtlich) nicht missachtet bleiben und kein Übersetzungsprozess darf zu einem \enquote{verstecktem} Inhalten führen. Diese Verbergungen von evtl.\ relevanten Informationen resultieren aus verschiedenensten Layouting-Problemen, ähnlich wie bei der Skalierung von Boxen auf Webseiten (Anhang~\ref{anhang:webFontScaling}) und ist von den jeweiligen Paaren an Sprachen abhängig (zwischen welchen übersetzt wird).\footnote{Zwei adjazente Textfelder müssen sich zwangsläufig überlagern, wenn eine dieser Flächen größer werden muss, da z.B.\ die Textgröße nicht verkleinert werden kann und das Wachstum dieses Textfeldes nur in das Gebiet einer Anderen stattfinden kann. Dies wäre z.B.\ mit Hilfe von Inhaltsangaben auf Lebensmitteln vorstellbar. Sollten die Wörter der ersten Spalte übersetzt werden und dadurch mehr Textfläche nach rechts benötigen, entstehen beschriebene Überlappungen.}% Meint: TikZ und irgendwelche hübschen Bilder; verschiedene Sprachen sind unterschiedlich groß (benötigen unterschiedlich viel Fläche)
Wünschenswert ist neben vorigen Aspekten auch Möglichkeiten für einen Endnutzer zu finden, welche die Möglichkeit bewahren manuelle Anpassungen vorzunehmen, insofern dies gewünscht ist. Außerdem sollte ein möglichst hoher Support für verschiedene menschliche Sprachen, aber auch verschiedene \LaTeX{}-Pakete gegeben sein, wobei Letzteres nur ein Bonus ist, sollten Systeme wie Ti\textit{k}Z, bzw.\ \texttt{pgfplots} oder Bib\TeX{} innerhalb des \LaTeX{} Dokumentes nutzbar bleiben\pdfcomment{Letzter Teilsatz geht noch nicht eindeutig hervor}.% Da sich durch diese sämtliche Verhalten anderer Pakete reproduzieren ließen.
% reviewed: 1
