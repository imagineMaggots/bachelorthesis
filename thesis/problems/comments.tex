% In dieser Darstellung ist zudem vorteilhaft, dass anhand der gebildeten \enquote{Muster} erkenntlich wird, dass nicht alle 4 bit eines Wortes gepüft werden müssen, um eine Richtigkeit zu bestätigen, wenn man davon ausgeht, dass Zeichenketten auch von rechts nach links gelesen werden können. Diese rein logische Operation führt dazu, dass bei einem Zeitgleichen überprüfen in beide Richtungen beide \enquote{Tests}, nachdem sie sich in der Mitte trafen, im nächsten bit fertig sind. Nach dem Ausschlussverfahren müss nämlich entweder eine \textit{links-nach-rechts} Iteration beachten, wie viele 1 vorliegen, genause wie die \textit{rechts-nach-links}. Jede dieser Iterationen muss dafür den gesamten Prüfungs-Prozess abbrechen können, sobald sie mehr als nur eine 1 liest (was 7 Fehler in zwei Schritten abdeckt). Im (logischen) dritten Schritt würden beide Iterationen ihre gesammelten auf Ungleichheit prüfen. Hätten beide Iteratoren je eine oder keine 1 gelesen, wäre diese Äquivalenzrelation gebrochen, wodurch ein Fehler erkennbar wäre (denn die Abfrage schließt die übrigen 5 Fehler mit ein).\footnote{Vorteil:\ $\mathcal{O}(\log{n}) statt \mathcal{O}(n)$ zur Überprüfung einzelner Zeichenketten auf Fehler. Diese könnte unter Anderem auf Rechtschreibkorrekturen angewendet werden.}% prüfen nochmal
%\begin{table}
%    \centering
%    \begin{tabularx}{\textwidth}{X|XXX}
%        \toprule
 %       Richtige Übersetzungen & Fehlerhafte Übersetzungen & &\\
%%            \verb|1000| & \verb|1111| & \verb|0011| & \verb|1010|\\
%            \verb|0100| & \verb|1110| & \verb|1101| & \verb|0101|\\
%            \verb|0010| & \verb|0111| & \verb|1011| & \verb|0110|\\
 %           \verb|0001| & \verb|1100| & \verb|0000| & \verb|1001|\\
 %   \end{tabularx}
 %   \caption{2-dimensionale Darstellung von kleinerem Quellcode (illustrativ)}\phantomsection\label{tab:problems:exampleExample}
%\end{table}


\begin{comment}
\subsubsection{Interne Referenzen}
Mittels \texttt{ref} oder \texttt{hyperref} (oder ähnlichem) wird auf einen Teil des Dokumentes verwiesen, in welchem ein Kontext für eine Übersetzung gesetzt wird (bspw.\ kann eine Referenz auf den Euklid einen mathematischen Kontext setzen). Jedoch produziert nur die Kenntnis, das eine Referenz auftritt und das Wort (z.B.) \enquote{ungerade} die Übersetzung:\ \enquote{crooked}, statt \enquote{odd}.

Funktionen in \TeX{}, wie bspw.\ das Referenzieren (via \verb|~\ref{key}|) machen es möglich andere Teile des Dokumentes zu erwähnen und dadurch einen Kontext innerhalb eines Satzes oder Paragraphen zu implizieren. Diese Information liegt jedoch nicht direkt beim \enquote{lesen} des Quelltextes vor, sondern erst nach der Auflösung dieser internen Referenz.


\subsubsection{Externe Referenzen}
Mittels \texttt{cite} oder Ähnlichem wird auf ein anderes Werk verwiesen, welches nicht das aktuelle Dokument ist, jedoch einen neuen Kontext für die Übersetzung schafft.


\subsubsection{Laufzeiten}
\paragraph*{Beschreibung}
Zudem müssen einige Instanzen bedacht werden, in welchen zwar nicht eine Übersetzung selbst stattfinden muss, aber Texte in einem Dokument verändert werden müssen, nachdem diese bereits kompiliert wurden, bzw.\ in einer PDF vorliegen, welche ihrerseits angepasst werden müsste, was sich nur mit einem erneuten Kompilieren ändern lässt (da logische Änderungen innerhalb des Dokumentes auftraten).% Zeitformen bitte nochmals prüfen... Meint übrigens: Inhaltsverzeichnisse, Abbildungsverzeichnisse (wenn deren Captions übersetzt werden, wie sie es werden sollten), ...



% Die gröbste Struktur eines Dokumentes entscheidet die Präambel von \TeX{}-Dokumenten, welche sich aus verschiedenen Formen von Befehlen zusammensetzt, von welchen nur begrenzt viele Inhalte übersetzt werden dürften (meist:\ keine).
% Nach der Präambel, also im Teil, der \textit{tatsächliche} Inhalte im Dokument beschreiben soll, ist nach Befehlen (Kommandos) und Umgebungen zu klassifizieren, deren Funktion innerhalb des Quellcodes auch nach einem Übersetzen erhalten bleiben muss. Im Normalfall sei davon auszugehen, dass Kommandos innerhalb eines Dokumentes entweder bestimmte globale Parameter definieren (bspw.\ in der Präambel) oder in andere graphische Elemente (bspw.\ Formelzeichen, Stilisierung von Zeichenketten, \ldots) aufgelöst werden und Umgebungen Änderungen an größeren Teilen eines Abschnittes vornehmen\pdfcomment{beliebiger hierarchischen Höhe im DOM}. Einzelne Elemente der Präambel deuten jedoch darauf hin, welche Art von anderen Technologien (auf \TeX{}-aufbauend) in einem Dokument genutzt werden und wohingegen ein übersetztendes Programm die Suche nach zu übersetzenden Strings über das eigentliche Dokument (bzw.\ einen einzelnen Quellcode) hinaus ausweiten muss.



%Eine Datei nutzt ein Literaturverzeichnis (Bib\TeX{}), pdf\TeX{} (\verb|\pdfcomment{}|), \verb|\footnote|, ein Inhaltsverzeichnis, ein Glossar, 


%\paragraph{Beispielliste}
%\subparagraph*{Inhaltsverzeichnisse, Abbildungsverzeichnisse, Tabellenlisten}

%Ändern wir den Titel eines Paragraphen oder Abschnittes, dann erfasst dies der \TeX{} Compiler beim ersten Durchlauf. Jedoch der String im Inhaltsverzeichnis kann nur verändert werden, sobald diese Information zu Beginn des nächsten Kompilierungsprozesses in der entsprechenden Hilfsdatei vorliegt (\texttt{.toc}).

%\subparagraph{Backrefs}
%\paragraph*{Beispiel}
%Bedarf evtl.\ einer bildlichen Veranschaulichung. Meint: Inhaltsverzeichnis, Tabelle,\ldots vor einer \textit{backwards reference} verschiebt die echte Position der (Phantom-) Sektion. Daher muss zunächst bestimmt werden, wo im Dokument eine Referenz auf ein existierendes Label stattfindet, welches vorherig bereits vergeben wurde\pdfcomment{Ein Erneuern eines Labels via renewcommand ergibt keinen Sinn, da dieses forward references verfälschen würde.}.
%\paragraph*{Erläuterung}
%Ein Verweis auf einen vorherigen Paragraphen kann nur klickbar verlinkt werden, wenn die Information, an welcher Stelle er sich im Dokument befindet, bereits klar ist. Da nach der Referenz weitere Abschnitte folgen können, welche vorherige Elemente mit variabler Größe verändern könnten, muss zunächst die Größe dieser bestimmt sein und gegenüber dieser kann dann der 

%\subparagraph*{Literaturverzeichnisse}
%\paragraph*{Beispiel}
%\paragraph*{Erläuterung}

%\subparagraph*{\enquote{footnotes}}
%\paragraph*{Beispiel}
%\paragraph*{Erläuterung}



%\subsubsection{Figuren und Tabellen}\phantomsection\label{problems:advanced:tables}
%\paragraph*{Beispiele}
%\paragraph*{Beschreibungen}
%\paragraph*{Abstrahierung}

%\subsubsection{Literaturverzeichnisse}\phantomsection\label{problems:advanced:bibtex}
%\paragraph*{Beispiele}
%\paragraph*{Beschreibungen}
%Bib\TeX{} erlaubt es an vielerlei Stelle eigene Strings in einer kompilierten \TeX{}-Datei zu verbergen.
%\paragraph*{Abstrahierung}


%\subsubsection{Category Codes}\phantomsection\label{problems:advanced:catcode}
%\paragraph*{Beispiele}
%\paragraph*{Beschreibungen}
%\paragraph*{Abstrahierung}



\end{comment}