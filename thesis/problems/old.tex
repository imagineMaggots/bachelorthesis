\paragraph*{Whitespace in Parametern}% "Vermisst" meint eigentlich nur, dass man nicht davon ausgehen soll, dass Wörter _nie_ freistehen _könnten_.
Die Hoffnung besteht, dass jegliche \TeX{}-Syntax anhand von Sonderzeichen (bspw.\ \verb|\|,\verb|{|,\verb|}|,\ldots) erkennbar wäre. Dies ist allerdings nicht der Fall, da sich z.B.\ Farben via \verb|\definecolor{super light red}{rgb}{1,.5,.5}| definieren lassen könnten. Diese Zeichenkette enthält ein freistehendes englisches Wort (\enquote{light}). Einerseits kann an jeder Stelle im Dokument das freistehende Wort zu \enquote{Licht} übersetzt werden oder an keiner. Es könnte per se auch der Fall eintreten, dass die Wortfolge im menschlichen Sprachegebrauch innerhalb des Dokumentes erfolgt. Sollte an einer Stelle, an welcher die definierte Farbe \texttt{super light red} gemeint ist, das Wort \enquote{light} übersetzt werden, so würde die Farbe auf ihren initialen Standardwert von $0$ für alle Farbkanäle (also: schwarz) gesetzt sein. Sollte ein Dokument einen schwarzen Hintergrund besitzen, wären diese textlichen Inhalte nicht mehr leserlich.


\paragraph*{Das Makro-Dilemma}% Wie kann man mit Makros schnell mal etwas verstecken? Wie kann es unklar werden welche Sprache übersetzt werden soll? 
Makros sind eine Möglichkeit mehrere \TeX{}-Befehle zusammenzufassen. Vor allem in \LaTeX{} sind eine Vielzahl dieser bereits vordefiniert, jedoch handelt es sich bei diesen meist um Wörter der englischen Sprache (\enquote{meist}: manche dieser englischen Wörter treten auch in anderen Sprachen auf, bspw.\ \textit{paragraph}$\leftrightarrow$\enquote{Paragraph}). Sollte es einem \TeX{}-User leichter fallen in der z.B.\ französischen Sprache zu arbeiten, so könnte dieser beispielsweise neue, französische Makros mit \\\verb|\newcommand{\anglais}{This is some \textit{formatted} \texttt{english} \TeX{}-t}|\\erzeugen. Das vorige Beispiel zeigt zudem auf, wie Texte innerhalb von \TeX{}-Makros \enquote{verschwinden} können und wirft die Frage auf, wann und wie solche Texte übersetzt werden sollten. Am sinnvollsten erscheint zunächst nur Zeichenketten zu übersetzen, welche sich mit der prominentesten Sprache des gesamten Dokumentes decken, welche allerdings nicht ohne weiteres bekannt ist. Selbst wenn in dem gesamten Dokument größtenteils englische Wörter vorliegen, ist eigentlich nur interessant, in welcher Sprache die reinen Strings (welche auf der PDF lesbar erscheinen) geschrieben sind. Selbst diese Information alleine ist theoretisch gesehen noch keine Grundlage für eine Aussage darüber, welche Sprache in solche einem Fall übersetzt werden müsste, da man hier Kenntnis des eigentlichen, entgültigen Dokumentes bräuchte, denn es könnte auch von Interesse sein, innerhalb eines größtenteils z.B.\ deutschsprachigen Dokumentes nur vereinzelte, englische Sätze zu übersetzen. Hierauf wird in Abschnitt~\ref{subsec:weitereschwierigkeiten} näher eingegangen, da sich dieses Problem zunächst recht einfach durch eine Auswahlmöglichkeit der Ausgangssprache (= die zu Übersetzende) lösen ließe.\\% und da man davon aussgeht, dass einzelne Dokumente üblicherweise, überwiegend in ein- und derselben Sprache verfasst werden und eher seltener Sprachwechsel vorkommen.
\noindent
Gleiches ist zu berücksichtigen, sollte das Kommando \verb|\renewcommand| verwendet werden, wobei dieses allerdings noch ein wenig mehr zulässt. Hiermit ist man auch dazu in der Lage existierende Befehle der \LaTeX{}-Syntax zu ändern, wodurch ein \verb|\Abschnitt{Einleitung}| ebenfalls valide \LaTeX{}-Syntax werden könnte, welche ein \TeX{}-Compiler als \verb|\section{Einleitung}| richtig interpretieren könnte, aber ein übersetzendes Programm könnte dieses womoglich in \verb|\section{example}| überführen. Dies scheint zunächst kein Problem zu sein, jedoch hätte zwischen einem \verb|\renewcommand{\section}{\Abschnitt}| genauso ein \verb|\renewcommand{\section}{\frac{1+\sqrt{5}}{2}}| stattfinden können, wodurch \verb|\section{example}| nicht in einem Abschnitt mit Titel \enquote{example}, sondern in $\frac{1+\sqrt{5}}{2}${example} resultieren würde.


\paragraph*{Umgebungen} 
\begin{Verbatim}
    \newenvironment{boxed}[2][this is an example]
    {
        \begin{center}
        Argument 1 (\#1)=#1\\[1ex]
        \begin{tabular}{|p|}
        \hline\\
        Argument 2 (\#2)=#2\\[2ex]
    }
    { 
        \\\\\hline
        \end{tabular} 
        \end{center}
    }
\end{Verbatim}
Diese Umgebungen selbst sind zunächst nur von Interesse, wenn sie \textit{default} Werte beinhalten, wie obiges \texttt{this is an example}, bei welchem es wünschenswert wäre, wenn ein Programm, welches \TeX{} übersetzt, diese erfässt.% und auch im Kontext richtige Übersetzungen wählt


\subparagraph*{Tabellen} 
\begin{Verbatim}[breaklines=true, breakanywhere=true]
    \begin{table}[h!tb]
        \centering
        \begin{tabular}[l r]
            \toprule
                distance $[$m$]$ & time $[$s$]$
            \midrule
                $400$ & $60$   \\ % starting at a fast pace
                $800$ & $121$  \\
                $1200$ & $183$ \\
                $1600$ & $242$ \\
                $2000$ & $300$ \\ % starts to sprint
                $2400$ & $350$ \\
                $2800$ & $420$ \\ % starts feeling fatigued
                $3200$ & $470$ \\
                $3600$ & $550$ \\ % fatigue ultimately looses time from this point on
                $4000$ & $710$ \\ 
            \bottomrule
        \end{tabular}
        \caption{Track-record of a fictional runner's pace on \today. This table requires the packages \texttt{caption} and \texttt{booktabs}!}
        \label{tab:1}
    \end{table}
\end{Verbatim}



\paragraph*{Definitionen} 
\begin{Verbatim}[breaklines=true, breakanywhere=true]
    \def\a{Nummer}
    \def\b{Zahl}

    \ifx\a\b
    This sentence will be expected to be read, if this document has been translated into english.
    \else
    Dieser Satz wird zu sehen sein, wenn dies ein deutsches Dokument ist.
    \fi

    \a\b\a\b
\end{Verbatim}
Wenn nun die beiden Definitionen übersetzt werden würden, dann würden die inhaltlichen Aussagen der beiden Sätze stimmen. Dreht man jedoch die Logik der beiden Sätze um (\textit{has}$\rightarrow$\textit{hasn't} und \textit{ein}$\rightarrow$\textit{kein}), dann würde eine Übersetzung die inhaltlichen Aussagen der Sätze verfälschen. Ob in solchen Fällen \texttt{def}'s übersetzt werden sollen, wäre unvorhersagbar, sollte man keine Informationen darüber erhalten können, inwiefern sie logisch genutzt werden. Diese Information liegt jedoch im Dokument vor, zumindest rein theoretisch gesehen und es wäre denkbar, dass ein übersetzendes Programm erst prüfen könnte, ob sowohl \verb|\def| genutzt wurde und danach diese definierten Makros mit einer Logik verknüpft wurden, damit danach innerhalb dieser Verknüpfung (bspw.\ die obige \texttt{if-else}) danach geforscht werden kann, inwiefern dies Rückschlüße darauf liefern kann, ob übersetzt werden soll oder nicht.

\paragraph*{Category Codes} 
Jedem Unicode-konformen Zeichen (welches in einer Textdatei auf einem Computer, bzw.\ innerhalb der \TeX{}-Engine landen kann) könnte eine Bedeutung für den \TeX{}-Parser zugewiesen werden. Die Buchstaben \texttt{c} und \texttt{b} könnten von ihrer Bedeutung mit den Zeichen \verb|{| und \verb|}| gleichgesetzt werden. 
\begin{Verbatim}[breaklines=true, breakanywhere=true]
    {
        \catcode99=1 % c={
        \catcode98=2 % b=}
    
        c\Large This is large textb\\
    }
    and this is regular one.
\end{Verbatim}
% Was meinte ich hier?
Geht man bedacht an die Sache heran und definiert passend jeweilige Makros um:
\begin{Verbatim}[breaklines=true, breakanywhere=true]
    {
        \catcode99=1 % c={
        \catcode98=2 % b=}
    
        c\Large This is large textb\\
    }
    and this is regular one.
\end{Verbatim}
So lassen sich einzelnen Zeichen völlig neue Bedeutungen für die \TeX{}-Engine geben! Vermutlich könnte man sogar dazu in der Lage sein, dass man jegliche Zeichenkette zu einem denkbaren und beliebig interpretierbaren Makro macht, sodass selbst:
\begin{Verbatim}[breaklines=true, breakanywhere=true]
    afcq2h9d.bcshd<
\end{Verbatim} 
äquivalent zu 
\begin{Verbatim}[breaklines=true, breakanywhere=true]
    \section{begin}
\end{Verbatim} 
werden könnte. Hierbei stellt sich jedoch die Frage, inwiefern dies sinnvoll ist und eine Erleichterung in der Erstellung von Dokumenten bietet.

\subsection{Weitere Schwierigkeiten}\label{subsec:weitereschwierigkeiten}
Beabsichtigt ist dieser Abschnitt nicht in der Reihe von Problemen aufgefasst, sondern als Schwierigkeit$($en$)$ formuliert, da man sich hier von den Problemen abwenden würde, welche in der \TeX{}-Syntax auftreten und bei sprachliche Hürden angelangt, welche sich für und zwischen verschiedenen Sprachen zeigen könnten.
% Siehe uni/thesisbsc/tests/prblems/list.md/##9

\paragraph*{Mehrdeutigkeiten} innerhalb einer Sprache führen unter Umständen zu missverständlichen Übersetzungen. Ein recht einfaches Beispiel bietet bereits das sehr allgegenwärtige Wort \enquote{ungerade}, welches je nach Kontext als \enquote{schief} interpretiert werden könnte, oder aber für die Aussage, dass eine Zahl modulo 2 nicht 0 ergibt. Weiterhin existieren selbst sprachunabängig Mehrdeutigkeiten für bestimmte Wörter/Konstante. So muss zum Beispiel für eine \enquote{Meile} je nach Kontext abgewogen werden, ob es sich um eine Seemeile oder eine Landmeile handelt (zwischen welchen immerhin rund 200 Meter Unterschied bestehen).  % Hier würde das von Warren, siehe andere Einleitung (introduction/intro.tex) evtl. eignen für weitere Beispiele, nicht nur welche von mir? k.P.
% Beispiel: Seemeile vs. Landmeile auch sprachenübergreifend unterschiedlich und kontextabhängig

\paragraph*{Redewendungen} sind eine Art und Weise anderweits nicht beschreibbare Inhalte und Situationen zu schildern. Jedoch unterscheiden sich diese je nach Sprache, sodass das deutsche \enquote{Ich glaub ich spinne} im Englischen nur Verwirrung schüren würde, sollte es Wort für Wort übersetzt worden sein.

\paragraph*{Abkürzungen}

\paragraph*{SI-Einheiten}\label{par:siunits}sind die eigentlichen Grundeinheiten, auf welche man versucht physikalische Größen (genauer:\ für diese hergeleiteten Einheiten) zurückzuführen. Mit sehr wenigen Ausnahmen sollte davon auszugehen sein, dass diese international Verwendung finden. Einzig und allein % die Ami's
Distanz- und Masseangaben \textit{könnten} je nach Sprache variieren, wie das imperiale Maß gegenüber dem metrischen Maß (e.g., Zoll und Meile ggb.\ Zentimeter und Kilometern, \texttt{lbs} ggb.\ \texttt{kg}; bei welchen es sich zwar nicht um die \textit{eigentlichen} Grundeinheiten handelt, jedoch in dieser Form in der realen Größenordnung miteinander vergleichbarer werden).% vergleichbarer um nicht näher zu sagen, weil "näher" wäre hier objektiv falsch (1kg\approx 2.205 lbs und 1km\approx 1.6 Landmeilen und 1km\approx 1.8 Seemeilen)

\paragraph*{Wirrer Sprachwechsel} meint ein rapides Springen zwischen verschiedenen Menschensprachen innerhalb eines Dokumentes. Die Fragestellung hierbei ist, inwiefern ein sprachlicher Wechsel innerhalb eines Dokumentes erfasst wird, sollte eine automatische Spracherkennung der Ausgangssprache stattfinden. Dabei können verschiedenste (theoretisch: überabzählbar viele) Fälle auftreten, unter welchen z.B.\ Wechsel aus dem Deutschen in das Englische an beliebieger Stelle im Dokument, satzweisige Wechsel zwischen zwei und mehreren Sprachen, sowie ein nur kurzfristiger Wechsel in eine Sprache, innerhalb eines ansonstig monolingualen Dokumentes, welche allerdings Lexeme dieser beinhält (bspw.:\ ein norwegisches Dokument beinhält ein dänisches Zitat).

\paragraph*{Whitespace} lässt sich in \TeX{} nicht nur mit \' ' erzeugen. Die Zeichen, bzw.\ Zeichenketten \verb|\ |, \verb|\@| und \verb|~| können genauso Freifläche zwischen einzelnen Strings produzieren.
