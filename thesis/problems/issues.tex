%%%%%%%%%%%%% Versuche die deutsche wissenschaftliche Sprache verständlich zu halten:
%%%%%%%%%%%%% - Infinitiv so oft wie möglich (durch Substantivierung oder Erweiterung), damit Texte leichter zu verstehen sind.
%%%%%%%%%%%%% - Reale Konditionalsätze nutzen.
%%%%%%%%%%%%% - Auch ein wenig den Leser abzuholen und nicht nur stumpf und maschinell die beiden oberen Regeln einzuhalten :) 


\section{Problemfälle}
Sprachliche Uneindeutigkeiten können in vielen Sprachen auftreten und dadurch Missverständnisse produzieren (bspw.\ sarkastische Kommentare oder Mehrdeutigkeiten einzelner Worte). Besonders kritisch sind solche Uneindeutigkeiten jedoch für übersetzende Programme, welche Dokumente in eine andere, menschliche Sprache überführen zu suchen, da hier bereits einzelne missinterpretierte Wörter die \TeX{} Syntax brechen könnten, wodurch nur noch ein unzureichender Teil der Beschreibung des Dokumentes bestehen bleibt, aus welcher kein echtes Dokument entstehen kann\pdfcomment{Abschnitt: technische Semantik. Tue mich schwer mit der Namensgebung der Kapitel.}. Das Einhalten der \TeX{} Syntax alleine genügt jedoch nicht, um erwähnte Mehrdeutigkeiten von einzelnen Wörtern zu verhindern, da sie abhängig ihres Kontexts eine andere Übersetzung verlangen\pdfcomment{Abschnitt: Sprachliche Semantik}.% Diese Art von PDF-Kommentar ist sichtbar. Geht in Firefox. Dann gehe ich davon aus, dass es in einer kostenpflichtigen software (acrobat) ebenfalls umgesetzt ist. 
Weiterhin können innerhalb von verschiedenen, in Kombination mit \TeX{} genutzten Systeme einzelne \enquote{Fehler} entstehen. Ansätze zur Behebung dieser sind jedoch in manchen Fällen bereits konzeptionell unmöglich. Hieran anknüpfend existieren einige sehr spezifische sprachliche\pdfcomment{meint:\ sowohl \TeX{} und co., also auch menschliche Sprache}.\\% par
\noindent
Aus kurzer Nachverfolgung und Einschätzung dieser Probleme werden anschließend konkrete Anforderungen gestellt. Eine Erläuterung dieser Art macht es allerdings unabdingbar eine Software zu nutzen, welche auf das übersetzen spezialisiert ist und nicht von sich aus \LaTeX{} oder \TeX{}-konforme Dokumente erwartet. Daher werden Beispiele anhand von Google's Web-Service \enquote{Translate} aufgeführt.% parsec; review: 1




% Bleibt die Bedeutung für die TeX-Engine bestehen?
\subsection{Technische Bedeutung}\phantomsection\label{problems:technological}
% Hiermit sagen wir: Punkte/Elemente der nullten Dimension sind veränderbar.
\subsubsection{Sonderzeichen}\phantomsection\label{problems:dim0}
Manche menschliche Sprachen beinhalten gelegentlich Zeichen, welche keine direkte Bedeutung tragen und kein Teil auch nur eines Wortes der Sprache sind. Denkbare Beispiele hierfür sind Klammern, die (üblicherweise) für ein impliziert Erwähntes, allerdings nicht vorzeiglich verwendetes Wort genutzt werden.% Meint: Eigentlich sollte man das Wort dort nicht oder nicht unbegründet verwenden, jedoch passt es gerade im Redefluss, um den Kontext schneller rüberzubringen (und mit schneller ist immer "besser" verbunden)
Stilistische Mittel für Texte sind prinzipiell nicht an spezifische Zeichen gebunden und ein Programm kann immer davon ausgehen, dass ein \enquote{Tippfehler} entstehen kann. Beispiel~\ref{tab:problems:dim0} zeigt allerdings, wie vereinzelte Zeichenketten produzieren können, welche fälschlich (in diesem Kontext:\ unerwünscht) übersetzt werden.

\begin{table}[h!]
    \centering
    \begin{tabularx}{\textwidth}{X X}
        \toprule
            Original & Übersetzung\\
        \midrule
        % Siehe ~/tests/readme.md für namensgebung und "Wo ist die Datei?"
            Korrekt & \\[-13px] % relative Angabe ggb. echtem Zeilenumbruch (1em) // relativ in dem Sinne: wir gehen von der Position des erwarteten Zeilenumbruches aus und verschieben nach unten oder oben (+ bzw - in der Höhe)
            \commoncode{Test}{../examples/technical/0d/correct_original.tex} & \commoncode{Test}{../examples/technical/0d/correct.tex}\\[1em]
            Unerwünscht & \\[-13px]
            \commoncode{Test}{../examples/technical/0d/wrong_original.tex} & \commoncode{Test}{../examples/technical/0d/wrong.tex}\\[-1em]
        \bottomrule
    \end{tabularx}
    \caption{Der für einen \TeX{} Compiler relevante Befehl \texttt{label} bleibt unverändert, allerdings \texttt{section} wird f\"alschlicherweise als \texttt{Abschnitt} \"ubersetzt}\label{tab:problems:dim0}
\end{table}

\paragraph*{Vermutung}
Fraglich ist, warum \texttt{label} nicht erfasst werden sollte, obwohl die folgenden drei Wörter übersetzt werden. Ein String, welcher menschliche Sprache mit Sonderzeichen vermischt, kann dahingehend interpretiert werden, dass diese Sonderzeichen \textit{wie} Klammern verwendet werden. Seinerseits könnte \verb|:| also nicht als bekanntes \enquote{geteilt} aufgefasst werden, sondern als Klammern. Ersetzt man diese Klammern mit Leerzeichen resultiert aus dem ersten Beispiel \verb|\label problem encounter solve| und in zweitem Beispiel \verb|\section example|. Zu sehen ist hier also bereits, dass Google Translate bei einer, wenn man es so interpretieren möchte, \enquote{Vernestung} zweiten Grades scheitert, jedoch einfache Vernestungen noch erkennt\footnote{\enquote{Vernestung} meint die Verschachtelung von Klammern}.

\newpage

\paragraph*{Takeaway}
Teile der \TeX{}-Syntax lassen sich anhand von \verb|\|, \verb|{|, \verb|}|, \verb|[|, \verb|]|, \verb|$|, \verb|$$| oder \verb|\%| erkennen und müssten daher ausgeschlossen werden. Anders als in mathematischen Formeln zeigen sich Sonderzeichen jedoch nicht paarweise auf, sodass sie nicht paarweise ignoriert werden können. Man kann sich diese Art von Fehlern wie 0-dimensionale Fehler vorstellen, wobei die nullte Dimension hierbei bei einem einzelnen Wort beginnt (welche als Punkte verstanden werden).% von mir zumindest



% Hiermit sagen wir: Kanten/Elemente der ersten Dimension sind veränderbar.
\subsubsection{Leerzeichen}\phantomsection\label{problems:dim1}% Meint: Wort wurde nicht als Syntaktisch relevant erkannt; Wort dürfte nicht übersetzt werden
Dem vorangegangenem Beispiel (zunächst) widersprechend, offenbart ein Auslassen von Zeichen (und dadurch ein Trennen von Worten) eine Vielzahl anderer möglicher Fehler, welche sich glücklicherweise schnell einheitlich beschreiben lassen.% oh fuck non onon on n on onn on. .... habs unironisch gedacht und geschrieben gerade. .-d d..d . .d .a. .d. . wtffff
Hierunter fallen meist freistehende Worter, welche als Parameter für verschiedene \LaTeX{} Umgebungen dienen. Die Übersetzung solcher Parameter kann schnell zu Fehlern in einem \TeX{}-Compiler führen.

\begin{table}[h!]
    \centering
    \begin{tabularx}{\textwidth}{X X}
        \toprule
            Original & Übersetzung\\
        \midrule
        % Siehe ~/tests/readme.md für namensgebung und "Wo ist die Datei?"
            Korrekt & \\[-13px]
            \commoncode{Test}{../examples/technical/1d/correct_original.tex} & \commoncode{Test}{../examples/technical/1d/correct.tex}\\[1em]
            Unerwünscht & \\[-13px]
            \commoncode{Test}{../examples/technical/1d/wrong_original.tex} & \commoncode{Test}{../examples/technical/1d/wrong.tex}\\[-1em]
        \bottomrule
    \end{tabularx}
    \caption{Fehler in einem einzeiligen Dokument}\phantomsection\label{tab:problems:dim1}
\end{table}
\paragraph*{Verdeutlichung}% Warum das ein Problem ist
Die Optionen innerhalb eckiger Klammern lassen Whitespace zu. Dies kann jedoch für die Nutzung einiger Funktionen in z.B.\ wichtigen Paketen wie \texttt{hyperref} dazu führen, dass falsche Wörter übersetzt werden, die ein Kompilieren des Dokumentes verhindern.

\newpage
\paragraph*{Takeaway}% ich nutze "take-away" statt "lessons-learned", da "lessons-learned" bwl-talk ist.........!
Teile der \TeX-Syntax lassen sich nicht nur anhand der~\hyperref[problems:unexpectedCharacters]{zuvor} beschriebenen Zeichenketten erkennen, sondern lassen sich auch in Zeilen wiederfinden. Diese Art von Fehlern bahnt den Weg zu einer Dimension, wodurch nicht nur innerhalb eines Wortes (Punktes), sondern auch zwischen verschiedenen Punkten Fehler entstehen könnten (also innerhalb einer Zeile).







\newpage



% Hiermit sagen wir: Flächen/Elemente der zweiten Dimension sind veränderbar.
\subsubsection{Zeilenbrüche}\phantomsection\label{problems:dim2}
Abstrahiert man nun über einzelne Zeilen hinweg, so wird die folgende Art von Fehlerquelle direkt offensichtlich, sodass sie keiner detaillierten Schilderung mehr bedarf. Sollte man versuchen ein Dokument Zeile nach Zeile zu übersetzen und den Kontext der vorigen Zeile zu ignorieren, so werden schnellig Zeilen übersetzt (obwohl:\ diese Zeilen nicht übersetzt werden durften, da sie Befehle für \TeX{} beinhalten (siehe:~\ref{tab:problems:dim2})).

\begin{table}[h!]
    \centering
    \begin{tabularx}{\textwidth}{X X}
        \toprule
            Original & Übersetzung\\
        \midrule
        % Siehe ~/tests/readme.md für namensgebung und "Wo ist die Datei?"
            Korrekt & \\[-13px]%
            \commoncode{Test}{../examples/technical/2d/correct_original.tex} & \commoncode{Test}{../examples/technical/2d/correct.tex}\\[1em]%
            Unerwünscht & \\[-13px]%
            \commoncode{Test}{../examples/technical/2d/wrong_original.tex} & \commoncode{Test}{../examples/technical/2d/wrong.tex}\\[-1em]%
        \bottomrule
    \end{tabularx}
    \caption{Das Übersetzen der \texttt{hyperref} Optionen würde den Kompilier-Prozess scheitern lassen}\label{tab:problems:dim2}
\end{table}



\newpage


% Hiermit sagen wir: Körper/Elemente der dritten Dimension sind veränderbar.
\subsubsection{Dokumentenbrüche}\phantomsection\label{problems:dim3}
Noch abstrakter wird hier die dritte Dimension erreicht, indem Teile von \TeX{} nicht in einer einzelnen Datei vorliegen müssen, sondern auch in anderen Dateien vorliegen könnten. Diese Tasache wirft eine Vielzahl neuer (und teilweise system-abhängiger) Probleme auf. Hier wurd sich zunächst jedoch nur auf die Fähigkeiten der unveränderten \TeX{}-Engine konzentriert und deren vorgesehene Primitiven (für diesen Zweck).

\begin{table}[h!]
    \centering
    \begin{tabularx}{\textwidth}{X X}
        \toprule
            Original & Übersetzung\\
        \midrule

        % Siehe ~/tests/readme.md für namensgebung und "Wo ist die Datei?"
            Unerwünscht & \\[-13px]%
            \commoncode{Test}{../examples/technical/3d/correct_original.tex} & \commoncode{Test}{../examples/technical/3d/correct.tex}\\[1em]%

            Übersehen & \\[-13px]%
            \commoncode{Test}{../examples/technical/3d/wrong_original.tex} & \commoncode{Test}{../examples/technical/3d/wrong.tex}\\[-1em]%
        \bottomrule
    \end{tabularx}
    \caption{Übersetzung in einem \texttt{include} führt zum nicht-Übersetzen einer Datei. Ein zunächst nicht sonderlich \enquote{interessant} wirkendes Beispiel}\label{tab:problems:dim3}
\end{table}



%%%%%%%%%%%%%%%%%%%%%%%%%
%%%%% AB HIERRRRRRRRRRRRR GEHHHHHHTSS WEITER. BEDENKE ABSENDEFRIST BIS MIDNIGHT!
\newpage


% Bleibt die kontextuelle Richtigkeit bestehen? Wird beispielsweise aus einer Referenz ein anderer Kontext übermittelbar, welcher eine andere Übersetzung provoziert?
\subsection{Sprachliche Semantik}\phantomsection\label{problems:linguistical}
\subsubsection{Ausgangssituation}
Probleme des vorherigen Teils~\ref{problems:technological} zu verhindern, stellt sowohl ein Kompilieren/Entstehen eines übersetzten Dokumentes sicher, als auch ein Erkennen alelr Inhalte von diesem. Dies ist jedoch als minimale~\hyperref[technologies:demands]{Anforderung} zu sehen sein. Die Fähigkeit alle Inhalte eines \LaTeX{} Dokumentes lesen zu können alleinig, zeigt sich jedoch als unausreichend, da das Übersetzen zwischen menschlichen Sprachen kontextuell in verschiedenen Lexemem (Wörtern) enden sollte.%\footnote{Das Übersetzen aus und in Gebärdensprachen ist vorerst nicht im Fokus. Denkbarer Ansatz wäre hier allerdings Dokumente mit Bildern zu erwarten, bei welchen man davon ausgeht, dass diese mit bekannten Methoden der Mustererkennung erkannt und in Wörter (wie sie in den lateinischen Sprachen verwendet werden) überführt werden können und dann maschinell ausgewertet werden können. Hierzu gibt es zahlreiche Technologien~\citep{expertSystemsWithApplications:rastgooRazie2021:signLanguageRecognitionADeepSurvey}, jedoch beschäftigt dies ein Übersetzen aus einer menschlichen Sprache nach \LaTeX{} und weicht damit von der gegebenen Aufgabenstellung ab. (Obwohl auch hierfür Ansätze existieren:~\cite{cornellComputerScienceComputerVisionPatternRecognition:korzhDmitrii2025:speechToLaTeXnewModelsAndDatasetsForConvertingSpokenEquationsAndSentences})}.
% Hier die eigentliche Annahme
Dieser Kontext kann einem Quellcode aus mehr als nur Wörtern entnommen werden. Das Bestimmen des Kontextes erfordert Kenntnis über \textit{bestimmte Elemente} eines Dokumentes, welche nicht vorhersehbar sind und charakterisiert werden könnten. Elemente dieser Art können sich verschieden äußern und sind nicht zwangsweise standardisiert, sodass das Einhalten von eventuell etablierten Paradigma (hinsichtlich einiger instituellen Kontexte) geprüft werden muss. 

Vorangegangenes zeigt dementsprechend erneut eine üppige Menge an erwartbaren Problemen. Damit Grenzen der Realität erhalten bleiben, % siehe Commit: 2025-10-13
müssen sich denkbare Anwendungsfälle von \LaTeX{} auf\ (z.B.) wissenschaftliche Arbeiten oder Veröffentlichungen konzentrieren und die dafür benötigten Darstellungsmöglichkeiten in den Vordergrund rücken (bspw.\ Zitationen, Tabellen, Formeln, Graphiken,\ldots).% Graphiken hier mit Beschreibung zu erwarten. Wird diese erfasst? Was wenn es sich um einfache Plots handelt, jedoch die Formel in der Caption nicht der Formel, welche eigentlich geplottet wird, gleicht? An welcher stelle wird der Kontext erfasst? Wie beim Lesen von oben nach unten oder _rein_ aus der Caption?
Exemplarisch werdem daher Beispiele aufgezeigt, deren Phrasen einen Kontext tragen, der nicht aus der sprachlichen Satzstellung allein hervorgeht. 
% Kontexte nicht aus der sprachlichen Satzstellung alleine hervorgehen.\\% Hier zu unterscheiden von: Wie könnten einzelne Wörter Fehler in TeX produzieren, schon behandelt.



% Examplarische Beispiele: Wenn unendlich viele Beispiele theoretisch gesehen möglich wären, aber "unendlich viel" nicht in die Abschlussarbeit passt. 
\subsubsection{Exemplarische Beispiele}
\paragraph*{Autoren/Titel}% Hier wird noch davon ausgegangen: Alle Info's liegen in dem Dokument vor, eher weniger Bezug auf etwas Technisches. 
%
% Kann der Ansatz eigenständig recherchieren ist die Frage?
%
% Fokus auf einen (1) Autoren. Vielleicht ist (1) ja bekannt. Hey wie cool.
%
Das Erwähnen des wortwörtlichen Namens oder Titels eines Dokumentes/Werkes wäre innerhalb dieses normalerweise nicht zu erwarten, aber aus diesem lassen sich unter Umständen mehr Informationen über den Kontext gewinnen (und:\ beide sind normalerweise in der Präambel eines Dokumentes vorhanden). Handelt es sich um einen bekannten Autoren könnte der Name einen Kontext liefern (welcher jedoch im Vorwege nur als eine Heuristik angesehen werden kann, da der Autor authentifiziert werden muss) oder der Titel trägt den Namen von einer bekannten Methode aus einem wissenschaftlichen Gebiet oder nennt ein solches Gebiet. Denkbare Beispiele umfassen:\ 
\enquote{$[\ldots]$\ of Communication}, \enquote{Bayesian $[\ldots]$}, \enquote{$[\ldots]$\ inference\ $[\ldots]$}
(entnommen aus einigen \textit{subject areas} von ACM:\ Transactions on Probabilistic Machine Learning).
Vergisst aber ein übersetzendes Programm diese Wörter des Titels an einer kritischen Stelle im Text, so würden fälschliche Übersetzungen entstehen.

%
% Kann der Ansatz eigenständig recherchieren ist die Frage?
%
% Fokus auf einen (1) Autoren. Vielleicht vermerkt er eine Referenz auf ein vorheriges Kapitel im Dokument? Kann er wörtlich mit Kapitel 2.1.x oder per \hyperref[chapter2:section1:paragraphx]
%
%
\paragraph*{Dokumenten-interne Verweise}% Bezug auf Paragraphen im Werk mit \ref und Hyperref. Wieder technischeres Problem. Inhalte alle da, 
Hyperreferenzen \enquote{erlauben} zunächst (scheinbar) recht wenig, implizieren jedoch eine Giganz an Möglichkeiten. Es wäre zu erwarten, dass jegliche Information, welche für die Übersetzung eines Satzes erforferlich ist, textlich vorliegt. \TeX{} erlaubt es allerdings auf solch einen Kontext zu referenzieren (via:\ \texttt{ref} oder \texttt{hyperref}) und ihn dadurch zu \textit{implizieren}, statt zu nennen.


\paragraph*{Externe Referenzen}% beinhält: wild reingeknallte links. nicht best practise, aber entstandener Software auf der ersten Seite einen Link zu einem git anzugeben (Fußnote) ist ganz nett für Spätere. Hier: Ohne BibTeX!)
%
% Kann der Ansatz eigenständig recherchieren ist die Frage? 
% 
% Fokus ist nicht auf dem einen Autoren, der ein Werk schreibt, sondern ein beliebiges Werk, welches sich auf die Arbeiten anderer bezieht
%
In \LaTeX{} kann auf verschiedene Arten und Weisen auf anderweitige Quellen verwiesen sein. 


%
% Kann der Ansatz das Dokument verstehen ist die Frage?
%
% 
%
\paragraph*{Formeln, Tabellen}



\paragraph*{Graphiken}% Eig nicht so recht lösbar, nur annäherbar. Hierauf wird näher im "spezifische Technologien" Part eingegangen. Bedarf evtl. Hilfe des Users (ein wenig kästchen umherschieben, bis es "so gut wie möglich" alles lesbar ist)







\begin{comment}

%%%%%%%%%% Für später relevant, für Makros und beliebige Nutzereingaben etc......:!:!:!!::!!::!
%Die Möglichkeiten andere Dateien in einem \LaTeX{} Dokument einzubeziehen, könnten nähern sich einer Unendlichkeit.% Warum? ist schwer zu beantworten, daher sollte man sich zunächst vor Augen führen, für welche Zwecke man auf andere Dokumente innerhalb eines Dokumentes verweisen möchte. // Warum eine Unendlichkeit? Es gibt unendlich viele Unendlichkeiten. Beweis bitte von Prof. Cap
%Ein Beschäftigen mit dieser führt zu nichts\footnote{invers},%, sodass zunächst die Frage gestellt werden müsste, wie diese entstehen können, indem alle denkbaren Arten, wie sich zwei oder mehrere Dateien gegenseitig referenzieren/benötigen könnten, aufgelistet wird. 
%weshalb ein Betrachten möglicher versteckter interner Abhängigkeiten zwischen verschiedenen \LaTeX{} Dokumenten unabdingbar ist.% [...], hence describing the possible internal dependencies of \LaTeX{} documents is necessary. // um von der Formulierung "weshalb mögliche interne Abhängigkeiten zwischen verschiedenen Dateien erläutert werden müssen" wegzukommen, sah ich mich gezwungen in die englische Sprache auszuweichen (da es mir dabei half zu erkennen, wie ich einen direkten Konjunktiv II vermeiden kann).
%%% Inwiefern wirkt sich das auf die gegebene Problematik aus
%Nur einen Teil eines Dokumentes zu übersehen, provoziert kontextuellen Verlust für Übersetzungen (Abschnitt~\ref{problems:dim3}). 
%\begin{enumerate}
%%    \item Permutationen (insgesamt 4, da wir nur zwischen $1$ und $n \textit{(beliebig vielen)}$ zu wechseln wünschen (2*2 Optionen / Urnen=1;Kugeln=2;Farben=2;zurücklegen)): 
%    \item Ein Dokument kann ein Anderes in sich tragen. 
%    \item Ein Dokument kann n Andere in sich tragen
%    \item n Dokumente können ein Anderes in sich tragen
%    \item n Dokumente können n Andere in sich tragen geht aus den beiden vorigen statements hervor und ist redundant (weil wir kennen bereits 1 dokument, welches n Andere tragen kann).
%\end{enumerate}
% Realität es gibt nicht unendlich viele Dokumente, da physikalisch nicht möglich (Archimedis). 
%Hier stoßen wir 


%%% Meint: Hier müssen wir die eigentlichen Literaturverweise kennen, um einen Kontext zu kennen.
\subsubsection{Unerreichbare Informationen}

\paragraph*{Beispiele}
\begin{table}[h!]
    \centering
    \begin{tabularx}{\textwidth}{X X}
        \toprule
            Original & Übersetzung\\
        \midrule
        % Siehe ~/tests/readme.md für namensgebung und "Wo ist die Datei?"; hoffentlich sieht sich der Herr Prof. Dr. rer. nat. habil. nicht den Quellcode an dieser Stelle an.
            Dokument & \\
            \lstinputlisting[language=TeX]{../examples/advanced/literature/example_original.tex} & \lstinputlisting[language=tex]{../examples/advanced/literature/example.tex}\\[2em]
            Bibliothek & \\
            \lstinputlisting[language=tex]{../examples/advanced/literature/example_original.bib} & \lstinputlisting[language=tex]{../examples/advanced/literature/example.bib}\\
            %%% Bemerkung: Übersetzung noch nicht erstellt.
        \bottomrule
    \end{tabularx}
    \caption{Beispiel für einen verpassten literarischen Kontext}\label{tab:problems:nonexisting}% mit reference 
\end{table}

\paragraph*{Beschreibungen}
Ein Dokument erwähnt ein Werk, in welchem es um die C-Programmierung geht. Rein aus den im System vorliegenden Dateien ist kein Kontext für das Wort \enquote{String} erkennbar, sodass ein Zugriff auf eine externe Ressource unabdingbar ist.

% Bemerkung: Nur die Suche (google.com) nach "salomon c programmierung" führt bspw. Gemini zu einer vermuteten Verwechslung mit dem Begriff "System" (Stand: 09.10.2025, 12:29).
% ISBN führt zur gleichen Minute direkt zum Institut (Angewandte Mikroelektronik und Datentechnik)... wobei Thalia denkt, dass ich "1984" online kaufen möchte...
\paragraph*{Abstrahierung}
Einfache Cloud-Architektur. Ein Client möchte auf ein beliebiges Wissen einer Webseite (bzw.\ dem Server und den beanspruchten Speicherplätzen in einem (beliebigen) Rechenzentrum\footnote{Hierbei ist nicht von Festspeicher zu reden. Aus Sicherheitsgründen sei davon auszugehen, dass sich die physischen Adressen des wissensrepräsentierenden Speichers regelmäßig und unvorhersehbar ändern} zugreifen).




%\subsubsection{Figuren und Tabellen}\phantomsection\label{problems:advanced:tables}
%\paragraph*{Beispiele}
%\paragraph*{Beschreibungen}
%\paragraph*{Abstrahierung}

%\subsubsection{Literaturverzeichnisse}\phantomsection\label{problems:advanced:bibtex}
%\paragraph*{Beispiele}
%\paragraph*{Beschreibungen}
%Bib\TeX{} erlaubt es an vielerlei Stelle eigene Strings in einer kompilierten \TeX{}-Datei zu verbergen.
%\paragraph*{Abstrahierung}


%\subsubsection{Category Codes}\phantomsection\label{problems:advanced:catcode}
%\paragraph*{Beispiele}
%\paragraph*{Beschreibungen}
%\paragraph*{Abstrahierung}


%%%%%%%%%%%%%%%%%%%%%%%
%%%%%%%%%%%%%%%%%%%%%%%% In Review. Richtige Inhalte, inadäquate Sortierung.
%%%%%%%%%%%%%%%%%%%%%%%%


\end{comment}

\subsection{Spezifischer Technologien}\phantomsection\label{problems:special}% Verändern die Übersetzung nicht direkt / Können diese nicht direkt verändern, jedoch Unleserlichkeiten (unschön anzusehen) oder Unlesbarkeiten (nicht zu sehen) im kompilierten Dokument erzeugen
% Mitunter NP-schwer (zur behebung dieser muss man meist heuristiken eingehen. was ist eine heuristik? an sich gar nichts, meint nur, dass wir fehler zulassen (bzw. eine fehlerwahrscheinlichkeit)).
Hier wenden wir uns von Problemen einer Übersetzung ab und widmen uns denen eines Lesers. Alle textlichen Inhalte eines Dokumentes zu übersetzen, als auch eine kontextuelle Fachsprache zu bewahren scheint aus abstrakterer Perspektive ausreichen, kann allerdings zu Situationen führen, in welchen Informationen verloren gehen, da diese vom Endnutzer nicht mehr gesehen werden können.% Meint: Wirklich gesehen, da in der PDF hinter etwas Anderem verborgen 
% Passiert wann? Fläche, welche die Texte benötigen werden nach der Übersetzung grßer
% Dazu: Wie viel Fläche braucht die Sprache mit geringster Flächer (für alle grammatikalisch richtigen Wortkombinationen?)
% Dazu: Wie viel Fläche braucht die Sprache mit größter Flächer (für alle grammatikalisch richtigen Wortkombinationen?)
%%% Wie zu bestimmen: folgt.



\subsubsection{Kommentare}\phantomsection\label{problems:advanced:comments}
\paragraph*{Beispiele}
\paragraph*{Beschreibungen}
%- zunächst als Unterklasse von~\ref{problems:unexpectedCharacters} zu erwarten
%- kann jedoch auch~\ref{problems:verticalSpacing} umfassen
Wohingegen sich~\ref{problems:advanced:comments} nicht mit anderen, in Kommentaren referenzierten, Dateien beschäftigt, soll sich hier auf solche Fälle konzentriert werde.
\paragraph*{Abstrahierung}
Hier treffen technische Fehler aus den ersten drei Kategorien (in~\ref{problems:dim0},~\ref{problems:dim1} und~\ref{problems:dim2} geschildert) aufeinander. In die dritte Dimension, also in andere Dateien, wird jedoch (\hyperref[problems:special:comments]{vorerst}) nicht traversiert, da auskommentierte Datei-Einbindungen nicht erfasst werden dürften. 
Ausgehend von~\ref{problems:advanced:comment} wird nun erwartet, dass eine Referenzierung von Dateien erwartet wird, welche sich in Kommentaren verbergen. Dies kann jedoch~\ref{problems:special:sourcecode} beinhalten.


\subsubsection{Dilemmatische Makros}\phantomsection\label{problems:special:macrodilemma}% dilemmatasitische Makros... Kombi aus dilemmatisch und fantastisch?
\paragraph*{Beispiele}
\paragraph*{Beschreibungen}
\paragraph*{Abstrahierung}

\subsubsection{TikZ und Layouting}\phantomsection\label{problems:advanced:layouting}
\paragraph*{Beispiele}
\paragraph*{Beschreibungen}
\paragraph*{Abstrahierung}

\subsubsection{Quellmehrsprachigkeit}\phantomsection\label{problems:special:sourcecode}
\paragraph*{Beispiele}
\paragraph*{Beschreibungen}
\paragraph*{Abstrahierung}
Quelltexte anderer Quellsprachen (Programmiersprachen) können ihrerseits auf andere Dateien verweisen, oder andere Syntaktik tragen. Das Erkennen dieser ist theoretisch gesehen leicht, jedoch praktisch gesehen schnellig zu übersehen. 









\subsection{Weitere Schwierigkeiten}\phantomsection\label{problems:additional}
\paragraph*{Kommentare}


\subsubsection{Glossare und Nomenklaturen}
\paragraph*{Beispiele}
\paragraph*{Beschreibungen}
\paragraph*{Abstrahierung}

\subsubsection{Weitere}
\paragraph*{Beispiele}
\paragraph*{Beschreibungen}
\paragraph*{Abstrahierung}
