% Zusammenfassung der Inhalte und Ziele der jeweiligen Kapitel

%%% Problemfälle (muss fertig sein bis montag):
%%%     - Herangehensweise
%%%         - Wie ist das Kapitel strukturiert?
%%%         -> Kleinste Struktur, welche ein Übersetzer sehen könnte, hin zu immer größer Werdenen
%%%         - Wie sehen Beispiele aus(?)
%%%         -> Abhängig von Größe der Struktur und möglichen Permutationen unterschiedlich
%%%         --> Kleine Struktur, wenige (eine) Permutation -> kurzes Verbatim
%%%         --> Kleine Struktur, viele Permutationen -> möglichst viele auf eine Seite
%%%         --> Große Struktur -> auf möglichst wenig Seiten
%%%     - Elemente in einem \TeX{}-Code
%%%         - Das kleinste Element: ein Befehl (Indikator '\') -> nicht übersetzen
%%%             - Folgt das kleinste Element einer festen Form?
%%%             -> Nein, da Catcode, Verbatim, ...
%%%         - Danach: Optionen -> Achtung, mehrere Arten -> Indikatoren (standardmäßig) '[]','{}' 
%%%             - Wie unterscheiden die sich? 
%%%             - Kann man die ändern? Wo entstehen potentiell unendlich viele Permutationen
%%%         - Danach: Mehrzeilige Optionen / Key-Value-Paare
%%%             - Achtung, mehrere Arten denkbar.
%%%             - Entweder: Value ist übersetzbar (da String-Literal) oder darf nicht übersetzt werden (da: Wert des Strings erfragt)
%%%             - Erwähnen: Insbesondere schwer nachvollziehbar bei der Arbeit mit mehreren Quelltexten (wird ne tolle Überleitung)
%%%     - Elemente in mehreren Quelltextdateien
%%%         - Wofür braucht man aus logischer Perspektive mehrere Dateien
%%%         -> Bibliotheken
%%%         -> Lesbarkeit des Codes (ist schwierig manchmal. bei TeX ist so ab 200 Zeilen adé)
%%%     - Spezifische Unlösbarkeiten
%%%         - kp. den ****, den du dir mit TikZ ausgedacht hast
%%%
%%%
\section{Problemfälle}% 
Schildert alle denkbaren Probleme, auf die ein Übersetzungsprogramm stoßen könnte, wenn es nicht weiß, dass ein \LaTeX{} Dokument vorliegt.
\subsection{Technische Semantik}% Die allgemeine Ausgangslage ist ein unbekanntes Dokument, von einem beliebigen Autoren und Schilderung erfolgt unabhängig von den eigentlichen Inhalten des Dokumentes (im Sinne: "leeres" Dokument, unbedeutend, aber LaTeX).
Listet die Probleme, welche Unverständnis für \LaTeX{} produzieren (hier:\ übergreifend, meint:\ die Organisation dahinter).
Hier noch kein Fokus auf menschensprachliche Inhalte, denn die sprachliche Semantik würde vollständig verloren gehen, sollte ein Dokument nicht produzierbar werden.

\subsubsection{Ausgangssituation (Was macht diese Art von Fehler anders als die danach?)}
\subsubsection{Exemplarische Beispiele}
\begin{itemize}
    \item Sonderzeichen 0d
    \item Leerzeichen 1d
    \item Zeilenbrüche 2d
    \item Dokumentenbrüche 3d
\end{itemize}

\subsection{Sprachliche Semantik}% Die allgemeine Ausgangslage ist ein Dokument, welches Inhalte trägt, aus welchen Rückschlüsse auf die Inhalte möglich sind (bspw. Autoren, Titel, Referenzen, ...). Dokument lässt Rückschlüsse auf die eigentlichen Inhalte zu, aber diese gehen aus dem Quellcode nicht hervor.
Listet die Probleme, welche kontextuell falsche Übersetzungen provozieren. An einigen Stellen eignet sich evtl.\ Google Translate aus technischen Gründen nicht mehr zur Demonstration.
Fokus auf Vorgaben, wie Quellverweise, welche den Kontext des Satzes abhängig von diesen Referenzen ändern und dadurch andere Wörter produzieren sollten.
Kontext-Quellen:
\subsubsection{Ausgangssituation (Was macht diese Art von Fehler anders als die danach?)}
\subsubsection{Exemplarische Beispiele}
\begin{itemize}
    \item Titel/Autoren,
    \item Dokumenten-interne Verweise/Referenzen
    \item Externe Referenzen
    \item Formeln/Tabellen
    \item Graphiken
    \item \ldots
\end{itemize}

\subsection{Spezifischer Technologien}
Listet die Probleme, welche spezielle Techniken berücksichtigen müssten.\\
Meint:\ Mehrfach-Kompilation erforderlich? Wurde bspw.\ ein Kapiteltitel (o.Ä.) übersetzt (table of contents o.Ä)?\\
Meint:\ Fokus auf entstehende Layouting-Probleme durch den unterschiedlichen Platzbedarf geschriebener Sprachen.\\
Meint auch:\ Übersetzen von Quellcodes anderer Programmiersprachen.\\
Meint auch:\ Dilemmas, die durch Makros entstehen könnten.\\
Meint hier endlich:\ Catcode.\\

\subsection{Weitere Schwierigkeiten}
Alltagsbeispiele. Abkürzungen. 