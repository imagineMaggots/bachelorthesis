%%% Problemfälle (muss fertig sein bis montag):
%%%     - Herangehensweise
%%%         - Wie ist das Kapitel strukturiert?
%%%         -> Kleinste Struktur, welche ein Übersetzer sehen könnte, hin zu immer größer Werdenen
%%%         - Wie sehen Beispiele aus(?)
%%%         -> Abhängig von Größe der Struktur und möglichen Permutationen unterschiedlich
%%%         --> Kleine Struktur, wenige (eine) Permutation -> kurzes Verbatim
%%%         --> Kleine Struktur, viele Permutationen -> möglichst viele auf eine Seite
%%%         --> Große Struktur -> auf möglichst wenig Seiten
%%%     - Elemente in einem \TeX{}-Code
%%%         - Das kleinste Element: ein Befehl (Indikatior '\') -> nicht übersetzen
%%%             - Folgt das kleinste Element einer festen Form?
%%%             -> Nein, da Catcode, Verbatim, ...
%%%         - Danach: Optionen -> Achtung, mehrere Arten -> Indikatoren (standardmäßig) '[]','{}' 
%%%             - Wie unterscheiden die sich? 
%%%             - Kann man die ändern? Wo entstehen potentiell unendlich viele Permutationen
%%%         - Danach: Mehrzeilige Optionen / Key-Value-Paare
%%%             - Achtung, mehrere Arten denkbar.
%%%             - Entweder: Value ist übersetzbar (da String-Literal) oder darf nicht übersetzt werden (da: Wert des Strings erfragt)
%%%             - Erwähnen: Insbesondere schwer nachvollziehbar bei der Arbeit mit mehreren Quelltexten (wird ne tolle Überleitung)
%%%     - Elemente in mehreren Quelltextdateien
%%%         - Wofür braucht man aus logischer Perspektive mehrere Dateien
%%%         -> Bibliotheken
%%%         -> Lesbarkeit des Codes (ist schwierig manchmal. bei TeX ist so ab 200 Zeilen adé)
%%%     - Spezifische Unlösbarkeiten
%%%         - kp. den ****, den du dir mit TikZ ausgedacht hast
%%%
%%%
%%%
%%%
%%%
%%%
%%%
%%%
\section{Problemfälle}
\subsection{Herangehensweise}
\TeX{}-Quellcodes \enquote{einfach} zu übersetzen, birgt Gefahr Fehler zu produzieren. Bereits innerhalb eines \TeX{}-Quellcodes können (theoretisch) unendlich viele Fehler entstehen, sollten (wohlmöglich) vorfindbare Zeichenketten unbedacht übersetzt werden (\enquote{unbedacht}:\ ohne vorher zu prüfen, ob das Wort eine semantische Bedeutung für \TeX{} trägt). Bevor sich Unendlichkeiten genähert wird, werden hier, abhängig von der Größe der innerhalb \TeX{}-Quelltexten auftretenden Strukturen (welche der vorgesehenen Struktur folgen~\cite{texbook}), jegliche Form von denkbar auftretenden Fehlerquellen präsentiert. 
Hierbei wären die kleinsten denkbaren Strukturen, sollte man sich auf den eigentlichen Verwendungszweck von \TeX{} konzentrieren, einzelne Zeichen oder Buchstaben. Diese alleinig bilden allerdings noch keine logische Struktur in \TeX{}, sondern erst konkrete Befehle/Funktionen und diesen übergebbare Parameter. Hierbei zeigen sich diverse Wege, wie sowohl die Übergabe dieser Parameter aussehen \textit{könnte},% meint hier: welche Klammern können genutzt werden und wie "jedes Zeichen" als Klammer verstanden werden kann
als auch die Parameter selbst.% Meint/Impliziert: Parameter in den Klammern können codierte Informationen sein (ein bestimmtes String-Muster erwarten, bspw. "1 0 0" für RGB) und key-value-pairs (in welchen ihrerseits abzuwägen ist, ob übersetzt werden soll, oder nicht)
Zwar ist nicht jede Form solcher Parameter unbedingt die Sinnvollste, kann aber mittels \TeX{} produziert werden und muss daher betrachtet sein.% klingt komisch

Aufbauend auf diesen kleinsten Strukturen bahnt sich der Weg Richtung größeren Strukturen, welche sich nicht nur in einem einzelnen Quelltext befinden, sondern mitunter auf mehrere Quelltextdateien zugreifen (müssen), die nicht zwingend denselben syntaktischen Vorgaben obliegen.% "Quelltextdateien" als Implikation für alle möglichen auxilliaries 
Einige dieser Strukturen neben reinem Quellcode auch einzelne String-Literale, welche nicht immer übersetzt werden müssen, aber von Interesse für eine Übersetzung werden könnten.% Meint: Titel von Werken (BibTeX) müssen nicht zwingend übersetzt werden, da der Originaltitel es am Einfachsten macht, das erwähnte Werk zu finden

Hiernach bleiben jedoch einige spezifischere Probleme, für welche sich nicht immer eindeutige Lösungen finden lassen.% so langsam krieg ich das glaube ich besser hin, also das Schreiben ^^

\subsection{Struktur einzelner Beispiele}
Um alle möglichen Permutationen von Problemen abzudecken, die bei einem Übersetzen von \TeX{} entstehen könnten, und dabei in eine übersichtliche Form zu überführen, wird durch verschiedene mögliche/denkbare Darstellungsformen gewechselt. Hierbei kann, aus zunächst, abstraktem Blickwinkel recht einfach nach Größe der Struktur und Zahl an möglichen Permutationen unterschieden werden, wobei weder noch eine konkrete, quantitative Angabe zulassen. Daher unterscheiden sich \enquote{kleine} und \enquote{große} Strukturen mit \enquote{wenigen} oder \enquote{vielen} \textit{erlaubten} Permutationen (in wohlmöglich entstehenden Übersetzungen).% Übersetzungen, die entstehen könnten.
Wenn auch schwer zu beschreiben, lassen sich für diese abstrakten Strukturen schnell einfache Darstellungsmöglichkeiten finden, abhängig von 2-dimensionalem Platzverbrauch, bzw.\ in dem Minimieren dieses. So werden kleine Beispiele mit wenigen Permutationen (innerhalb eines, wie hier, Dokumentes) möglichst nebeneinander gelistet, nachdem sie zuvor kurz beschrieben waren
% Hatte hier die deutsche Grammatik verletzt: englisch: "So small examples with limited permutations will be listed most adjacently" geht gerade eben noch, aber: "So kleine Beispiele mit wenigen Permutationen möglichst nebeneinander gelistet" entstand, da ich das "werden" schon bedacht hatte, es in der englischen Grammatik nicht dort folgte, wo es in der Deutschen hätte platziert werden müssen, wodurch ich es übersprang... was mir erst im nachherein aufgefallen war
(bspw.:\ <Beispiel A> wird Fehler in möglichen Übersetzungen <Beispiel B> oder <Beispiel C> produzieren, da <B> die deutsche Grammatik oder <C> die \TeX{} Syntaktik verletzt).
Bei einer höhreren Zahl an Permutationen von kleinen Strukturen kann es von Interesse sein, diese möglichst dicht aneinander in zwei Dimensionen darzustellen, damit wohlmöglich kleine Unterschiede in diesen leicht erkenntlich zu machen.% Verdeutlichen
% Noch fehlend: Überleitung: kleine Strukturen bilden Große (tun sie das in TeX immer, wirklich? unsicher aktuell)
Große Strukturen mit wenigen (erlaubten) Permutationen sind wiederum leicht zu veranschaulichen, da selbst kleine Fehler in den großen Strukturen verfälschend wirken. Dadurch genügt es ein Beispiel mit einem Fehler (der sich bereits in einer kleineren Struktur zeigte) aufzuführen. 
Solch größere Strukturen werden allerdings problematisch, wenn sie mehrere Permutationen \textit{erlauben}.% Warum?

%%% Ab hier Kriega-zone. (KRIEGe ich es in die Arbeit, oder fliegt es raus?) Beispiele sind teilweise mitnehmbar



Einzelne Beispiele werden nacheinander in einer originalen Datei und einer idealen, richtigen Übersetzung dargestellt.\pdfcomment{Erstreckt sich noch nicht über das hierige Doc}
Anstatt jegliche mögliche Permutation zu listen, eignet es sich konkrete Muster abzuleiten und anhand dieser zu beschreiben, an welchen Stellen übersetzt werden darf und an welchen nicht. 
%
Das einleitende Beispiel (~\ref{tab:problems:example}) zeigt allerdings auf, welche Fehler sich bei einem imaginären Befehl \texttt{ink} zeigen könnten und inwiefern einige es erlauben \textit{würden}\pdfcomment{evtl. gebraucht, falls solche Unterscheidung später benötigt}, dass kleinere Fehler missachtet werden können. Der Befehl selbst soll einen String mit einer bestimmten Farbe hinterlegen und besitzt einen zusätzlichen optionalen Farbparameter. 
\enquote{Richtig} wäre es im originalen String nur das Wort in den geschwungenen Klammern zu übersetzen, da hierbei an keiner Stelle Information verloren geht und das Wort, nachdem es vom Deutschen ins Englische übersetzt wurde, weiterhin so wie vorgesehen hervorgehoben wird.
\enquote{Zulässige} Übersetzungen treten dann auf, wenn nur für die Formatierung (insofern hieraus keine weiteren Probleme entstehen) verloren geht. Im gegebenen Beispiel würde dann zwar die farbige Hinterlegung verloren gehen, das Wort allerdings trotzdem übersetzt werden und würde den Weg in ein Dokument finden, ohne einen sprachlichen Informationsverlust zu riskieren (für den Endnutzer/Leser).\footnote{Selbst bei weißer Schriftfarbe kann das Wort in einem PDF-Reader markiert und kopiert werden.}
\enquote{Unerwünscht} sind Fälle, in denen ein Übersetzen Fehler für die \TeX{}-Engine produziert. Übersetzt man hier z.B.\ \texttt{ink} nach \texttt{Tinte} könnte es sich bei Zweiterem wiederum um einen anderen Befehl handeln, das Wort \textit{Wort} einliest, aber eigentlich den alphanumerischen Wert von \textit{word} erwartet hätte.\footnote{$57_{16}+6f_{16} + 72_{16} + 74_{16} = 25\times 16^1 + 28 = 400$ statt:\ $77_{16}+6f_{16} + 72_{16} + 64_{16} = 26\times 16^1 + 28 = 416$. Wofür der Befehl \texttt{Tinte} einen/den Integer 416 benötigt, kann ich Ihnen allerdings nicht erläutern.} Ein Fehlschlagen des Befehls \texttt{Tinte} würde zwar einen Fehler für den \TeX{}-Parser produzieren, dieser wüsste dann aber, dass dieser Befehl bereits einmal fehlgeschlagen ist und eine neues Kompilieren verlangen, in welchem dieser Befehl und seine Optionen ignoriert werden, wodurch das Wort auch hier im Dokument landen würde. Man kann allerdings nicht bei jedem beliebigen \TeX{}-Befehl davon ausgehen, dass dieses Verhalten einheitlich auftreten wird. Hierbei existieren Fälle, welche dafür sorgen könnten, dass andere Wörter nun nicht mehr Teil eines Dokumentes werden könnten~\ref{}.% Meint das \include{clock} zu \include{Uhr} vs \include{clock.tex} zu \include{clock.tex} Beispiel.
\enquote{Fehlerhaftes} Verhalten beim Übersetzen von \TeX{}-Quelltextdateien führt zu einem Informationsverlust, da das zu übersetzende Wort entweder nicht mehr übersetzt wird oder nicht mehr im Dokument wiederzufinden ist. Sobald man beginnt mit mehreren Dateien ein einziges Dokument zu beschreiben, riskiert ein na\"\i ives Übersetzen nur von einem Quelltext ausgehend, dass aus unerwünschten Fehlern innerhalb von einem Dokument fehlerhaftes Verhalten für das entstehende Produkt (meint:\ die kompilierte PDF) entsteht. 
Abstrahiert man von diesem detaillierterem Beispiel, so sind \enquote{richtige} Übersetzungen frei von Informationsverlust, \enquote{zulässige} Übersetzungen nur dazu fähig Informationen für die graphische Aufbereitung (allerdings nicht den sprachlichen Inhalten) zu entwenden, \enquote{unerwünschte} Übersetzungen dazu in Lage Informationen verbergen können und \enquote{falsche} Übersetzungen fehlende sprachliche Inhalte innerhalb eines Dokumentes, sowie fehlende Übersetzungen dieser. Nicht jede Gruppe von Beispielen führt dazu, dass alle benannten Kategorien auftreten.

% Mit wenigen Ausnahmen lassen sich in vielen Beispielen größere Strukturen innerhalb des Quelltextes erkennen, welche musterhaft geschildert werden.

\newpage


\begin{table}[h!tb]
    \centering
    \begin{tabularx}{\textwidth}{X}
        \toprule
            Englisches Original\\
            \commoncode{Original}{../examples/example/original.tex}\\
        \midrule
            Richtige Übersetzung\\
            \commoncode{Beispielübersetzung}{../examples/example/ideal.tex}\\
        \midrule
            Zulässiges Verhalten\\
            \commoncode{Beispielübersetzung}{../examples/example/okay.tex}\\
        \midrule
            Unerwünschtes Verhalten\\
            \commoncode{Beispielübersetzung}{../examples/example/problematic.tex}\\
        \midrule
            Falsches Verhalten\\[-13px]
            \commoncode{Beispielübersetzung}{../examples/example/bad.tex}\\
        \bottomrule
    \end{tabularx}
    \caption{Mögliche Permutationen in einer Übersetzung eines Befehles mit zwei Parametern}\label{tab:problems:example}
\end{table}

\newpage

\subsection{Elemente in einem \TeX{}-Quellcode}
\subsubsection{Die Präambel}
Der unsichtbare Header eines \TeX{}-Dokumentes zeigt an (zunächst) wenigen, sehr spezifischen Stellen eine Schwierigkeit auf. In dieser sind meistenfalls Informationen enthalten, welche nicht zu übersetzen sind, da sie z.B.\ Parameter für dokumentenweite (globale) Einstellungen setzen. Die Art und Weise \textit{diese} zu setzen ist in späteren Fehlerbeschreibungen theoretisch abgedeckt und wäre daher eigentlich nicht von weiterem Interesse, eignet sich allerdings dazu, einige triviale Fehlerquellen abzudecken (bzw.\ ungeeignete Ansätze). 
So kann man sich nicht immer gewiss sein, man einzelne Zeilen gesondert auswerten kann, da z.B.\ Pakete wie \texttt{hyperref} es in ihren Optionen (\texttt{hypersetup}) erlauben einige Einstellungen durch Zeilenbrüche voneinander getrennt darzustellen. Nur weil ein Quelltext also \TeX{} Syntaktik beinhält (und demnach \TeX{}-Semantik trägt), kann nicht direkt ein ganzer Quelltext von der Übersetzung ausgeschlossen sein, darf aber auch nicht vollständig übersetzt werden, ohne einzelne Zeilen genauer zu betrachten.% Beispiel einfügen, Hyperref denkbar, existiert bereits.
Genauso sind allerdings auch einzelne Zeilen nicht direkt als \enquote{insgesamt \TeX{} syntaktisch} zu betrachten, nur weil ein syntaktisches Element von \TeX{} innerhalb dieses Strings vorliegt. Auch innerhalb einzelner Zeilen (bspw.\ \verb|\title{carriage return, line feed}|) können sowohl \TeX{}-syntaktische (dadurch:\ \TeX{}-Semantik tragende), als auch wortsprachliche Inhalte vorliegen, welche bei einem Übersetzen nicht verloren gehen dürfen. (Genanntes Beispiel müsste in:\ \verb|\title{Karren-Rückkehr, Zeilen-Einspeisung}| oder Ähnliches übersetzt werden).

\subsubsection{Kommandos}
%\subparagraph*{Ohne}
Kommandos selbst sind in der reinen \TeX{}-Syntax durch ein Backslash \verb|\| gekennzeichnet. Die Frage, ob das Wort, das einem Kommando folgt, übersetzt werden sollte, oder nicht, bringt Konflikte\footnote{Kommandos selbst werden hier noch als fester Teil der \TeX{}-Syntax betrachtet und als auszuschließender Teil für den Übersetzer gewertet}. Beispielsweise müsste innerhalb von Auflistungen der nach einem \verb|\item| folgende String übersetzt werden. Andererseits wäre in einer Definition mit \verb|\def| vorerst davon abzusehen den folgenden String zu übersetzen, da dieser ein neues Makro/Kommando/etc.\ definieren zu sucht. Demnach darf \verb|\def\hello| nicht zu: \verb|\def\hallo| werden (es sei denn alle folgenden \verb|\hello| werden auch übersetzt).\footnote{Da aber, wie bereits bekannt, immer eine Wahrscheinlichkeit, so gering sie auch sein mag, besteht, dass an einer nicht vorgesehenen Stelle eine Übersetzung auftreten \textit{könnte}, kann man sich nicht auf die vorherige Aussage verlassen}.

% Muster:
Innerhalb des Quelltextes können also Strukturen auftreten, welche direkt folgende Zeichenketten als syntaktisches Element zur Dokumentenbeschreibung benötigen. Da hierbei beide Fälle auftreten könnten und nicht an dem Kommando selbst erkennbar sind, muss hierbei nach Art des Kommandos selbst abgewogen werden\pdfcomment{um dieses auf eine begrenzte Zahl an \TeX{} Primitiven zurückzuführen, welche darauf hindeuten können, ob übersetzt werden soll, oder nicht. (An dieser Stelle noch zu vorgreifend, daher der Kommentar)}. Demnach genügt es nicht nur reine Befehle anhand des \verb|\| zu erkennen oder als festes Maß dafür zu nehmen, ob ein folgender String (der kein weiteres Muster aufweist) zu übersetzen ist, oder nicht.

\paragraph*{Parameter}
% Welche kennen wir?
\subparagraph*{Geschwungene Klammern}
Auf viele Kommandos (Makros) folgen ein Paar geschwungener Klammern. Die Inhalte dieser sind in einigen Fällen zu übersetzen und in Anderen nicht, aber insgesamt als eine Art Parameter für dieses Kommando zu verstehen. 

Beispielsweise wäre das Kommando \verb|\paragraph{This is a string of many characters}| eines, in welchem die Inhalte der Klammern übersetzt werden sollen und das Kommando \verb|\usepackage{geometry}| ein Beispiel, in welchem das Übersetzen von \verb|geometry| zu \verb|Geometrie| dazu führen würde, dass das entsprechende Paket nicht im Quelltext verwendet wird. Diese Art von Fällen, welche der Übersetzung ausgeschlossen werden soll, ist besonders kritisch, wenn mit mehreren Dateien gearbeitet wird und so würde ein Übersetzen von z.B.\ \verb|\include{clock.tex}| (oder \verb|\include{clock}|) zu \verb|\include{Uhr.tex}| (oder:\ \verb|\include{Uhr}|) dazu führen, dass größere Teile eines Dokumentes komplett ausgeschlossen werden. 
% Muster:
Klammern dieser Art genügen also alleine noch nicht dafür eine Aussage darüber zu treffen, ob innenliegende Strings übersetzt werden dürfen, allerdings diese Zeichen (meint:\ die geschwungenen Klammern) kein Teil direkter Teil einer menschlichen Sprache und man würde davon meist erwarten, dass diese Klammern im Normalfall ein Indikator dafür sind, dass nicht übersetzt werden soll und in bestimmbaren Ausnahmefällen (bspw.\ \verb|\section{}|,\verb|\paragraph*{}|, \verb|\chapter{}|, \verb|\title{}|, etc.) von Interesse für ein Übersetzen sind. 

% Welche könnte es geben?
\subparagraph*{Nutzer-eigene Klammern}
Klammern für Parameter und Optionen belaufen sich nicht nur auf eckige und Geschwungene, sondern jegliche Zeichen könnten als Klammern betrachtbar werden. Einfachstes Beispiel liefert hierbei der \verb|\verb|-command, dessen (auf diesen) folgendes Zeichen als Klammer für die umschlossene Zeichenkette (bis benanntes Zeichen wieder auftritt) zu betrachten sei. Daher wäre ein \verb+\verb|something|+ mit verschiedensten Sonderzeichen denkbar (bspw.\ +,-,*,/, \ldots\ anstatt \verb/|/).
Darüber hinaus lässt es der Befehl \verb|\catcode| zu, dass einzelne Zeichen, die üblicherweise in der standardmäßigen \TeX{}-Syntax eine eigene Semantik tragen (bspw.\ eckige, geschwungene Klammern oder das Backslash, die Tilde oder das At-Zeichen, bzw.\ Wort-Charaktere oder Zahlen), eine neue/andere Semantik ggb.\ des \TeX{}-Parsers erhalten könnten.
%%%%%% BEISPIEL

Hieraus entsteht also eine darstellbare Zeichenkette nach dem Kompilieren von \TeX{}, ein Übersetzer könnte allerdings Strings, wie:\ \verb|\verb AthingA| finden, in welcher ein Versuch einer Rechtschreibkorrektur in einer Übersetzung von nicht:\ \verb|\verb ADingA|, sondern:\ \verb|\verb Ein Ding| enden könnte (wodurch bis zum nächsten \texttt{E} im Text alles als die tatsächliche und nicht weiter interpretierte Zeichenkette gedruckt werden könnte, was in einigen Dokumenten Textüberläufe produzieren kann).\footnote{Inwiefern sich dies auf einen Informationsverlust für ein Dokument selbst auswirkt, ist fraglich, denn eigentlich liegen alle Informationen immer noch in der virtuellen Datei vor.}\par
\verb|Inwiefern sich dies auf einen Informationsverlust für ein Dokument selbst auswirkt, ist fraglich, denn eigentlich liegen alle Informationen immer noch in der virtuellen Datei vor.|



%%% \label{key} und \ref{key}, bzw. deren keys dürfte man prinzipiell auch alle übersetzen, sollten die keys persistent im Dokument bestehen bleiben und entsprechende Referenzierungen richtig im Dokument angelangen.
%%% Da dies allerdings ein unnötiges Fehlerrisiko produziert, sollte davon abgesehen werden.

\subsubsection{Optionen}\par
% Welche kennen wir?
\paragraph*{Darstellungsform}\par
Optionen sind, neben Parametern, eine weitere Möglichkeit einem Kommando in \TeX{} zusätzliche Informationen mitzuliefern, z.B.\ für zusätzliche Formatierung. Aus logischer Reihenfolge müssten diese Optionen \textit{eigentlich} immer vor der Zeichenkette, welche formatiert werden soll, stehen, da erst nach Einlesen der Parameter die Information dieser erkannt und auf den folgenden String angewendet werden kann.% TIKZ IST MAL WIEDER SCHULD
Diese Aussage wäre logisch, gilt allerdings nicht einheitlich (bspw.\ bei der Verwendung von Ti\textit{k}Z-Bibliotheken). Optionen in eckigen Klammern insgesamt einer Übersetzung zu entziehen, verhindert allerdings die Möglichkeit, dass die Optionen einiger Befehle selbst ein String sein könnten, welcher im Dokument gedruckt und somit übersetzt werden muss.
Die übliche \TeX{}-Notation sieht hierbei geschwungene Klammern für zwingend erforderliche Parameter vor und Eckige für optionale. Einzelne Kommandos erwarten zunächst optionale, zusätzliche Parameter in eckigen Klammern, welche vorher festgesetzte wurden und danach übrige, benötigte in Geschwungenen (ein Beispiel hierfür zeigt~\ref{subpar:problems:macros}). Für manche Befehle impliziert allerdings der Befehl selbst (bspw.\ \texttt{includegraphics}), dass eine Parameter folgen muss (hier:\ URL, bei welchen auch erwartet werden muss, dass sie menschensprachliche Wörter/Phrasen enthalten). 
Aus theoretischer Sicht wäre die Menge an möglichen Parametern unendlich, \TeX{} selbst begrenzt diese jedoch auf 9 und setzt damit ein technisches Limit. Dieses ist auf verschiedene Arten umgehbar,\footnote{Einzelne Parameter tragen mehr Informationen\pdfcomment{bei bekannter Codierung} Definitionen mit Relays oder Newcommand in Verbindung mit \textit{key-value}-Paaren (elegantere Lösung, bspw.\ in \texttt{hyperref}'s \verb|\hypersetup| zu finden)} zeigt aber nur Probleme auf, wenn in irgendeiner Form mit Name-Wert-Paaren gearbeitet wird, in welchen der erstere Key nicht übersetzt werden darf, jedoch zweiter Parameter es soll.\pdfcomment{oh oh der Part muss noch größer werden, da er mehr Permutationen mitbringt, als ich vorher gesehen hatte\ldots}.

\begin{comment}
\subparagraph*{Beispiele:}\par
% Hier brauch ich dann "größere" Beispiele (im Sinne: echter Platz, ab hier hübschere Formatierung)
\subparagraph*{Muster:}\par
Die hier anfallenden Fehler äußern sich in sechs Permutationen, in welchen immer nur ein String (\texttt{translatable}) übersetzt werden darf und ein weiterer String  (\texttt{non-translatable}) nicht.
Sie zeigen sich in 
\verb|\command[non-translatable] translatable|, \pdfcomment{bspw.\ \marginpar[translatable] text, benötigt logischerweise den Platz dafür}  
\verb|\command[translatable] translatable| \pdfcomment{bspw.\ \item[translatable] text} 
\verb|\command[non-translatable]{non-translatable}|, 
\verb|\command[non-translatable]{translatable}|, 
\verb|\command[translatable]{non-translatable}|, oder
\verb|\command[translatable]{translatable}|
und sind anhand spezifischer Kommandos erkennbar.
\end{comment}

% Welche könnte es geben?
\paragraph*{Unvorhersagbarkeiten}% -> spezielle fehler
Eigene, definierte Makros erlauben für eine quasi-beliebige Zahl an Stellen, innerhalb welcher Strings erwartet werden könnten. So könnte beispielsweise ein Befehl definiert werden, welcher 7 Eingabeparameter erwartet:\ \verb|\def\whatev a#1a#2a#3a#4a#5a#6a#7{This #1 is #2 to #3 be #4 troublesome. Hence we will only print one hundred now #5#6#7}| bei einem Aufruf der Form \verb|\whatev aMACROagoingatoaverya1a0a0| erwarten, dass sowohl alle menschensprachlichen Zeichenketten innerhalb der Definition (\textit{This is to be troublesome. Hence we will only print one hundred now}), als auch außerhalb dieser (\textit{MACRO going to very}) übersetzt werden und einen schlüssigen Satz bilden. 
\def\whatev a#1a#2a#3a#4a#5a#6a#7{This #1 is #2 to #3 be #4 troublesome. Hence we will only print one hundred now #5#6#7}

Dies ist auf technischer Ebene nachvollziehbar (\verb|\whatev| produziert:\ \enquote{\whatev aMACROagoingatoaverya1a0a0}), kann aber erneut zu unendlichen Möglichkeiten führen, 

\paragraph*{Strukturen}
\subparagraph*{Umgebungen und Makros}
Größere Strukturen in der \TeX{}-Syntax zeigen sich oftmals in sog.\ Umgebungen auf und stellen gerne/oft Graphiken\pdfcomment{Tabellen sind eine Untergattung von Graphiken} dar (wie bspw.\ in Ti\textit{k}Z), müssen dies allerdings nicht zwingend. Genauso wäre zu erwarten, dass größere Texte innerhalb dieser Umgebungen reinen Text-Formatierungen obliegen und damit gänzlich zu übersetzen wären. Allerdings können verschiedene Umgebungen auch eigene Syntaktik tragen (bspw.\ Tabellen), wodurch sich innerhalb von solchen Umgebungen sowohl zu übersetzende Strings, als auch zu Erhaltende (= nicht zu übersetzende Strings) verbergen können. 

\textbf{Vordefinierte}\\
\noindent Insbesondere Problematisch wird die Ermittlung solcher bei Umgebungen, welche nicht mittels entsprechender \texttt{begin} und \texttt{end} Kommandos betreten/verlassen werden, sondern auf welche mittels eigener Symbolik ein access gewährt werden kann. Innerhalb reinen \TeX{}'s sind diese \pdfcomment{glücklicherweise}auf wenige Zeichenketten beschränkt und so können nur ein/zwei Dollar-Zeichen (\$,\$\$) oder \verb|\(|,\verb|\)|, \verb|\[|, \verb|\]|auf den/das Beginn/Ende einer ein-/ mehrzeiligen mathematischen Umgebung hindeuten. Darüberhinaus erlauben gerade diese Umgebungen den Wechsel in \textit{normale} Umgebungen (also hinein in die Textumgebung des eigentlichen Dokumentes, aus welcher solche mathematisch Umgebungen zu fliehen suchten) innerhalb welcher wieder in beschriebene mathematische Umgebungen gewechselt werden kann. Solange auf jeden Wechsel \textit{in} eine solche Umgebung ein Wechsel \textit{aus} einer solchen Umgebung heraus erfolgt, kann \TeX{} dies interpretieren und alle rein textlichen Strings \textit{sollten} übersetzt werden.

\subparagraph*{Tabellen und Formeln}
Tabellarische Strukturen eignen sich für eine übersichtliche Darstellung von z.B.\ Messwerten oder auch mathematischer Formeln. Zwar ist bei z.B. 
\begin{align*}\label{problems:tables:eq}
    \sin(\omega t - k\vec{r})                    &= 0 \\
    \arcsin(\sin(\omega t - k\vec{r}))           &= \arcsin(0) \\
    (\omega t - k\vec{r})                        &= 0 \\
    \omega t                                     &= k\vec{r} \text{ note that: $\omega = 2\pi \text{$f$ (frequency)}$}
    % Ab hier bräuchte man Raum und Zeit. Dieses Beispiel kommt allerdings ohne aus.
\end{align*}
keine \enquote{gewöhnliche} Tabelle zu erkennen, allerdings kann man sich einzelne Terme innerhalb vor und/oder nach einer Äquivalenzrelation wie Inhalte einer Zelle\pdfcomment{geht aus der Satzstellung eindeutig hervor, dass es sich um eine Zelle einer Tabelle handelt? zu prüfen -> Hendrik} vorstellen. Gegebenes Beispiel erhält seine Struktur im \TeX{}-Quelltext durch eine eigene Syntax innerhalb solcher Umgebungen. Hierbei kann allerdings auch der Fall, wie zum Beispiel innerhalb einer Tabelle, Strings einzelner Zellen zu übersetzen sind, andere jedoch nicht. Bereits~\hyperref[problems:tables:eq]{obiges Beispiel} zeigt Wörter, welche interessant für einen Übersetzer sein sollten. Anders ist dies jedoch bei
\begin{table}[h!tb]
    \centering
    \begin{tabularx}{\textwidth}{X}
        \toprule
            English Original \\ 
        \midrule
            \commoncode{Original}{../examples/tables.tex} \\
    \end{tabularx}
\end{table}
\newpage
\begin{table}[h!tb]
    \begin{tabularx}{\textwidth}{X}
        \toprule
            Ideale Übersetzung\\
        \midrule
            \commoncode{Beispielübersetzung}{../examples/tables.tex}\\[-1em]
        \bottomrule
    \end{tabularx}
    \caption{Beispiel für die Vermischung von Tabellarischen Strukturen und Texten}\label{tab:problems:sections}
\end{table}

\newpage

\subparagraph*{Eigene Makros und Logik}\phantomsection\label{subpar:problems:macros}
Unteres Beispiel ist eine Möglichkeit einen String (\newcommand*{\appendstring}[3]{#2 #3 #1}\appendstring{world:}{Hello}{to this}) zu erzeugen. Jedoch ist mit dieser Definition auch der String:\ \appendstring{now}{Goodbye}{for} in der Form \verb|\appendstring{now}{Goodbye}{for}| produzierbar. Diesem Muster entsprechend könnten (theoretisch) unendlich viele eckige Klammern zu übersetzende Strings beinhalten. Dies wäre zunächst unproblematisch unter der Vorannahme, dass alle diese Strings zu übersetzen sind. Was allerdings, wenn eine Definition der Form \verb|\newcommand{\addandappend}[3][1][2]{#1+#2 ist #3 #1+#2}| vorliegt, in welcher für \verb|#3| ein String (hier denkbar:\ gleich) erwartet wird? Solche Erwartungen sind nicht immer vorhersagbar und können dadurch nur schwer beschrieben werden (sind allerdings deterministisch nachvollziehbar, da die \TeX{}-Engine solche Makros schließlich auch interpretieren und rendern kann).% ? hoffentlich
\begin{table}[h!tb]
    \centering
    \begin{tabularx}{\textwidth}{X}
        \toprule
            English Original \\ 
        \midrule
            \commoncode{Original}{../examples/commands.tex} \\
    \end{tabularx}
\end{table}
\newpage
\begin{table}[h!tb]
    \begin{tabularx}{\textwidth}{X}
        \toprule
            Ideale Übersetzung\\
        \midrule    
            \commoncode{Beispielübersetzung}{../examples/commands_translated.tex}\\[-1em]
        \bottomrule
    \end{tabularx}
    \caption{Beispiel für die Fähigkeit, dass Makros einzelne Strings einlesen können}\label{tab:problems:order}
\end{table}

\newpage
Problematisch wird dieser Fall, wenn solche Makros eigene und optionale Parameter in beliebiger Reihenfolge erwarten wollen, wie das folgende Beispiel (~\ref{sunnyRainy}). Übersetzt man z.B. das Wort \enquote{sunny} auch nur an einer Stelle nicht, riskiert man Fälle, in welchen falsche Inhalte angezeigt werden. Auch kann man sich nicht gewiss sein\pdfcomment{und darauf will ich hier hinaus}, dass zu übersetzende Optionen oder Parameter immer in einheitlicher und vorhersagbarer Reihenfolge auftreten werden. Ob man einen String innerhalb eines Makros übersetzen darf, bestimmt sich danach, ob das String-Literal logisch benötigt wird. Dies ist daran erkennbar, dass wie im gelisteten Beispiel Vergleiche mit dem String geschehen und liegt zur Kompilierzeit im Quelltext vor.% da ansonsten: kein Dokument

\begin{table}[h!tb]
    \centering
    \begin{tabularx}{\textwidth}{X}
        \toprule
            English Original \\ 
        \midrule
            \commoncode{Original}{../examples/commandscont.tex} \\
    \end{tabularx}
\end{table}
\newpage
\begin{table}[h!tb]
    \begin{tabularx}{\textwidth}{X}
        \toprule
            Ideale Übersetzung\\
        \midrule    
            \commoncode{Beispielübersetzung}{../examples/commandscont_translated.tex}\\[-1em]
        \bottomrule
    \end{tabularx}
    \caption{Beispiel für die Fähigkeit, dass Makros Strings logisch verarbeiten können}\label{tab:problems:sunnyRainy}
\end{table}
\newpage


%%%%%%%%%
%%%%%%%%% Brauche ab hier ein neues File
%%%%%%%%%
\begin{comment}
\section{Durch mehrere Quellcodes und Hilfsdateien}% Idee für Paket: vorgefertigte Tabelle mit Strings in der ersten Spalte. bekomm ich da irgendetwas interessantes, zufälliges hin?
\subsection{Pakete}
Pakete sind ihrerseits keine, innerhalb eines Quelltextes direkt vorliegende Struktur, aber besitzen die Fähigkeit, dass sie Zeichenketten erwarten/tragen könnten, welche es zu übersetzen gilt. Zwar erfolgt die Einbindung solcher immer auf gleicherlei art und weise, aber hier einzelne befehle sind... übersetzbar... morgen anzupassen, weil ich kann nicht mehr
% https://www.overleaf.com/learn/latex/Writing_your_own_package





\subsection{Klassen}
% https://www.overleaf.com/learn/latex/Writing_your_own_class#General_structure         scheint gleich zu packages. wobei, man kann hier in sowas wie \title (oder \@title) eigene Strings integrieren? glaub ich?


%%%%%%%%%%%%%%%%%%%%%%%%%%%%%%%%%%%%%%%%%%%%%%%%%%%%%%%%%%%%%%%%%%%%%%%%%
%%%%%%%%%%%%%%%% Not according to github-issue-structure %%%%%%%%%%%%%%%%
%%%%%%%%%%%%%%%%%%%%%%%%%%%%%%%%%%%%%%%%%%%%%%%%%%%%%%%%%%%%%%%%%%%%%%%%%

% \command translatable <---> \item wasauchimmer
% \command non-translatable <---> \def \string sumn... / consider talking about \renewcommand here?



\subsubsection{Paragraphen und Abschnitte}
% Meint: \command{translatable}

Der zuvor etablierten logischen Struktur der Beispiellistung folgend, würde man zunächst einen Abschnitt oder Paragraphen erwarten. Technisch gesehen äußern sich die Beispiele allerdings alle sehr ähnlich. Hier erwartet man immer einen Befehl, welcher nicht übersetzt werden darf (da er Teil der logischen Struktur des Dokumentes ist) und einen Wert, welcher diesem Befehl übergeben ist und übersetzt werden soll. Das präsentierte Beispiel~\ref{tab:problems:sections} zeigt hier wohlmöglich auftretende Fälle. Erstrebenswert ist hierbei \textbf{nur} die Übersetzung des in geschwungenen Klammern stehenden Strings, unerwünscht das Übersetzen von sowohl Befehl, als auch den umklammerten Teilen und falsch ein Übersetzen von Befehl ohne den in Klammern stehenden Inhalten.% ich hasse abstraktes denken so fucking sehr
\begin{table}[h!tb]
    \centering
    \begin{tabularx}{\textwidth}{X X}
        \toprule
            English & Mögliche Übersetzung\\
        \midrule
            Richtiges Verhalten & \\[-13px]
            \commoncode{Original}{../examples/sections/original.tex} & \commoncode{Beispielübersetzung}{../examples/sections/ideal.tex}\\[1em]
        \midrule
            Unerwünschtes Verhalten & \\[-13px]
            \commoncode{Original}{../examples/sections/original.tex} & \commoncode{Beispielübersetzung}{../examples/sections/problematic.tex}\\[1em]
        \midrule
            Falsches Verhalten & \\[-13px]
            \commoncode{Original}{../examples/sections/original.tex} & \commoncode{Beispielübersetzung}{../examples/sections/bad.tex}\\[-1em]
        \bottomrule
    \end{tabularx}
    \caption{Abstrakte Struktur der folgenden Beispiele}\label{tab:problems:sections}
\end{table}

Abstrahiertes Muster: \verb|\command{translatable-string}|.


\subsubsection{Kommandos}
\paragraph{Ohne Optionen}
% Meint: \command{non-translatable}

Den Implikationen, welche Referenzen auf Abschnitte eines Dokumentes\pdfcomment{weiter verfolgt in späterem Abschnitt} mit sich bringen, abgesehen, müssen jegliche Fälle abgedeckt werden, in welchen möglicherweise übersetzbare Strings mit Sonderzeichen vermischt stehen könnten. % Hier auch das erste label beispiel denkbar. Siehe: irgendwo im git. ich muss aufräumen und geistig klarkommen. hilfe
Verweise auf vorangegangene oder folgende Paragraphen, Abschnitte oder Inhalte (via \texttt{ref} auf ein \texttt{label}) müssen einer einheitlichen Übersetzung obliegen, in welcher denkbare Label-Tags mit korrespondierenden Referenzierungen auch nach Übersetzung korrespondierend bleiben. % BEISPIEL NOCH ANPASSEN.

\begin{table}[h!tb]
    \centering
    \begin{tabularx}{\textwidth}{X X}
        \toprule
            English & Mögliche Übersetzung\\
        \midrule
            Richtiges Verhalten & \\[-13px]
            \commoncode{Original}{../examples/references/original.tex} & \commoncode{Beispielübersetzung}{../examples/references/ideal.tex}\\[1em]
        \midrule
            Zulässiges Verhalten & \\[-13px]
            \commoncode{Original}{../examples/references/original.tex} & \commoncode{Beispielübersetzung}{../examples/references/okay.tex}\\[1em]
        \midrule
            Unerwünschtes Verhalten & \\[-13px]
            \commoncode{Original}{../examples/references/original.tex} & \commoncode{Beispielübersetzung}{../examples/references/problematic.tex}\\[1em]
        \midrule
            Falsches Verhalten & \\[-13px]
            \commoncode{Original}{../examples/references/original.tex} & \commoncode{Beispielübersetzung}{../examples/references/bad.tex}\\[-1em]
        \bottomrule
    \end{tabularx}
    \caption{Abstrakte Struktur der folgenden Beispiele}\label{tab:problems:referencesInDoc}
\end{table}

Abstrahiertes Muster: \verb|\command{non-translatable-string}|.

\newpage

\paragraph{Kommandos mit Optionen}
\subparagraph{Ohne Übersetzungen} % Ohne bedeutet: im String darf rein gar nichts übersetzt werden
% Meint: \command[non-translatable]{non-translatable}

Abstrahiertes Muster: \verb|\command[non-translatable]{non-translatable-string}|.

\subparagraph{Vorhersagbare Übersetzungen} % Vorhersehbar bedeuted:


\subsubsection{Mathematische Theoreme}
% Meint: \command[translatable]{non-translatable-parameter} oder \command{non-translatable}[translatable]

\newpage



\newpage

\subsubsection{Inhaltsverzeichnisse}



\newpage

\subsubsection{Tabellen}

\newpage

\subsubsection{Auflistung von Tabellen und Abbildungen}
iwtkms
\newpage

\end{comment}