\section{Problemfälle}
\subsection{Herangehensweise}
\subsubsection{Vorgehen}
%%% Ziel betrachten
Auf dem Weg \LaTeX{}-Dokumente zu übersetzen, startet man unausweichlich bei der Betrachtung der Entstehungsweise eines solchen Dokumentes. Üblicherweise soll ein fertiges Dokument in einem allgemein verbreitetem und zugänglichen Format vorliegen. Etabliert hat sich hierbei das Portable Document Format (PDF), welches ursprünglich von der Firma Adobe entworfen und spezifiziert wurde.% Langweilig
Ein \TeX{}-Compiler produziert ein solches Dokument ausgehend von einer einzelnen \texttt{.tex} Datei, welche ihrerseits Quellcode trägt, welcher mehrere Sprachen vermengt (die Programmiersprache \TeX{} und menschliche, druckbare Sprachen).% "druckbar" mag verwirrend klingen, schließt aber Gebärdensprachen aus
Bereits ein \TeX{}-Code besitzt (theoretisch) das Potential unendlich viele Fehler zu tragen, sollten alle von einem Übersetzer vorgefundene Zeichenketten übersetzt werden. 
Der sprachlichen Syntax folgend könnte man bestimmte Elemente eines Quelltextes umgehen, aber die genaue semantische Bedeutung einzelner Strings im Quelltext muss immer dahingegen evaluiert sein, ob die vorliegende Zeichenkette Teil des \enquote{fertigen} Dokumentes werden soll, oder nicht.

%%% Ausgangssituation beschreiben // Nötige Strukturen erklären // Abstrakte logische Strukturen
Klar ist also, dass etwaige Lösungen nicht auf Grundlage der sprachlichen Struktur des Quelltextes arbeiten dürfen, sondern die logische Struktur des Dokumentes verstehen/erkennen sollten um mögliche Fehlinterpretationen von Strings zu vermeiden.
Abhängig von der Größe der innerhalb von \TeX{}/\LaTeX{}-Dokumenten auftretenden kleineren, logischen Strukturen (insb.\ den nach~\cite{texbook} vorgesehenen) treten verschiedene Fehlerquellen auf. Fehler in den Kleinsten bewirken immense Änderungen auf die Inhalte und den Aufbau des entgültigen Dokumentes. Sie tragen zur Folge, dass der übersetzte Quelltext kein äquivalentes Dokument (sprachlich, als auch strukturell) zu der originalen Beschreibung erzeugen kann. 

%%% Ausgangssituation beschreiben // Nötige Strukturen erklären // Bennenung bekannter logischer Strukturen
Ähnlich wie bei der Erstellung von Dokumenten insgesamt teilt sich jedes \LaTeX{}-Dokument in verschiedene Bereiche. Diese können sich sowohl echte/relative Fläche auf einer Seite (einem Papier/Bildschirm), textliche Paragraphen/Abschnitte, Graphiken, Überschriften und dergleichen beziehen, aber auch auf größere Konzepte, wie Inhaltsverzeichnisse, Literaturverzeichnisse (Zitationen, insb.\ Links) oder spezifischere Vorgaben für Dokumente beziehen (bspw.\ die Form wissenschaftlicher Arbeiten).

%%% Ausgangssituation beschreiben // Nötige Strukturen erklären // Realisierung der Strukturen im Kontext
Wohingegen man solche Flächen und Texte auf einem Papier als sehr direkt zugänglich betrachten kann, muss eine rein textliche Beschreibung solcher Dokumente alternativen finden und zum Beispiel Farben und Flächen durch Wörter und Zahlen beschreiben. Zu erwarten wäre also, dass die kleinste zu betrachtende Struktur von \TeX{}-Quelltexten einzelne Zeichen sind. Diese allein bilden allerdings noch keine interessante (als auch logische) Struktur in \TeX{}, sondern erst konkrete Befehle/Funktionen, sowie diesen übergebbare Parameter. Hierbei zeigen sich diverse Wege, wie sowohl die Übergabe dieser Parameter aussehen \textit{könnte}, als auch die Parameter selbst und ob ein solcher Parameter zu übersetzen ist, oder nicht.

%%% Verallgemeinertes Problem schildern // Technische Perspektive beschreiben
Fernab der Sinnhaftigkeit vereinzelter Formen solcher Parameter/-übergaben, sind alle möglichen Äußerungsformen von \TeX{}-Quellcodes zu betrachten.
Aufbauend auf diesen kleinsten Strukturen muss sich nun ein Weg in Richtung größerer Strukturen gebahnt werden, welche teilweise auf andere Quelltexte zugreifen und nicht zwingend denselben, einheitlichen syntaktischen Vorgaben unterliegen.% "Quelltextdateien" als Implikation für alle möglichen auxilliaries 
Besonders Quelltexte, deren absolute Position im System (meint:\ ein beliebiges Betriebssystem) nicht immer bekannt ist, können vereinzelt Strings tragen, deren Übersetzung erforderlich wäre, aber schnell übersehen werden könnten.% Meint BibTeX Citation Style Dateien 

%%% Verallgemeinertes Problem schildern // Allgemeine Perspektive bewahren --- Erwartete Anforderungen der Endnutzer
Aus Sicht des Endnutzers sollen hierbei natürlich alle ablaufenden Prozesse verborgen sein, sodass er/sie/etc.\ nur das (übersetzte) Dokument wahrnimmt. Insbesondere müssen Fehler vermieden sein, welche überschüssige Wörter im Dokument produzieren oder Texte von einer Übersetzung ausschließen, bzw.\ dafür sorgen, dass vereinzelte Texte (beliebiger Größe) nicht den vorgesehenen Weg in das Dokument finden. Zudem muss darauf geachtet werden, dass präzise Formulierungen gewählt werden, sollte ein Text einen spezifischen fachlichen Kontext tragen, in denen bestimmte Übersetzungen von Wörtern erwartet werden, welche von den intuitiven sprachlichen Pendants dieser Wörter abweichen.
Vereinzelte Fällen zeigen Ausnahmen, welche technisch realisierbare Probleme hervorrufen, aber keine eindeutig richtigen Lösungen mit sich führen. Probleme/Fehler dieser Art lassen sich sprachlich, als auch graphisch finden.

\newpage

\subsubsection{Darstellung der Beispiele}
Für einzelne Beispiele werden erwartete \textbf{Soll}-Übersetzungen und mögliche \textbf{Ist}-Übersetzungen gegenüber gestellt. Als Parade soll hierbei das Wort \textit{set} dienen, welches kontext-abhängig eine andere Übersetzung verlangt. Gewünscht wäre in dem hierigen Beispiel, dass das Wort zu \textit{Menge} im Deutschen übersetzt wird, da der Satz \enquote{Let's consider a finite set of vertices and edges.} als \enquote{Betrachten wir nun eine endliche Menge an Knoten und Kanten.} erwartet wird.
\footnote{In diesem Kontext wäre auch die Formulierung \enquote{abgeschlossene} statt \enquote{endliche} denkbar}
Würde man alle Wörter einzeln, für sich selbst, betrachten, dann würde man zu einer Übersetzung von \textit{set} zu \textit{setzen} tendieren. Dies würde allerdings sowohl sprachlich keinen Sinn mehr ergeben, als auch die syntaktischen Regeln der deutschen Sprache brechen (denn:\ Verben werden nicht durch Adjektive beschrieben, sondern durch ein Adverb).
Die Darstellung des erläuterten Beispiels gilt musterhaft für folgende Beispiele.\pdfcomment{Und fliegt heraus, falls dieser Abschnitt ggb.\ der folgenden Listung überflüssig wird. Aktuell finde ich es jedoch sinnvoll den Leser vorher Kenntnis mitzugeben, wie Beispiele aussehen, falls diese/r/\ldots nur diese und nicht die Fließtexte betrachten zu sucht}



%Direkte Quelltext-Beispiele sind dem Anhang~\ref{appendix:c} zu entnehmen. Dieses Kapitel schildert zunächst das gröbere Aussehen der verschiedenen Fehlerarten, statt konkrete Beispiele aufzuführen. 


% Für iwo unten: Titel von Werken (BibTeX) müssen nicht zwingend übersetzt werden, da der Originaltitel es am Einfachsten macht, das erwähnte Werk zu finden

\newpage




%%% Darstellung einzelner Beispiele // Tabellarische Auflistung
\subsection{Elemente in einem Quelltext}

%%% Darstellung einzelner Beispiele // Tabellarische Auflistung // Häufige Fälle

%% Listung von häufig nach seltener (Was benutzt man in \TeX{}? Bzw. in welcher Reihenfolge werden diese Elemente interessant/relevant?): Paragraphen -> Abschnitte -> Fuß-/Randnotizen -> Listen -> Photos und Tabellen -> Untertitel/Beschreibungen -> Mathematische Formeln/Gleichungen -> Umgebungen -> Graphiken und Diagramme -> Referenzen/Verweise -> Literatur- und Inhaltsverzeichnisse -> Eigene Kommandos -> Eigene Umgebungen -> Nutzung von Paketen -> Einbinden von anderen Dokumenten/Quelltexten

%% Story wohl lieber: "Wir bauen ein Dokument auf, wie können wir vorgehen"... zumindest von der Denkweise


\subsubsection{Paragraphen, Überschriften und Anmerkungen}\phantomsection\label{subsubsec:problems:texts}% Alt.: Texte?
Am leichtesten wäre einem Übersetzer das Übersetzen von reinen Fließtexten. Strukturlose und unformatierte Blocktexte erschweren einem Leser (Endnutzer) aber das Nachvollziehen der Inhalte und verhindern es, dass zum Beispiel Inhalte von Paragraphen mit deren Überschriften assoziiert werden könnten, um hierbei den Fokus von der exakten Wortwahl hin zur Bedeutung des Themas zu rücken.% Das gut?
Als einfachstes stilistisches Mittel sind hier, wie angedeutet, Überschriften oder Kapiteltitel denkbar. Diese sind (vorerst) einfach an einer endlichen Menge an Befehlen erkennbar. Wichtig ist in solchen Fällen, dass --- selbstverständlich --- die eigentlichen Befehle nicht übersetzt werden, damit ihre Bedeutung innerhalb der Programmiersprache \TeX{} erhalten bleibt. Konkreter sollte zum Beispiel \verb|\chapter{Bracelets}| zu \verb|\chapter{Armbänder}| werden und nicht in \verb|\Kapitel{Armbänder}|.
% Umfasst auch: footnotes, marginnotes, ...
Parallel hierzu sollten auch Untertitel (Captions), Randbemerkungen (Fußzeile, Randbereiche, als auch sonstige Orte) und der Titel des Dokumentes selbst übersetzt sein. Technisch sind solche Wort- bzw.\ Zeichenketten im Quelltext in der Form \verb|\command{Translatable content}| vorliegend, wobei Kommandos generell nicht übersetzt werden sollten.

% ToDo-Listen sind toll. Tabellen können Ergebnisse der Arbeit gut festhalten.
\subsubsection{Listen und Tabellen}% Hier auch: \command[translatable]
Weitere stilistische Mittel, welche sich innerhalb von Dokumenten finden lassen, sind beispielsweise Listen oder Tabellen. Diese erscheinen auf den ersten Blick nicht als neues Problem, da innerhalb dieser Strukturen freistehende (Fließ-) Texte für gewöhnlich ihren Weg in das Dokument finden sollen. Problematisch können allerdings Fälle werden, in denen solche Strukturen Inhalte tragen, die nicht übersetzt werden dürfen.  

% Oh weh, es wird gerechnet.
\subsubsection{Mathematische Typsetzung}
Üblicherweise wird man in einem \TeX{}-Dokument mathematische Inhalte vorliegen haben, bspw.\ Formeln, Matrizen oder Gleichungen. \TeX{} selbst wechselt hier durch ein Dollarzeichen \verb|$| in den \textit{math mode} hinein (und heraus), aber ein Übersetzer steht zunächst nur vor Fragezeichen. Im ersten Moment wäre davon auszugehen, dass solche \enquote{Mathematik-Bereiche} in einem Dokument keine sprachlichen Inhalte tragen und daher ohnehin nicht übersetzt werden würden. Es muss allerdings darauf geachtet werden, dass diese Bereiche sprachliche Inhalte tragen \textit{könnten} und das in einer entweder direkt erkennbaren Form (da bspw.\ mittels \verb|\text{}| innerhalb des Bereiches ein Textbereich erzeugt wird) oder in einer Form, welche Interpretationsspielraum lässt. 
So wäre es für zum Beispiel \verb|$x = 4 \textbf{means: something is equal to four.}$| klar, dass zu \verb|$x = 4 \textbf{bedeutet: etwas ist gleich vier.}$| übersetzt werden soll. 
In der Zeichenfolge \verb|$x = 4 means: something is equal to four.$| fehlt ein solcher Indikator allerdings, wodurch es fraglich ist, ob eine Übersetzung \verb|$x = 4 bedeutet: etwas ist gleich vier.$| entstehen wird. Genauso könnte der Fall eintreten, dass an der Stelle von einzelnen Zeichen, ganze Wörter als Variablennamen genutzt werden. Beispielsweise sollte \verb|$cars = 3, price = 10\$ $| nicht zu \verb|$Autos = 3, Preis = 10\$ $| werden, da ein Verändern der Variablennamen das Nachvollziehen von mathematischen Herleitungen unmöglich macht. Dementsprechend wäre ein Übersetzen von Wörtern innerhalb mathematischer Umgebungen zwar nicht immer falsch, aber sicherer wäre ein Absehen von diesem (da es vorgesehene und erkennbare Wege gibt, Texte in mathematischen Bereichen zu integrieren).

\subsubsection{Umgebungen und Verweise}% Wird evtl.: Umgebungen und Pakete und dann Verweise einzeln?
Zuvor beschriebene \enquote{mathematische Bereiche} sind nur eine Art spezielle Einstellungen für bestimmte Flächen eines Dokumentes vorzunehmen. Ob innerhalb einer Umgebung übersetzt werden soll, muss für diese spezifisch ausgewertet werden (worauf in~\ref{subsubsec:problems:uniqueenvironments} näher eingegangen wird). Wichtiger ist zunächst, dass die syntaktischen Regeln bei der Nutzung von Umgebungen innerhalb \TeX{} erhalten bleiben. So darf \verb|\begin{escape}| nicht zu \verb|\begin{Flucht}| werden und \verb|\end{escape}| nicht zu \verb|\end{Flucht}|. 
Was zunächst trivial wirkt, kann aus technischer Sicht Probleme hervorrufen. Entgegen der in~\ref{subsubsec:problems:texts} geschilderten Art und Weise mit welcher zu übersetzende Strings erkannt werden könnten, sind in gleicher Form auch Zeichenketten vorliegend, welche nicht übersetzt werden dürften. Dies gilt für mehr als nur die Arbeit mit Umgebungen, welche vermutlich seltener auftreten könnten, denn auch (Hyper-) Referenzen und Links fallen in diese Kategorie. Einfache URL zu erkennen, ist auf Grund deren vorgegebener Struktur sehr leicht 
%(\verb|\url{port-association://location-relative-to-domain.extension}|)
, wodurch man davon ausgehen könnte, dass selbst der Versuch diese zu übersetzen unveränderte Inhalte produziert (da ein URI eine etablierte Struktur ist). Möchte man jedoch Schlüsselwerte nutzen, wie bspw.\ bei \verb|\label| und \verb|\ref| darf nicht die Situation entstehen, in welcher z.B.\ aus \verb|\ref{topic:content}| plötzlich \verb|\ref{Thema:Inhalt}| wird oder das ein veränderter Schlüssel in einem Label zu einem fehlenden Anker für jegliche Referenzierungen des Labels (bzw.\ seiner Dokumenten-internen Position) führen.
\footnote{Hierbei entsteht allerdings die Ausnahmesituation, in welcher \textit{alle} Schlüsselwerte erkannt und übersetzt wurden. Dieser Fall wird genau dann problematisch, wenn dieser Schlüssel in einem anderen Label bereits verwendet wurde.}

Inhalte von geschwungenen Klammern sind demnach individuell auszuwerten. Allerdings ist dies nicht die einzige Art von Klammer, welche eine syntaktische Bedeutung in \TeX{} trägt. Beispielsweise können benannte Labels auch mit \verb|\hyperref| referenziert werden. Dieser Befehl erlaubt es den Schlüssel in eckige Klammern zu schreiben und danach in geschwungenen Klammern den Text zu denotieren, welcher im Dokument als Maske für die Verknüpfung dienen soll. 
Es zeigt sich also eine weitere Form \textit{zu übersetzende} Texte zu erkennen und zwar \verb|\command[non-translatable string]{translatable content}|.
% \marginpar[left text]{right text} might be more suitable as an example?


\subsubsection{Verzeichnisse für bestimmte Elemente}% Meint: Table of Contents, List of Figures/Tables, aber auch Literaturverzeichnisse (Änderungen in der .bib müssen auch im Dokument landen, nicht nur in der Bib) und dergleichen
%\subsubsection{Literaturverzeichnisse}% Was man BibTeX aus einem Quelltext heraus mitgeben kann.


%%% Gefällt mir hier in der Reihenfolge nicht... bekomm ich das an den Anfang / nach oben? Siehe "Umgebungen und Verweise".
\subsubsection{Die Präambel}% Im wesentlichen eigentlich key-value Paare, hypersetup ist nur gutes Beispiel und wird üblicherweise nach dem Einbinden des Paketes direkt verwendet (was aber nicht zwingend ist)

Das gleichnamige Paket, aus welchem dieser \texttt{hyperref}-Befehl stammt, zeigt, wie es einige Pakete erlauben Schlüssel-Wert Paare für das Setzen einiger spezifischer Optionen zu übergeben. Beispielsweise würde der Befehl \verb|\usepackage[pdftitle=something]{hyperref}| den Titel der entstehenden PDF (für einen PDF-Reader/Renderer) zu \enquote{something} ändern. Nachdem das Dokument übersetzt wurde, wäre es durchaus wünschenswert, wenn auch der Titel übersetzt wurde (\verb|\usepackage[pdftitle=Irgendetwas]{hyperref}|). Demnach können auch eckige Klammern Strings enthalten, welche es zu übersetzen gilt. 
Neben Paketen, wie \texttt{hyperref}, erlaubt die Präambel einer Vielzahl an solchen strukturellen Änderungen oder anderen Einstellungen, welche sich über das ganze Dokument erstrecken. 
Insbesondere ist in dieser die Definition von eigenen Umgebungen von Interesse, da diese textliche Inhalte in verschiedenen Formen tragen können, welche es alle zu übersetzen gilt.
\footnote{Makros lassen sich im gesamten Dokument definieren. Umgebungen allerdings ausschließlich in der Präambel.}


%%% Darstellung einzelner Beispiele // Tabellarische Auflistung // Seltenere Fälle
\subsubsection{Eigene Makros}
\paragraph*{Eigene Umgebungen}\phantomsection\label{subsubsec:problems:uniqueenvironments}


\subsubsection{Der Weg zu anderen Quelltexten (und Dateien)}% Überleitung zu mehreren Quelltexten/Paketen
Technisch gesehen trifft man hier nicht direkt auf ein neues Problem, aber insbesondere das Übersehen von direkt eingebundenen, anderen \TeX{}-Quelltexten darf nicht stattfinden. Das Übersetzen vereinzelter Teile bestimmter \TeX{}-Befehle kann diese Situation provozieren. Ein einfaches Beispiel zeigt die (hier) imaginäre Quelltextdatei \texttt{clock.tex}, welche in einem ursprünglichen Dokument eingefügt werden soll.  
Würde beispielsweise \verb|\include{clock}| zu \verb|\include{Uhr}| übersetzen werden, kann dieser Befehl nun nicht mehr als \verb|\include{clock.tex}| interpretiert werden und wird als \verb|\include{Uhr.tex}| aufgegriffen. Unabhängig davon, ob \texttt{Uhr.tex} existiert, werden fälschlicherweise die Inhalte von \texttt{clock.tex} nicht mehr aufgegriffen und werden daher nicht Teil des beschriebenen Dokumentes (\textit{beschrieben} durch den Quelltext).
Das Gleiche gilt auch für den Befehl \verb|\input| und \verb|\includepdf| (oder auch:\ \verb|\includegraphics| und Weitere). Interessant wird an dieser Stelle auch, inwiefern die Inhalte von anderen Dokumenten (PDF) erkannt werden, bzw.\ wie mit diesen Umgegangen wird oder werden sollte/könnte. Dies wird im späteren Kapitel~\ref{subsubsec:problems:pdfinclusion} näher betrachtet.
Fehlerquellen dieser Art sind im Quelltext in der Form \verb|\command{Non-translatable content}| vorliegend.

\newpage
%%% Darstellung einzelner Beispiele // Tabellarische Auflistung // Seltenere Fälle

%% Listung von häufig nach seltener (Was muss man bei der Übersetzung eines Dokumentes, das aus mehrerer Dateien besteht, beachten? Bzw. welche Art von anderen Quelltexten sind bekannt / werden öfters genutzt?): 
%%% Einfach .tex -> .bib -> .sty -> .cls -> .bst -> andere, beliebige Quelltexte

\subsection{Probleme für mehrere Quelltexte}
\subsubsection{Triviale Fälle}% Einbindung von einfachen .tex... da springt das Übersetzen dann einfach in diese Datei rüber. Und Root+Unterverzeichnisse
Bereits bekannt ist, dass Befehle, welche andere Quelltexte einbeziehen, um ein gesamtes Dokument zu beschreiben, nicht übersetzt werden dürfen, dürfen nun aber auch die Inhalte dieser Dateien nicht unübersetzt bleiben.
% Hier: Wenn Befehl \include{document} benutzt wird, dann müssen die Inhalte von document.tex übersetzt werden. Beispielhafte Inhalte von document.tex:
%%% \section{document} this content should be translated!
% muss zu:
%%% \section{Dokument} diese Inhalte sollten übersetzt werden!
% werden!
%% Das Gleiche gilt für jede Form von Input reiner .tex-files!
Wichtig ist bei einem Einbinden von anderen Quelltexten auf diese Art und Weise, dass die Dateipfade relativ zum Ausgangsquelltext vermerkt werden. 
Sieht ein Übersetzer also ein \verb|\input{clock.tex}|, dann muss er diese Datei beginnend im \textit{root}-Verzeichnis des Dokumentes suchen und darf nicht ausgehend von der aktuellen Datei, in welcher er sich befindet, eine Suche beginnen.~\verb|\include| können sich nur im ursprünglichen Quelltext befinden, \verb|\input| kann allerdings auch in einem Quelltext stehen, der seinerseits bereits über ein \verb|\include| oder \verb|\input| eingebunden wurde und könnte in einem Unterverzeichnes von \textit{root} liegen.

\subsubsection{Pakete und Klassen}% Sind andere Quelltexte, da .sty... wird evtl. zur "Präambel"
% Klassen sind an sich das Gleiche, wie Pakete nur andere Dateiendung/Einbindung in TeX. Für den Übersetzer ist _diese_ technische Bedeutung aber zweitrangig
Viele Pakete erlauben es, dass die in Ihnen vorliegenden textlichen Inhalte, die auch im enstehenden Dokument gedruckt werden, angepasst werden können. Dies ist allerdings nicht immer (einfach) möglich. Nicht jedes Paket liefert einen eigenen Befehl hierfür oder muss die spezifischen Makros und deren Definitionen erkenntlich/öffentlich machen, damit deren Funktion reproduziert und gewünschte Strings geändert werden können. Allerdings müssen alle mit einem Dokument assoziierten Dateien (zu welchen Pakete \texttt{.sty} zählen) vorliegen, sobald dieses entstehen soll.
% Hier examplarisches Paket zeigen und wie es übersetzt werden soll.
Nun 

\subsubsection{Zitationsstil und Bibliotheken}% BibTeX-assoziierte Dateien sollen angepasst werden, insb. .bst, nicht nur .bib!
\subsubsection{Nicht zu interpretierende Quelltexte}% Meint: minted und lstlisting




%%% Darstellung einzelner Beispiele // Tabellarische Auflistung // Spezielle technische Probleme
%% Listung von häufig nach seltener (Welche speziellen Aktionen sorgen für uneindeutige Erkennung von den geschilderten Problemen? Genauer, warum ist die Syntax nicht als festes Maß zu werten? Eigentlich: Ausnahmen von den Regeln!)
%% Ändern von einzelnen Zeichen ist möglich -> Einige Befehle erlauben die Nutzung von beliebigen Zeichen als Klammern -> Alte Makros können neu/anders definiert werden -> Zusätzliche Funktionen in digitalen Dokumenten
\subsection{Spezielle/Spezifische Probleme}

\subsubsection{Category Codes}
\subsubsection{Nutzer-eigene Klammern}
\subsubsection{Neudefinition bekannter Makros}% Man kann sich nicht einfach nur die Titel und Autoren merken und Standardmäßig bei jedem bspw. \maketitle die normalen Angaben drucken, da sich dieser Befehl ändern könnte
\subsubsection{PDF-Kommentare}% Nicht wichtig, aber nice to have

%%% Darstellung einzelner Beispiele // Tabellarische Auflistung // Andere Probleme
%% Reihenfolge hier wahrscheinlich nicht mehr so relevant. Sinnvoll ist: Menschensprachliche Probleme (Mehrdeutigkeiten, Abkürzungen, ...) -> Provozierte (technische) Mehrdeutigkeiten -> PDF-Erkennung/Auswertung -> Layouting-Probleme (TikZ und sonstige Überlappungen)
\subsection{Andere Probleme}
\subsubsection{Provozierte Schwierigkeiten}
\subsubsection{Erkennung und Auswirkung von PDF}\phantomsection\label{subsubsec:problems:pdfinclusion}