% note: commenting / keeping track of these paragraphs is being done in the associated github-issues (see project: @imagineMaggots's bachelors-thesis)
% note, too: the comments in this file are not necessary, really.
\section{Problemfälle}
\subsection{Herangehensweise}
\subsubsection*{Reihenfolge}
Die nachfolgende Auflistung verschiedener Fälle, welche Probleme gegenüber der \TeX{}-Syntax, bzw.\ innerhalb von \LaTeX{}-Dokumenten hervorrufen könnten, benötigt per se keine Reihenfolge, da sie möglichst alle unabhängig voneinander behoben sein sollen. Ein unbedachtes, zufälliges sequenzielles Nennen dieser könnte logische Lücken produzieren und damit ein Übersehen potentieller Fehler riskieren\pdfcomment{Der Satz wird definitiv noch überarbeitet.}. 
Deshalb wird eine Reihenfolge gewählt, welche nicht auf \LaTeX{}-Ebene beginnt, sondern sich so weit wie möglich dem Ursprung dieses Systemes nähert.
Von den bereits in \TeX{} auftretenden Problemen muss ein Weg in Richtung der auf dieser Software aufbauenden Technologien gebahnt werden. Da es sich der Gesamtheit der Quelltextdateien, auf welche ein Kompiliervorgang von \TeX{} zugreifen kann, immer um reine Textdateien handelt, werden spätere Beispiele nicht nach einzelnen Technologien betrachtet, sondern nach ihren Use-Cases (als relevante Systeme kämen hier zunächst \LaTeX{}, Bib\TeX{}, Ti\textit{k}Z und Weiterführende in Frage, welche teils andere Probleme nach sich ziehen). Eine Fehlerunterscheidung findet nach Funktionalität in einem Dokument (das nach dem Kompilieren entstehende, ausgehend von einer Technologie\pdfcomment{jedoch nicht zwingend von nur einem Quelltext, das wird die Überleitung}, beschrieben in~\ref{subsec:logicInDocuments}) und dessen virtuellem Pendant (mit Inhalten abhängig von anderen Dateien und Technologien, beschrieben in~\ref{subsec:logicOutsideDocuments}).% 
In einem einzelnen Beispiel wird zunächst von dem strukturellen Element ein Beispiel abgelitten und danach eine abstraktere Beschreibung dieser Fehlerquelle gefunden.% 
\subsubsection*{Logische Strukturen in Dokumenten}\phantomsection\label{subsec:logicInDocuments}
Beispiele in reellen Dokumenten sind nach den Strukturen sortiert, welche man in Dokumenten beliebiger Natur wiederfinden kann. Als kleinste Struktur würde man hier (abgesehen von einzelnen Worten und Sätzen) Paragraphen sehen, welche Abschnitte eines Dokumentes formen. Mehrere dieser Abschnitte ergeben einen größeren Abschnitt. Statt von einem \enquote{Überabschnitt} zu sprechen, wird daher die Formulierung umgekehrt. Ein Dokument ist zu Beginn als ein großer Abschnitt zu verstehen, welcher sich in verschiedene Unterabschnitte teilt, deren Namensgebung individuell sein kann. Hier wird (ähnlich wie bei:~\cite{texbook}) von Kapiteln, Abschnitten, Unterabschnitten und Paragraphen gesprochen, jedoch mögliche \enquote{Unterunterabschnitte} nicht als einzelne logische Struktur betrachtet, sondern als Unterabschnitt eines Unterabschnittes. 
Abschnitte selbst stellen aber nicht die einzige logische Struktur in einem Dokument dar, sondern es existieren auch andere, nicht-textliche Inhalte (die dem Kompilieren eines \TeX{}-Quellcodes das Aussehen des Dokumentes beeinflussen). 
Die gröbste Struktur eines Dokumentes entscheidet die Präambel von \TeX{}-Dokumenten, welche sich aus verschiedenen Formen von Befehlen zusammensetzt, von welchen nur begrenzt viele Inhalte übersetzt werden dürften (meist:\ keine).
Nach der Präambel, also im Teil, der \textit{tatsächliche} Inhalte im Dokument beschreiben soll, ist nach Befehlen (Kommandos) und Umgebungen zu klassifizieren, deren Funktion innerhalb des Quellcodes auch nach einem Übersetzen erhalten bleiben muss. Im Normalfall sei davon auszugehen, dass Kommandos innerhalb eines Dokumentes entweder bestimmte globale Parameter definieren (bspw.\ in der Präambel) oder in andere graphische Elemente (bspw.\ Formelzeichen, Stilisierung von Zeichenketten, \ldots) aufgelöst werden und Umgebungen Änderungen an größeren Teilen eines Abschnittes vornehmen\pdfcomment{beliebiger hierarchischen Höhe im DOM}. Einzelne Elemente der Präambel deuten jedoch darauf hin, welche Art von anderen Technologien (auf \TeX{}-aufbauend) in einem Dokument genutzt werden und wohingegen ein übersetztendes Programm die Suche nach zu übersetzenden Strings über das eigentliche Dokument (bzw.\ einen einzelnen Quellcode) hinaus ausweiten muss.

%\subsubsection*{Externe Abhängigkeiten virtueller Dokumente}
\phantomsection\label{subsec:logicOutsideDocuments}% Meint BibTeX, TikZ (wobei, eigentlich nicht mehr, es sei denn PGF, aber auch eigentlich nicht, es sei denn Graphik muss neu erzeugt werden?), ..., Pakete in general, und associated files
Ein Dokument (\TeX{} oder \LaTeX{}) kann/muss nicht nur mittels einem Quelltext beschrieben werden oder nur unter Nutzung eines \TeX{}-Compilers entstehen. Den eigentlichen (im Sinne:\ existierenden, mit Namen versehenen) Technologien abgesehen, müssen Dokumente teils aus Gründen der Lesbarkeit (des Quelltextes) oder logischen Reihenfolge von strukturellen Elementen Informationen in mehrere Dateien auslagern oder auf mehrere Dateien zugreifen, um gesamte Inhalte eines Dokumentes zu produzieren. 
% Voriger Satz schrieb: muss. Darauf muss ein kurzer Beweis folgen...
Insbesondere die Lesbarkeit von Quellcodes kann von einer solchen Nutzung von mehreren Quelltextdateien für ein \LaTeX{}-Dokument profitieren, da z.B.\ Graphiken (via Ti\textit{k}Z o.Ä.\ erstellt) oder Quellcodes (mittels bspw.\ \texttt{minted} oder \texttt{lstlistings} dargestellt) ansonsten Quelltexte (für das \TeX{}-Dokument) produzieren können, die abertausende Zeilen lang sind\pdfcomment{und persönlich werden mir selbst reine Fließtexte zu unübersichtlich. von dark-mode auf light-mode in meiner bevorzugten ide zu wechseln, hilft allerdings dabei}.
Andererseits kann (aus Sicht des \TeX{}-Parsers) keine Struktur in einem Dokument (ob übersetzt, ob unübersetzt) in einem bspw.\ Inhaltsverzeichnis gelistet sein, welches nach diesem Verzeichnis steht. Der Parser kann diese Struktur (beim Abarbeiten des Quelltextes von oben nach untern) nur sehen, nachdem er bereits den Befehl sah ein Inhaltsverzeichnis auszugeben. Die Informationen über die Inhalte dieses Inhaltsverzeichnis' erhält er logischerweise allerdings erst, nachdem er alle Teile des Dokumentes (dessen Beschreibung) gesehen hatte (und demnach in eine helfende Datei schreiben konnte, auf welche er beim nächsten Durchlaufen zugreifen kann).
% Überleitung fehlt

% Überleitung fehlt
% \subsubsection*{Weiteres}
Zudem sind ein paar zusätzliche, sprachliche und teils unlösbare Probleme gelistet, welche nicht unbedingt als Anforderungen der gegebenen Problemstellung zu verstehen sind und daher als \enquote{abweichend} zu verstehen sind, aus welchen sich aber spätere Erweiterungspotentiale zeigen könnten.

\subsubsection*{Struktur eines Beispiels}
Die Darstellung einzelner Beispiele erfolgt tabellarisch und demonstriert erst richtiges (bzw.\ zulässige) Verhalten und danach Fehlerhaftes (bzw.\ Unerwünschtes). Untiges Beispiel (Tabelle~\ref{tab:problems:example}) dient hierbei als einfaches Beispiel für Fehler, die sich bei einem imaginären Befehl \texttt{ink} zeigen könnten. Dieser soll einen String mit einer bestimmten Farbe hinterlegen und besitzt einen zusätzlichen optionalen Farbparameter. 
\enquote{Richtig} wäre es im originalen String nur das Wort in den geschwungenen Klammern zu übersetzen, da hierbei an keiner Stelle Information verloren geht und das Wort, nachdem es vom Deutschen ins Englische übersetzt wurde, weiterhin so wie vorgesehen hervorgehoben wird.
\enquote{Zulässige} Übersetzungen treten dann auf, wenn nur für die Formatierung (insofern hieraus keine weiteren Probleme entstehen) verloren geht. Im gegebenen Beispiel würde dann zwar die farbige Hinterlegung verloren gehen, das Wort allerdings trotzdem übersetzt werden und würde den Weg in ein Dokument finden, ohne einen sprachlichen Informationsverlust zu riskieren (für den Endnutzer/Leser).\footnote{Selbst bei weißer Schriftfarbe kann das Wort in einem PDF-Reader markiert und kopiert werden.}
\enquote{Unerwünscht} sind Fälle, in denen ein Übersetzen Fehler für die \TeX{}-Engine produziert. Übersetzt man hier z.B.\ \texttt{ink} nach \texttt{Tinte} könnte es sich bei Zweiterem wiederum um einen anderen Befehl handeln, das Wort \textit{Wort} einliest, aber eigentlich den alphanumerischen Wert von \textit{word} erwartet hätte.\footnote{$57_{16}+6f_{16} + 72_{16} + 74_{16} = 25\times 16^1 + 28 = 400$ statt:\ $77_{16}+6f_{16} + 72_{16} + 64_{16} = 26\times 16^1 + 28 = 416$. Wofür der Befehl \texttt{Tinte} einen/den Integer 416 benötigt, kann ich Ihnen allerdings nicht erläutern.} Ein Fehlschlagen des Befehls \texttt{Tinte} würde zwar einen Fehler für den \TeX{}-Parser produzieren, dieser wüsste dann aber, dass dieser Befehl bereits einmal fehlgeschlagen ist und eine neues Kompilieren verlangen, in welchem dieser Befehl und seine Optionen ignoriert werden, wodurch das Wort auch hier im Dokument landen würde. Man kann allerdings nicht bei jedem beliebigen \TeX{}-Befehl davon ausgehen, dass dieses Verhalten einheitlich auftreten wird. Hierbei existieren Fälle, welche dafür sorgen könnten, dass andere Wörter nun nicht mehr Teil eines Dokumentes werden könnten~\ref{}.% Meint das \include{clock} zu \include{Uhr} vs \include{clock.tex} zu \include{clock.tex} Beispiel.
\enquote{Fehlerhaftes} Verhalten beim Übersetzen von \TeX{}-Quelltextdateien führt zu einem Informationsverlust, da das zu übersetzende Wort entweder nicht mehr übersetzt wird oder nicht mehr im Dokument wiederzufinden ist. Sobald man beginnt mit mehreren Dateien ein einziges Dokument zu beschreiben, riskiert ein na\"\i ives Übersetzen nur von einem Quelltext ausgehend, dass aus unerwünschten Fehlern innerhalb von einem Dokument fehlerhaftes Verhalten für das entstehende Produkt (meint:\ die kompilierte PDF) entsteht. 

Abstrahiert man von diesem detaillierterem Beispiel, so sind \enquote{richtige} Übersetzungen frei von Informationsverlust, \enquote{zulässige} Übersetzungen nur dazu fähig Informationen für die graphische Aufbereitung (allerdings nicht den sprachlichen Inhalten) zu entwenden, \enquote{unerwünschte} Übersetzungen dazu in Lage Informationen verbergen können und \enquote{falsche} Übersetzungen fehlende sprachliche Inhalte innerhalb eines Dokumentes, sowie fehlende Übersetzungen dieser. Nicht jede Gruppe von Beispielen führt dazu, dass alle benannten Kategorien auftreten.

% Mit wenigen Ausnahmen lassen sich in vielen Beispielen größere Strukturen innerhalb des Quelltextes erkennen, welche musterhaft geschildert werden.

\newpage


\begin{table}[h!tb]
    \centering
    \begin{tabularx}{\textwidth}{X X}
        \toprule
            English & Mögliche Übersetzung\\
        \midrule
            Richtiges Verhalten & \\[-13px]
            \commoncode{Original}{../examples/example/original.tex} & \commoncode{Beispielübersetzung}{../examples/example/ideal.tex}\\[1em]
        \midrule
            Zulässiges Verhalten & \\[-13px]
            \commoncode{Original}{../examples/example/original.tex} & \commoncode{Beispielübersetzung}{../examples/example/okay.tex}\\[1em]
        \midrule
            Unerwünschtes Verhalten & \\[-13px]
            \commoncode{Original}{../examples/example/original.tex} & \commoncode{Beispielübersetzung}{../examples/example/problematic.tex}\\[1em]
        \midrule
            Falsches Verhalten & \\[-13px]
            \commoncode{Original}{../examples/example/original.tex} & \commoncode{Beispielübersetzung}{../examples/example/bad.tex}\\[-1em]
        \bottomrule
    \end{tabularx}
    \caption{Abstrakte Struktur der folgenden Beispiele}\label{tab:problems:example}
\end{table}

\newpage

\subsection{Elemente in einem \TeX{}-Quellcode}
\subsubsection{Die Präambel}
Der unsichtbare Header eines \TeX{}-Dokumentes zeigt an (zunächst) wenigen, sehr spezifischen Stellen eine Schwierigkeit auf. In dieser sind meistenfalls Informationen enthalten, welche nicht zu übersetzen sind, da sie z.B.\ Parameter für dokumentenweite (globale) Einstellungen setzen. Die Art und Weise \textit{diese} zu setzen ist allerdings genauso behandelbar, wie einige folgende Kommandos (in Abschnitt~\ref{}) und wäre daher eigentlich nicht von weiterem Interesse, zeigt sich allerdings sehr geeignet dazu, einige triviale Fehlerquellen abzudecken (bzw.\ ungeeignete Ansätze). So kann man sich nicht immer gewiss sein, man einzelne Zeilen gesondert auswerten kann, da z.B.\ Pakete wie \texttt{hyperref} es in ihren Optionen (\texttt{hypersetup}) erlauben einige Einstellungen durch Zeilenbrüche voneinander getrennt (und übersichtlicher) darzustellen.% Beispiel einfügen, Hyperref denkbar, existiert bereits.
Genauso sind allerdings auch einzelne Zeilen nicht direkt als \enquote{insgesamt \TeX{} syntaktisch} zu betrachten, nur weil ein syntaktisches Element von \TeX{} innerhalb dieses vorliegt.% Beispiel einzufügen


\subsubsection{Kommandos}
%%% Paragraph noch zusammenführen mit nachfolgendem

%\subparagraph*{Ohne}
Kommandos selbst sind in der reinen \TeX{}-Syntax durch ein Backslash \verb|\| gekennzeichnet. Die Frage, ob das Wort, das einem Kommando folgt, übersetzt werden sollte, oder nicht, bringt Konflikte. Beispielsweise müsste innerhalb von Auflistungen der nach einem \verb|\item| folgende String übersetzt werden. Andererseits wäre in einer Definition mit \verb|\def| vorerst davon abzusehen den folgenden String zu übersetzen, da dieser ein neues Makro/Kommando/etc.\ definieren zu sucht. Demnach darf \verb|\def\hello| nicht zu: \verb|\def\hallo| werden (es sei denn alle folgenden \verb|\hello| werden auch übersetzt).\footnote{Da aber, wie bereits bekannt, immer eine Wahrscheinlichkeit, so gering sie auch sein mag, besteht, dass an einer nicht vorgesehenen Stelle eine Übersetzung auftreten \textit{könnte}, kann man sich nicht auf die vorherige Aussage verlassen}.

% Muster:
Innerhalb des Quelltextes können also Strukturen auftreten, welche direkt folgende Zeichenketten als syntaktisches Element zur Dokumentenbeschreibung benötigen. Da hierbei beide Fälle auftreten könnten und nicht an dem Kommando selbst erkennbar sind, muss hierbei nach Art des Kommandos selbst abgewogen werden\pdfcomment{um dieses auf eine begrenzte Zahl an \TeX{} Primitiven zurückzuführen, welche darauf hindeuten können, ob übersetzt werden soll, oder nicht. (An dieser Stelle noch zu vorgreifend, daher der Kommentar)}. Demnach genügt es nicht nur reine Befehle anhand des \verb|\| zu erkennen oder als festes Maß dafür zu nehmen, ob ein folgender String (der kein weiteres Muster aufweist) zu übersetzen ist, oder nicht.

\paragraph*{Klammern für Parameter}
Auf viele Kommandos (Makros) folgen ein Paar geschwungener Klammern. Die Inhalte dieser sind in einigen Fällen zu übersetzen und in Anderen nicht, aber insgesamt als eine Art Parameter für dieses Kommando zu verstehen. 

Beispielsweise wäre das Kommando \verb|\paragraph{This is a string of many characters}| eines, in welchem die Inhalte der Klammern übersetzt werden sollen und das Kommando \verb|\usepackage{geometry}| ein Beispiel, in welchem das Übersetzen von \verb|geometry| zu \verb|Geometrie| dazu führen würde, dass das entsprechende Paket nicht im Quelltext verwendet wird. Diese Art von Fällen, welche der Übersetzung ausgeschlossen werden soll, ist besonders kritisch, wenn mit mehreren Dateien gearbeitet wird und so würde ein Übersetzen von z.B.\ \verb|\include{clock.tex}| (oder \verb|\include{clock}|) zu \verb|\include{Uhr.tex}| (oder:\ \verb|\include{Uhr}|) dazu führen, dass größere Teile eines Dokumentes komplett ausgeschlossen werden. 

% Muster:
Klammern dieser Art genügen also alleine noch nicht dafür eine Aussage darüber zu treffen, ob innenliegende Strings übersetzt werden dürfen, allerdings diese Zeichen (meint:\ die geschwungenen Klammern) kein Teil direkter Teil einer menschlichen Sprache und man würde davon meist erwarten, dass diese Klammern im Normalfall ein Indikator dafür sind, dass nicht übersetzt werden soll und in bestimmbaren Ausnahmefällen (bspw.\ \verb|\section{}|,\verb|\paragraph*{}|, \verb|\chapter{}|, \verb|\title{}|, etc.) von Interesse für ein Übersetzen sind. 

%%% \label{key} und \ref{key}, bzw. deren keys dürfte man prinzipiell auch alle übersetzen, sollten die keys persistent im Dokument bestehen bleiben und entsprechende Referenzierungen richtig im Dokument angelangen.
%%% Da dies allerdings ein unnötiges Fehlerrisiko produziert, sollte davon abgesehen werden.

\paragraph*{Optionen}\par
\subparagraph*{Details:}\par
Optionen sind, neben Parametern, eine weitere Möglichkeit einem Kommando in \TeX{} zusätzliche Informationen mitzuliefern, z.B.\ für zusätzliche Formatierung. Aus logischer Reihenfolge müssten diese Optionen \textit{eigentlich} immer vor der Zeichenkette, welche formatiert werden soll, stehen, da erst nach Einlesen der Parameter die Information dieser erkannt und auf den folgenden String angewendet werden kann.% TIKZ IST MAL WIEDER SCHULD
Diese Aussage wäre logisch, gilt allerdings nicht einheitlich (bspw.\ bei der Verwendung von Ti\textit{k}Z-Bibliotheken). Optionen in eckigen Klammern insgesamt einer Übersetzung zu entziehen, verhindert allerdings die Möglichkeit, dass die Optionen einiger Befehle selbst ein String sein könnten, welcher im Dokument gedruckt und somit übersetzt werden muss.  
\subparagraph*{Beispiele:}\par
% Hier brauch ich dann "größere" Beispiele (im Sinne: echter Platz, ab hier hübschere Formatierung)
\subparagraph*{Muster:}\par
Die hier anfallenden Fehler äußern sich in sechs Permutationen, in welchen immer nur ein String (\texttt{translatable}) übersetzt werden darf und ein weiterer String  (\texttt{non-translatable}) nicht.
Sie zeigen sich in 
\verb|\command[non-translatable] translatable|, \pdfcomment{bspw.\ \marginpar[translatable] text, benötigt logischerweise den Platz dafür}  
\verb|\command[translatable] translatable| \pdfcomment{bspw.\ \item[translatable] text} 
\verb|\command[non-translatable]{non-translatable}|, 
\verb|\command[non-translatable]{translatable}|, 
\verb|\command[translatable]{non-translatable}|, oder
\verb|\command[translatable]{translatable}|
und sind anhand spezifischer Kommandos erkennbar.

\subparagraph*{Problem:}% Problem in Problemen suggeriert: Oh oh
\verb|\newcommand{\appendstring}[3][hello][At]{#2 #3 #1} \appendstring{world:}| ist eine Möglichkeit einen String (\newcommand{\appendstring}[3][hello][At]{#2 #3 #1} \appendstring{world:}) zu erzeugen. Jedoch ist mit dieser Definition auch der String:\ \appendstring{now}{Goodbye}{for} in der Form \verb|\appendstring{now}{Goodbye}{for}| produzierbar. Diesem Muster entsprechend könnten (theoretisch) unendlich viele eckige Klammern zu übersetzende Strings beinhalten. Dies wäre zunächst unproblematisch unter der Vorannahme, dass alle diese Strings zu übersetzen sind. Was allerdings, wenn eine Definition der Form \verb|\newcommand{\addandappend}[3][1][2]{#1+#2 ist #3 #1+#2}| vorliegt, in welcher für \verb|#3| ein String (hier denkbar:\ gleich) erwartet wird? Solche Erwartungen sind nicht immer vorhersagbar und können dadurch nur schwer beschrieben werden (sind allerdings deterministisch nachvollziehbar, da die \TeX{}-Engine solche Makros schließlich auch interpretieren und rendern kann).% ? hoffentlich

\subparagraph*{Unvorhersagbarkeiten}% -> spezielle fehler
Eigene, definierte Makros erlauben für eine quasi-beliebige Zahl an Stellen, innerhalb welcher Strings erwartet werden könnten. So könnte beispielsweise ein Befehl definiert werden, welcher 7 Eingabeparameter erwartet:\ \verb|\def\whatev a#1a#2a#3a#4a#5a#6a#7{This #1 is #2 to #3 be #4 troublesome. Hence we will only print one hundred now #5#6#7}| bei einem Aufruf der Form \verb|\whatev amacroagoingatoaverya1a0a0| erwarten, dass sowohl alle menschensprachlichen Zeichenketten innerhalb der Definition (\textit{This is to be troublesome. Hence we will only print one hundred now}) als auch außerhalb dieser (\textit{macro going to very}) übersetzt werden und einen schlüssigen Satz bilden.

\paragraph*{Strukturen}
\subparagraph*{Umgebungen}
Größere Strukturen in der \TeX{}-Syntax zeigen sich oftmals in sog.\ Umgebungen auf und stellen gerne/oft Graphiken\pdfcomment{Tabellen sind eine Untergattung von Graphiken} dar (wie bspw.\ in Ti\textit{k}Z), müssen dies allerdings nicht zwingend. Genauso wäre zu erwarten, dass größere Texte innerhalb dieser Umgebungen reinen Text-Formatierungen obliegen und damit gänzlich zu übersetzen wären. Allerdings können verschiedene Umgebungen auch eigene Syntaktik tragen (bspw.\ Tabellen), wodurch sich innerhalb von solchen Umgebungen sowohl zu übersetzende Strings, als auch zu Erhaltende (= nicht zu übersetzende Strings) verbergen können. 

Insbesondere Problematisch wird die Ermittlung solcher bei Umgebungen, welche nicht mittels entsprechender \texttt{begin} und \texttt{end} Kommandos betreten/verlassen werden, sondern auf welche mittels eigener Symbolik ein access gewährt werden kann. Innerhalb reinen \TeX{}'s sind diese \pdfcomment{glücklicherweise}auf wenige Zeichenketten beschränkt und so können nur ein/zwei Dollar-Zeichen (\$,\$\$) oder \verb|\(|,\verb|\)|, \verb|\[|, \verb|\]|auf den/das Beginn/Ende einer ein-/ mehrzeiligen mathematischen Umgebung hindeuten. Darüberhinaus erlauben gerade diese Umgebungen den Wechsel in \textit{normale} Umgebungen (also hinein in die Textumgebung des eigentlichen Dokumentes, aus welcher solche mathematisch Umgebungen zu fliehen suchten) innerhalb welcher wieder in beschriebene mathematische Umgebungen gewechselt werden kann. Solange auf jeden Wechsel \textit{in} eine solche Umgebung ein Wechsel \textit{aus} einer solchen Umgebung heraus erfolgt, kann \TeX{} dies interpretieren und alle rein textlichen Strings \textit{sollten} übersetzt werden.

\subparagraph*{Pakete}
Pakete sind ihrerseits keine, innerhalb eines Quelltextes direkt vorliegende, Struktur, aber besitzen die Fähigkeit, dass sie Zeichenketten erwarten/tragen könnten, welche es zu übersetzen gilt. Zwar erfolgt die Einbindung solcher immer auf gleicherlei art und weise, aber hier einzelne befehle sind... übersetzbar... morgen anzupassen, weil ich kann nicht mehr
% https://www.overleaf.com/learn/latex/Writing_your_own_package
\subparagraph*{Klassen}
% https://www.overleaf.com/learn/latex/Writing_your_own_class#General_structure         scheint gleich zu packages. wobei, man kann hier in sowas wie \title (oder \@title) eigene Strings integrieren? glaub ich?


%%%%%%%%%%%%%%%%%%%%%%%%%%%%%%%%%%%%%%%%%%%%%%%%%%%%%%%%%%%%%%%%%%%%%%%%%
%%%%%%%%%%%%%%%% Not according to github-issue-structure %%%%%%%%%%%%%%%%
%%%%%%%%%%%%%%%%%%%%%%%%%%%%%%%%%%%%%%%%%%%%%%%%%%%%%%%%%%%%%%%%%%%%%%%%%

% \command translatable <---> \item wasauchimmer
% \command non-translatable <---> \def \string sumn... / consider talking about \renewcommand here?


\begin{comment}
\subsubsection{Paragraphen und Abschnitte}
% Meint: \command{translatable}

Der zuvor etablierten logischen Struktur der Beispiellistung folgend, würde man zunächst einen Abschnitt oder Paragraphen erwarten. Technisch gesehen äußern sich die Beispiele allerdings alle sehr ähnlich. Hier erwartet man immer einen Befehl, welcher nicht übersetzt werden darf (da er Teil der logischen Struktur des Dokumentes ist) und einen Wert, welcher diesem Befehl übergeben ist und übersetzt werden soll. Das präsentierte Beispiel~\ref{tab:problems:sections} zeigt hier wohlmöglich auftretende Fälle. Erstrebenswert ist hierbei \textbf{nur} die Übersetzung des in geschwungenen Klammern stehenden Strings, unerwünscht das Übersetzen von sowohl Befehl, als auch den umklammerten Teilen und falsch ein Übersetzen von Befehl ohne den in Klammern stehenden Inhalten.% ich hasse abstraktes denken so fucking sehr
\begin{table}[h!tb]
    \centering
    \begin{tabularx}{\textwidth}{X X}
        \toprule
            English & Mögliche Übersetzung\\
        \midrule
            Richtiges Verhalten & \\[-13px]
            \commoncode{Original}{../examples/sections/original.tex} & \commoncode{Beispielübersetzung}{../examples/sections/ideal.tex}\\[1em]
        \midrule
            Unerwünschtes Verhalten & \\[-13px]
            \commoncode{Original}{../examples/sections/original.tex} & \commoncode{Beispielübersetzung}{../examples/sections/problematic.tex}\\[1em]
        \midrule
            Falsches Verhalten & \\[-13px]
            \commoncode{Original}{../examples/sections/original.tex} & \commoncode{Beispielübersetzung}{../examples/sections/bad.tex}\\[-1em]
        \bottomrule
    \end{tabularx}
    \caption{Abstrakte Struktur der folgenden Beispiele}\label{tab:problems:sections}
\end{table}

Abstrahiertes Muster: \verb|\command{translatable-string}|.


\subsubsection{Kommandos}
\paragraph{Ohne Optionen}
% Meint: \command{non-translatable}

Den Implikationen, welche Referenzen auf Abschnitte eines Dokumentes\pdfcomment{weiter verfolgt in späterem Abschnitt} mit sich bringen, abgesehen, müssen jegliche Fälle abgedeckt werden, in welchen möglicherweise übersetzbare Strings mit Sonderzeichen vermischt stehen könnten. % Hier auch das erste label beispiel denkbar. Siehe: irgendwo im git. ich muss aufräumen und geistig klarkommen. hilfe
Verweise auf vorangegangene oder folgende Paragraphen, Abschnitte oder Inhalte (via \texttt{ref} auf ein \texttt{label}) müssen einer einheitlichen Übersetzung obliegen, in welcher denkbare Label-Tags mit korrespondierenden Referenzierungen auch nach Übersetzung korrespondierend bleiben. % BEISPIEL NOCH ANPASSEN.

\begin{table}[h!tb]
    \centering
    \begin{tabularx}{\textwidth}{X X}
        \toprule
            English & Mögliche Übersetzung\\
        \midrule
            Richtiges Verhalten & \\[-13px]
            \commoncode{Original}{../examples/references/original.tex} & \commoncode{Beispielübersetzung}{../examples/references/ideal.tex}\\[1em]
        \midrule
            Zulässiges Verhalten & \\[-13px]
            \commoncode{Original}{../examples/references/original.tex} & \commoncode{Beispielübersetzung}{../examples/references/okay.tex}\\[1em]
        \midrule
            Unerwünschtes Verhalten & \\[-13px]
            \commoncode{Original}{../examples/references/original.tex} & \commoncode{Beispielübersetzung}{../examples/references/problematic.tex}\\[1em]
        \midrule
            Falsches Verhalten & \\[-13px]
            \commoncode{Original}{../examples/references/original.tex} & \commoncode{Beispielübersetzung}{../examples/references/bad.tex}\\[-1em]
        \bottomrule
    \end{tabularx}
    \caption{Abstrakte Struktur der folgenden Beispiele}\label{tab:problems:referencesInDoc}
\end{table}

Abstrahiertes Muster: \verb|\command{non-translatable-string}|.

\newpage

\paragraph{Kommandos mit Optionen}
\subparagraph{Ohne Übersetzungen} % Ohne bedeutet: im String darf rein gar nichts übersetzt werden
% Meint: \command[non-translatable]{non-translatable}

Abstrahiertes Muster: \verb|\command[non-translatable]{non-translatable-string}|.

\subparagraph{Vorhersagbare Übersetzungen} % Vorhersehbar bedeuted:


\subsubsection{Mathematische Theoreme}
% Meint: \command[translatable]{non-translatable-parameter} oder \command{non-translatable}[translatable]

\newpage



\newpage

\subsubsection{Inhaltsverzeichnisse}



\newpage

\subsubsection{Tabellen}

\newpage

\subsubsection{Auflistung von Tabellen und Abbildungen}
iwtkms
\newpage

\end{comment}