\documentclass{article}
\usepackage{ifthen}

\begin{document}
\newcommand{\weather}[2][sunny]
{
    Heute war es 
    \ifthenelse{\equal{#1}{sunny}} 
    {sonnig.} 
    {nicht sonnig. Hoffentlich ist es morgen nach dem Regen wieder 
        \ifthenelse{\equal{#2}{rainy}} 
        {
            regnerisch, da die Wälder zu trocken sind.
        }
        {
            wieder sonnig werden.
        }
    };
}

\weather{sunny}\\\\
% produziert: "Heute war es sonnig."
\weather{rainy}\\\\
% produziert auch: "Heute war es sonnig." Da hier erwartet wird, dass Parameter 2 festgelegt wird, da für Par. 1 bereits ein Wert existiert.
\weather[sunny]{rainy}\\\\
% produziert auch: "Heute war es sonnig." Da 'sunny' = 'sunny'
\weather[rainy]{sunny}\\\\
% produziert: "Heute war es nicht sonnig. Hoffentlich ist es morgen nach dem Regen wieder wieder sonnig werden."
\weather{rainy}{rainy}\\\\
% produziert: "Heute war es sonnig.; rainy" Da, wie zuvor, 'rainy' als Parameter 2 aufgefasst wird. Das darauffolgende Wort wird also nicht als Teil des Befehls aufgefasst.

\end{document}